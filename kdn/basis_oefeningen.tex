\documentclass{ximera} 

\input{preamble.tex}

\begin{document}
\phantomsection
\addcontentsline{toc}{section}{Oefeningen}
\section*{Oefeningen reeks 1}

\begin{Oefening}[\bf \ref{antw1.1}.]\setcounter{enumi}{1} 
\hypertarget{oef1.1}{Welke} uitdrukkingen stellen na vereenvoudiging een eenterm in $x$ voor? 
\begin{multicols}{2}
\begin{enumerate}%[(a)]
\item
$x^{17}$
\item
$\D -\frac{17}{3}\,x^2$
\item
$x^{-1}$
\item
$\D x + 5\cdot \frac{x}{3}$
\item
$x + x^2$
\item
$\D \frac{6}{5}\cdot\frac{3}{8}$
\item
$(2x-3)(2x+3)+9$
\item
$\sqrt{2}\,x^4 + \sqrt{3}\,x^4$
\item
$\sqrt{x}$
\item
$\sqrt{x^2}$
\end{enumerate}
\end{multicols}
\end{Oefening}

\begin{Oefening}[\bf \ref{antw1.2}.]\setcounter{enumi}{2}
\hypertarget{oef1.2}{Schrijf} de volgende veeltermen uit.
\begin{enumerate}%[(a)]
\item
$\D A(x) = \sum_{i=0}^3 ix^i$
\item
$\D B(x) = \sum_{n=0}^3 \frac{x^{2n+1}}{n+1}$
\item
$\D C(x) = \sum_{i=2}^4 (-x)^i$
\end{enumerate}
\end{Oefening}

\begin{Oefening}[\bf \ref{antw1.3}.]\setcounter{enumi}{3}
\hypertarget{oef1.3}{Welke} uitdrukkingen stellen na vereenvoudiging een veelterm in $x$ voor? 
\begin{multicols}{2}
\begin{enumerate}%[(a)]
\item
$5x^4-3x^3+2x^2+x-1$
\item
$\D \frac{1}{2}\,x^3 + 3x^2-\pi\,x+2$
\item
$\D \frac{1}{2}+\frac{1}{3}$
\item
$x$
\item
$\sqrt{2}\,x + \sqrt{3}$
\item
$\D \frac{1-x^2}{2}$ 
\item
$\D 1-\frac{2}{x^2}$
\item
$(x-4)(x^2+2x+3)$
\item
$(-2x^2+5x-1)^{9}$
\item
$\sqrt{9x^4+25}$
\item
$\sqrt{9+25}$
\item
$\sqrt{x^4}$
\end{enumerate}
\end{multicols}
\end{Oefening}

\begin{Oefening} % oef 4
Bepaal telkens de graad, de hoogstegraadsco\"effici\"ent en de constante term van de veelterm.
\begin{multicols}{2}
\begin{enumerate}%[(a)]
\item
$x^4-3x^2+2x-1$
\item
$\sqrt{5}\,x^3-2x$
\item
$\D x+\frac{1}{3}\,x^3+x^2$
\item
$5x+2$
\item
$2$
\item
$(2x^3-7)^2$
\item
$(-5x+7)(3x^2-5x+8)$
\item
$(x-2)^5$
\end{enumerate}
\end{multicols}
\end{Oefening}

\section*{Oefeningen reeks 2}

\begin{Oefening}[\bf \ref{antw1.5}.]\setcounter{enumi}{5}
\hypertarget{oef1.5}{Bepaal} telkens de graad, de hoogstegraadsco\"effici\"ent en de constante term van de veelterm zonder deze veelterm uit te werken of te vereenvoudigen. 
\begin{multicols}{2}
\begin{enumerate}%[(a)]
\item
$(x^2-3)(x^3+5)$
\item
$(2x+3)(x^5+x^2+1)$
\item
$\D \left(\frac{2}{9}\,x^3+x^2-1\right)\left(3x^4+3x+\frac{3}{7}\right)$
\item
$(-x^3+5)(3x^2-7)(5x+2)$
\item
$(5x-3)^3+27$
\item
$(4x^2+5)^3-x^5-x^4-x^3-x^2-x$
\end{enumerate}
\end{multicols}
\end{Oefening}

\pagebreak

\begin{Oefening} % oef 6
Werk de volgende veeltermen uit, en vereenvoudig telkens zoveel mogelijk. Geef daarna telkens de graad, de constante term en de hoogstegraadsco\"effici\"ent.
\begin{multicols}{2}
\begin{enumerate}%[(a)]
\item
$x(x^3-3x+\sqrt{2})$
\item
$x+2x^3-1-2(x^7+8x^8+x^3)$
\item
$(x+1)(x-3)$
\item
$(2x-3)(-6x-1)$
\item
$(x^2+1)(x^3-x+3)$
\item
$(3x^2-6+2x)(x+2)^2$
\item
$-(\sqrt{3}+4x)(-\sqrt{3}+4x)+3x^2-1$
\item
$\D \Bigl(-x^6 + \frac{1}{3}x\Bigr)(2x-1) - 3x^2\Bigl(-x^5-\frac{1}{9}\Bigr)$
\end{enumerate}
\end{multicols} 
\end{Oefening}

\begin{Oefening}[\bf \ref{antw1.7}.]\setcounter{enumi}{7}
\hypertarget{oef1.7}{Werk} telkens uit, vereenvoudig en bepaal de graad. 
\begin{multicols}{2}
\begin{enumerate}%[(a)]
\item
$2x-(x^3-x^2-x-1)$
\item
$x(x^2-3x+2)-2x(x-1)$
\item
$\left(\sqrt{2}\,x+\sqrt{3}\right)\sqrt{5}$
\item
$-3(3x^3-x+2)(x-5)$
\item
$\D \frac{x^3-(x^2+1)(x+4)}{5}$
\item
$(x+2)(x-2)-(x-2)^2$
\item
$17-4x^2-(-2x+3)(-2x-3)-8$
\item
$\D \left(\frac{5}{6}\,x+\frac{1}{3}\right)(3x-2)(3x+2)$
\end{enumerate}
\end{multicols}
\end{Oefening}

\begin{Oefening}[\bf \ref{antw1.8}.]\setcounter{enumi}{8}
\hypertarget{oef1.8}{Bepaal} telkens de getalwaarde van de veelterm in de gegeven $x$-waarde. Gebruik de correcte notatie. Geef aan of die $x$-waarde ook een nulwaarde is. 
\begin{enumerate}%[(a)]
\item
$A(x) = x^3-3x^2-10x+24$ \quad in $x = 2$
\item
$B(x) = x^4-7x^2-6x+1$ \quad in $x = -1$
\item
$P(x) = x^4-4x^3-4x^2-4x-5$ \quad in $x = 5$
\item
$S(x) = x^3-x^2+x-1$ \quad in $x = 0$
\item
$C(x) = x^4-4$ \quad in $x = \sqrt{2}$
\item
$D(x) = 2x^3 + 3x^2 - 11x - 6$ \quad in $\D x = -\frac{1}{2}$
\end{enumerate}
\end{Oefening}

\begin{Oefening}[\bf \ref{antw1.9}.]\setcounter{enumi}{9} 
\hypertarget{oef1.9}{Bepaal} telkens de waarde(n) van de re\"ele parameters waarvoor de veeltermen $A(x)$ en $B(x)$ gelijk zijn.
\begin{enumerate}%[(a)]
\item
$A(x) = (ax+7)x$ en $B(x) = 7x-3x^2$ %waarbij $a \in \RR$
\item
$A(x) = (ax+b)(3x-2)$ en $B(x) = 3x^2+6-11x$ %waarbij $a,b \in \RR$
\end{enumerate}
\end{Oefening}

\begin{Oefening}[\bf \ref{antw1.10}.]\setcounter{enumi}{10} 
\hypertarget{oef1.10}{Bepaal} telkens de exacte waarde(n) van de re\"ele parameters waarvoor de veelterm voldoet aan de voorwaarde. 
\begin{enumerate}%[(a)]
\item
$A(x) = -x^3-8ax-3$  %waarbij $a \in \RR$ 
met $A(1) = 8$ 
\item
$G(x) = (c+1)x^2+3x-\sqrt{2}\,c$ %waarbij $c \in \RR$ en 
met nulwaarde $-1$
\end{enumerate}
\end{Oefening}

\begin{Oefening}[\bf \ref{antw1.11}.]\setcounter{enumi}{11} 
\hypertarget{oef1.11}{Gegeven} zijn de veeltermen 
\[
A(x) = -3x^4+2x^6-5x^6+6x^4+21x^4-5+12x^2+5 \quad \text{ en } \quad B(x) = -5x^2+2x-6-2x+6.
\]
\begin{enumerate}%[(a)]
\item
Vereenvoudig de veeltermen $A(x)$ en $B(x)$.
\item
Geef de graad van de veeltermen $A(x)$ en $B(x)$. Hanteer de correcte notatie.
\item
Bepaal $A(-2)$ en $B(\sqrt{3})$. 
\item
Werk uit, vereenvoudig en bepaal de graad van de som en het product van de veeltermen $A(x)$ en $B(x)$. 
\end{enumerate}
\end{Oefening}

\clearpage

\section*{Oefeningen reeks 3}

\begin{Oefening}[\bf \ref{antw1.12}.]\setcounter{enumi}{12}
\hypertarget{oef1.12}{Gegeven} zijn de veeltermen 
\[
A(x) = 2x^2 + 5x + 13, \quad B(x) =  x^3-3x-4  \quad \text{ en } \quad C(x) = xA(x) + cB(x)
\]
waarbij $c \in \RR$.
\begin{enumerate}%[(a)]
\item
Bepaal $A\bigl(B(3)\bigr)$.
\item
Bepaal de waarde(n) van de parameter $c$ waarvoor $\gr C(x) = 2$.
\end{enumerate}
\end{Oefening}

\begin{Oefening}[\bf \ref{antw1.13}.]\setcounter{enumi}{13} 
\hypertarget{oef1.13}{Bepaal} een kwadratische veelterm $A(x)$ waarvoor $A(0) = 2$ en waarbij $-2$ en $1$ nulwaarden zijn.
\end{Oefening}

\begin{Oefening}[\bf \ref{antw1.14}.]\setcounter{enumi}{14} 
\hypertarget{oef1.14}{Bepaal} een kubische veelterm $A(x)$ zonder constante term waarvoor
\[
A(x) = A(x-1) + x(x-1).
\]
\end{Oefening}

\begin{Oefening}[\bf \ref{antw1.15}.]\setcounter{enumi}{15} 
\hypertarget{oef1.15}{Als} je weet dat er precies \'e\'en veelterm is waarvan de derde macht gelijk is aan 
\[
P(x) = 8x^6 + 36x^5 + 66x^4 + 63x^3 + 33x^2 + 9x + 1,
\]
bepaal dan deze veelterm.
\end{Oefening}

\begin{Oefening}[\bf \ref{antw1.16}.]\setcounter{enumi}{16} 
\hypertarget{oef1.16}{Gegeven} is de veelterm
\[
A(x) = x^4 - 4x^3 + ax^2 + 2x + b
\]
waarbij $a,b \in \RR$. Bepaal de waarde(n) van de parameters $a$ en $b$ waarvoor $A(x)$ het kwadraat is van een veelterm. 
\end{Oefening}

\begin{Oefening}[\bf \ref{antw1.17}.]\setcounter{enumi}{17} 
\hypertarget{oef1.17}{Beschouw} de veelterm $P(x) = (3x - 1)^{12}$. Bereken algebra\"isch de som van de co\"effici\"enten. 
\end{Oefening}

\begin{Oefening}[\bf \ref{antw1.18}.]\setcounter{enumi}{18} 
\hypertarget{oef1.18}{Als} $a,b,c,d$ re\"ele getallen zijn waarvoor geldt dat
\[
ax^3 + bx^2 + cx + d = (x^2-6x-8)^{10}\cdot(4x+2)^5 - (x+1)^8\cdot(x^2+5x+4)^{27}
\]
bepaal dan algebra\"isch de waarde van $a-b+c-d$.
\end{Oefening}

\begin{Oefening} % oef 19
Van een rechthoekig stuk karton met afmetingen $20\:\text{cm}$ op $10\:\text{cm}$ knippen we in elke hoek een vierkant met zijde $x$ weg. Nadien plooien we het karton langs de stippellijnen, om zo een doos zonder deksel rechts te verkrijgen. 
\begin{enumerate}%[(a)]
\item 
Schrijf een veelterm $V(x)$ op dat het volume van de doos weergeeft. 
\item
Geef de graad en de constante term van deze veelterm, en verklaar jouw antwoord zowel algebra\"isch als met behulp van de context van deze oefening.
\item
Geef alle nulwaarden van deze veelterm, en verklaar jouw antwoord zowel algebra\"isch als met behulp van de context van deze oefening.
\end{enumerate}

\medskip

%%%%%%%%%%%%%%%%%%%%%%%%%%%%%%%%%%%%%%%%%%%%%%%%%%%%%%%%%%%%%%%%%%%%%%%%%
\begin{center}
\psset{xunit=1cm,yunit=1cm}
\begin{pspicture}(0,-1)(6,4)% co linksonder, co rechtsboven
\psline[](0,0)(6,0)(6,3)(0,3)(0,0)

\psline[linestyle=dashed](0,0.5)(6,0.5)
\psline[linestyle=dashed](0,2.5)(6,2.5)
\psline[linestyle=dashed](0.5,0)(0.5,3)
\psline[linestyle=dashed](5.5,0)(5.5,3)

\psline[linecolor=blue]{<->}(0,3.5)(6,3.5)
\uput[u](3,3.5){\color{blue}\SI{20}{\cm}}

\psline[linecolor=blue]{<->}(-0.5,0)(-0.5,3)
\uput[l](-0.5,1.5){\color{blue}\SI{10}{\cm}}

\psline[linecolor=red]{<->}(0,-0.5)(0.5,-0.5)
\uput[d](0.25,-0.5){\color{red}$x$}

\psline[linecolor=red]{<->}(5.5,-0.5)(6,-0.5)
\uput[d](5.75,-0.5){\color{red}$x$}

\psline[linecolor=red]{<->}(6.5,0)(6.5,0.5)
\uput[r](6.5,0.25){\color{red}$x$}

\psline[linecolor=red]{<->}(6.5,2.5)(6.5,3)
\uput[r](6.5,2.75){\color{red}$x$}
\end{pspicture}
\end{center}
%%%%%%%%%%%%%%%%%%%%%%%%%%%%%%%%%%%%%%%%%%%%%%%%%%%%%%%%%%%%%%%%%%%%%%%%%
\end{Oefening}

\clearpage

\begin{Uitbreiding}
\begin{Oefening}% oef 20
\label{oefgraadnulveelterm}
{\bf (min oneindig en de graad van de nulveelterm)} 
Om aan de nulveelterm ook een graad te kunnen toekennen, breiden we de verzameling van de re\"ele getallen uit met een element, voorgesteld door het symbool $- \infty$, lees als: min oneindig. Het element $-\infty$ is dus geen (re\"eel) getal. Men spreekt ook wel van het \xmemph{oneigenlijk getal} $-\infty$. Ook de orde en de optelling in $\RR$ worden uitgebreid door de volgende definities (waarbij $a \in \RR$):
\[
\begin{aligned}
\\[-0.5cm]
-\infty & < a \\
(-\infty) + a & = -\infty \\
a + (-\infty) & = -\infty \\
(-\infty) + (-\infty) & = (-\infty).
\end{aligned}
\]
De \underline{graad van de nulveelterm}\index{graad}\index{veelterm!graad} is nu per definitie gelijk aan het oneigenlijk getal $-\infty$. In symbolen: $\gr 0 = - \infty$.

Bewijs de volgende eigenschappen, waarbij $A(x)$ staat voor een willekeurige veelterm verschillend van de nulveelterm.
\begin{enumerate}%[(a)]
\item
$\gr\bigl( A(x) \cdot 0 \bigr) \, = \, \gr A(x) + \gr 0$
\item
$\gr\bigl( 0 \cdot 0 \bigr) \, = \, \gr 0 + \gr 0$
\item
$\D \gr\bigl( A(x) + 0 \bigr) \, \leq \, \max \bigl\{ \,\gr A(x)\, , \, \gr 0 \, \bigr\}$
\item
$\D \gr\bigl( A(x) + (-A(x)) \bigr) \, \leq \, \max \bigl\{ \,\gr A(x)\, , \, \gr \left(-A(x)\right) \, \bigr\}$
\end{enumerate}
\end{Oefening}
\end{Uitbreiding}


\begin{Antwoord} \label{antw1.1}
    \begin{enumerate}%[(a)]
    \item
    \hyperlink{oef1.1}{eenterm}
    \item
    \hyperlink{oef1.1}{eenterm}
    \item
    \hyperlink{oef1.1}{geen eenterm}
    \item
    \hyperlink{oef1.1}{eenterm}
    \item
    \hyperlink{oef1.1}{geen eenterm}
    \item
    \hyperlink{oef1.1}{eenterm}
    \item\hyperlink{oef1.1}{eenterm}
    \item
    \hyperlink{oef1.1}{eenterm}
    \item
    \hyperlink{oef1.1}{geen eenterm}
    \item
    \hyperlink{oef1.1}{geen eenterm}
    \end{enumerate}
    \end{Antwoord}
    
    \begin{Antwoord} \label{antw1.2}
    \begin{enumerate}
    \item
    \hyperlink{oef1.2}{$x + 2x^2 + 3x^3$}
    \item
    \hyperlink{oef1.2}{$x + \frac{1}{2}\,x^3 + \frac{1}{3}\,x^5 + \frac{1}{4}\,x^7$}
    \item
    \hyperlink{oef1.2}{$x^2 - x^3 + x^4$}
    \end{enumerate}
    \end{Antwoord}
    
    \begin{Antwoord} \label{antw1.3}
    \begin{enumerate}
    \item
    \hyperlink{oef1.3}{veelterm}
    \item
    \hyperlink{oef1.3}{veelterm}
    \item
    \hyperlink{oef1.3}{veelterm}
    \item
    \hyperlink{oef1.3}{veelterm}
    \item
    \hyperlink{oef1.3}{veelterm}
    \item
    \hyperlink{oef1.3}{veelterm}
    \item
    \hyperlink{oef1.3}{geen veelterm}
    \item
    \hyperlink{oef1.3}{veelterm}
    \item
    \hyperlink{oef1.3}{veelterm}
    \item
    \hyperlink{oef1.3}{geen veelterm}
    \item
    \hyperlink{oef1.3}{veelterm}
    \item
    \hyperlink{oef1.3}{veelterm}
    \end{enumerate}
    \setcounter{enumi}{4}
    \end{Antwoord}
    
    \clearpage
    
    \begin{Antwoord} \label{antw1.5}
    \begin{enumerate}
    \item
    \hyperlink{oef1.5}{graad $5$, hoogstegraadsco\"effici\"ent $1$, constante term $-15$}
    \item
    \hyperlink{oef1.5}{graad $6$, hoogstegraadsco\"effici\"ent $2$, constante term $3$}
    \item
    \hyperlink{oef1.5}{graad $7$, hoogstegraadsco\"effici\"ent $\frac{2}{3}$, constante term $-\frac{3}{7}$}
    \item
    \hyperlink{oef1.5}{graad $6$, hoogstegraadsco\"effici\"ent $-15$, constante term $-70$}
    \item
    \hyperlink{oef1.5}{graad $3$, hoogstegraadsco\"effici\"ent $125$, constante term $0$}
    \item
    \hyperlink{oef1.5}{graad $6$, hoogstegraadsco\"effici\"ent $64$, constante term $125$}
    \end{enumerate}
    \setcounter{enumi}{6}
    \end{Antwoord}
    
    \begin{Antwoord} \label{antw1.7}
    \begin{enumerate}
    \item
    \hyperlink{oef1.7}{$-x^3+x^2+3x+1$, graad $3$}
    \item
    \hyperlink{oef1.7}{$x^3-5x^2+4x$, graad $3$}
    \item
    \hyperlink{oef1.7}{$\sqrt{10}\,x+\sqrt{15}$, graad $1$}
    \item
    \hyperlink{oef1.7}{$-9x^4+45x^3+3x^2-21x+30$, graad $4$}
    \item
    \hyperlink{oef1.7}{$-\frac{4}{5}\,x^2-\frac{1}{5}\,x-\frac{4}{5}$, graad $2$}
    \item
    \hyperlink{oef1.7}{$4x-8$, graad $1$}
    \item
    \hyperlink{oef1.7}{$-8x^2+18$, graad $2$}
    \item
    \hyperlink{oef1.7}{$\frac{15}{2}\,x^3 + 3x^2 - \frac{10}{3}\,x-\frac{4}{3}$, graad $3$}
    \end{enumerate}
    \end{Antwoord}
    
    \begin{Antwoord} \label{antw1.8}
    \begin{enumerate}
    \item
    \hyperlink{oef1.8}{$A(2)=0$, nulwaarde}
    \item
    \hyperlink{oef1.8}{$B(-1)=1$, geen nulwaarde}
    \item
    \hyperlink{oef1.8}{$P(5)=0$, nulwaarde}
    \item
    \hyperlink{oef1.8}{$S(0)=-1$, geen nulwaarde}
    \item
    \hyperlink{oef1.8}{$C(\sqrt{2})=0$, nulwaarde}
    \item
    \hyperlink{oef1.8}{$D\left(-\frac{1}{2}\right)=0$, nulwaarde}
    \end{enumerate}
    \end{Antwoord}
    
    \begin{Antwoord} \label{antw1.9}
    \begin{enumerate}
    \item
    \hyperlink{oef1.9}{$a=-3$}
    \item
    \hyperlink{oef1.9}{$a = 1$ en $b = -3$}
    \end{enumerate}
    \end{Antwoord}
    
    \begin{Antwoord} \label{antw1.10}
    \begin{enumerate}
    \item
    \hyperlink{oef1.10}{$a = -\frac{3}{2}$}
    \item
    \hyperlink{oef1.10}{$c = \frac{2}{1-\sqrt{2}}$}
    \end{enumerate}
    \end{Antwoord}
    
    \begin{Antwoord} \label{antw1.11}
    \begin{enumerate}
    \item
    \hyperlink{oef1.11}{$A(x) = -3x^6+24x^4+12x^2$ en $B(x) = -5x^2$}
    \item
    \hyperlink{oef1.11}{$\gr A(x) = 6$ en $\gr B(x) = 2$}
    \item
    \hyperlink{oef1.11}{$A(-2) = 240$ en $B(\sqrt{3}) = -15$}
    \end{enumerate}
    \end{Antwoord}
    
    
    \begin{Antwoord} \label{antw1.12}
    \begin{enumerate}
    \item
    \hyperlink{oef1.12}{$475$}
    \item
    \hyperlink{oef1.12}{$c = -2$}
    \end{enumerate}
    \end{Antwoord}
    
    \begin{Antwoord} \label{antw1.13}
    \hyperlink{oef1.13}{$A(x) = -x^2-x+2$}
    \end{Antwoord}
    
    \begin{Antwoord} \label{antw1.14}
    \hyperlink{oef1.14}{$A(x) = \frac{1}{3}\,x^3-\frac{1}{3}\,x$}
    \end{Antwoord}
    
    \begin{Antwoord} \label{antw1.15}
    \hyperlink{oef1.15}{$2x^2+3x+1$}
    \end{Antwoord}
    
    \begin{Antwoord} \label{antw1.16}
    \hyperlink{oef1.16}{$a = 3$ en $b = \frac{1}{4}$}
    \end{Antwoord}
    
    \begin{Antwoord} \label{antw1.17}
    \hyperlink{oef1.17}{$4096$}
    \end{Antwoord}
    
    \begin{Antwoord} \label{antw1.18}
    \hyperlink{oef1.18}{$32$}
    \setcounter{enumi}{0}
    \end{Antwoord}
    

\end{document}
