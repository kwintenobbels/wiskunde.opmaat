\documentclass[a4paper,12pt]{article}
\usepackage[dutch]{babel}
\usepackage{geometry}
\geometry{margin=1in}
\usepackage{titlesec}

% Define custom commands for formatting
\newcommand{\hoofdstuk}[2]{
    \section*{
        \centering 
        \large \MakeUppercase \textbf{#1}
        \ Large \MakeUppercase \textbf{#2}
        }}

\newcommand{\onderwerp}[1]{\subsection*{\centering \textbf{#1}}}

\newcommand{\subonderwerp}[1]{\subsubsection*{\textbf{#1}}}

\newcommand{\opmerking}{\textbf{Opmerkingen.} }

\begin{document}

% Title
\hoofdstuk{Eerste hoofdstuk}{Het zelfstandig naamwoord}

% Section: Geslacht
\onderwerp{\S 1. Geslacht}

Het geslacht van vele zelfstandige naamwoorden kent men:

\subonderwerp{1. aan hun uitgang.}

Bij iedere verbuiging noemen wij de voornaamste uitgangen op, waaruit men het geslacht der substantieven kan kennen.

\subonderwerp{2. aan hun betekenis.}

\textbf{A. Mannelijk} zijn meestal:

\begin{enumerate}
    \item de namen van \textbf{mannen}:\\
    \textit{Caesar, Caesar ; latrō, rover ; scrība, klerk.}
    
    \opmerking
    \begin{enumerate}
        \item Het geslacht der verzamelnamen hangt altijd af van hun uitgang: \textit{Opera} v. \textit{werkheden} ; \textit{vigiliae} v. \textit{wachten} ; \textit{cohortes} v. \textit{troepen} ; \textit{auxilia} onz. \textit{hulp-troepen}.
        \item \textit{Mancipium, slaaf} (als een zaak beschouwd), is onzijdig.
    \end{enumerate}
    
    \item de namen van \textbf{winden en stromen} (de algemene benamingen \textit{ventus, fluvīus} zijn mannelijk):\\
    \textit{Borēās, noordenwind ; Scaldīs, Schelde.}
    
    Een paar namen van rivieren zijn vrouwelijk, o.a. \textit{Allia} (een riviertje benoorden Rome).\\
    Van enkele is het geslacht onbekend, o.a. van \textit{Mosa, Maas}.
    
    \item de namen der \textbf{maanden}, waarbij men het mannelijke \textit{mensis} kan aanvullen:\\
    \textit{September, september.}
\end{enumerate}

\textbf{B. Vrouwelijk} zijn:

\begin{enumerate}
    \item de namen van \textbf{vrouwen}:\\
    \textit{Dīdō, Dido ; māter, moeder ; uxor, echtgenote ; soror, zuster.}
    
    \item de meeste namen van \textbf{bomen} (de algemene benaming \textit{arbor} is vrouwelijk):\\
    \textit{mālus, appelboom ; pōpulus, populier.}
\end{enumerate}

\end{document}
