\documentclass[a4paper,12pt]{book}


\usepackage[dutch]{babel}
\usepackage{geometry}



\geometry{margin=0.8in}
\usepackage{titlesec}



% WE DEFNIFIEREN DE LAYOUT VOOR DE INHOUDSOPGAVE 

\newcommand{\hoofdstuk}[2]{
    \chapter*{
        \centering 
        \large{\MakeUppercase{\textbf{#1}}}\\[3mm]
        \Large{\MakeUppercase{\textbf{#2}}}
        }}

\newcounter{sectie_nummer}
\setcounter{sectie_nummer}{1}  
\newcommand{\sectie}[1]{
    \section*{
        \centering 
        \S \arabic{sectie_nummer}. \textbf{#1}}
        \refstepcounter{sectie_nummer} % Increment the counter
}

\newcounter{subsectie_nummer}
\setcounter{subsectie_nummer}{1}
\newcommand{\subsectie}[1]{
    \subsection*{
        \centering{ \normalsize{\Roman{subsectie_nummer}. \MakeUppercase{\textbf{#1}}}}}
        \refstepcounter{subsectie_nummer} % Increment the counter
}

\newcounter{subsubsectie_nummer}
\setcounter{subsubsectie_nummer}{1}  
\newcommand{\subsubsectie}[1]{
    \subsubsection*{
        \centering{ \small{\arabic{subsubsectie_nummer}. {\textbf{#1}}}}}
        \refstepcounter{subsubsectie_nummer}
}



\newcommand{\opmerking}{\textbf{Opmerkingen.} }



% WE DEFINIEREN DE GENESTE STRUCTUUR BINNEN PARAGRAFEN 

\newcounter{punt_nummer} % Create a counter for paragraph numbers
\setcounter{punt_nummer}{0} % Start the counter at 51

\newenvironment{punt}{
    \vspace{2mm}
    \setcounter{subpunt_nummer}{0} 
    \refstepcounter{punt_nummer} % Increment the counter
    \textbf{\arabic{punt_nummer}. --- } {}}


\newcounter{subpunt_nummer} % Create a counter for paragraph numbers
\setcounter{subpunt_nummer}{0} % Start the counter at 51

\newenvironment{subpunt}{
  \setcounter{Asubpunt_nummer}{0} 
  \refstepcounter{subpunt_nummer} % Increment the counter
  \arabic{subpunt_nummer}.   % Print the bold number and the paragraph text
}{}


% \newcounter{Asubpunt_nummer} % Create a counter for paragraph numbers
% \setcounter{Asubpunt_nummer}{1} % Start the counter at 51

% \newenvironment{Asubpunt}{
%   \refstepcounter{Asubpunt_nummer} % Increment the counter
%   \Alph{Asubpunt_nummer}.   % Print the bold number and the paragraph text
% }{}


% Tabellen fixen 

\usepackage{booktabs}

\usepackage{mdframed, xcolor}

\usepackage{graphicx}

\usepackage{tikz}



\begin{document}




\hoofdstuk{EERSTE HOOFDSTUK}{HET ZELFSTANDIG NAAMWOORD}


\sectie{Geslacht}

Het geslacht van zelfstandige naamwoorden kent men:

\begin{punt}

    Aan hun \textbf{uitgang.}
    - Bij elke verbuiging worden de voornaamste uitgangen opgesomd.
\end{punt}


\begin{punt}
    
Aan hun \textbf{betekenis.}:

\begin{description} 
    
    \item[A]. \textbf{Mannelijk} zijn meestal:
    \begin{description}
        \item[a] Namen van mannen:
        Caesar (Caesar); latro (rover); scriba (klerk)
        
        Opmerkingen:
        - Het geslacht van verzamelingen hangt af van hun uitgang:
        - operae (vr.) - toeschouwers
        - vigiliae (vr.) - wachten
        - copiae (vr.) - troepen
        - auxilia (onz.) - hulptroepen
        - Mancipium (slaaf, als zaak beschouwd) is onzijdig.
        
        \item[b] Namen van winden en rivieren (de algemene benamingen ventus en fluvius zijn mannelijk):
        - Boreas (noordenwind)
        - Scaldis (Schelde)
        
        Uitzonderingen:
        - Sommige riviernamen zijn vrouwelijk, zoals Allia (een riviertje ten noorden van Rome).
        - Van enkele riviernamen is het geslacht onbekend, zoals Mosa.
        
        \item[c] Namen van maanden (waarbij men het mannelijke "mensis" kan aanvullen):
        - September (september)
        
    \end{description}
    \item[B]. \textbf{Vrouwelijk} zijn:
    a) Namen van vrouwen:
    - Dido (Dido)
    - mater (moeder)
    - uxor (echtgenote)
    - soror (zuster)
    
    b) De meeste namen van bomen (de algemene benaming arbor is vrouwelijk):
    - malus (appelboom)
    - populus (populier)
    
    c) Bijna alle namen van landen, eilanden en steden (zelfs die op -us eindigen):
    - Aegyptus (Egypte)
    - Cyprus (Cyprus)
    - Pontus (landschap Pontus)
    - Hellespontus (de Hellespont) zijn echter mannelijk.
    - Steden in het meervoud zijn ook mannelijk: Veii (in Etrurië), Delphi.
    
    Opmerkingen:
    1. Sommige zelfstandige naamwoorden zijn gemeenschappelijk (mannelijk of vrouwelijk afhankelijk van de persoon):
    - adulescens (jongeman/meisje)
    - bos (os/koe)
    2. Andere hebben verschillende uitgangen voor mannelijk en vrouwelijk:
    - filius (zoon) - filia (dochter)
    - rex (koning) - regina (koningin)
    - avus (grootvader) - avia (grootmoeder)
    
    \item[C] Onzijdig zijn alle onverbuigbare woorden:
    - fas (de goddelijke wet)
    - Alle infinitieven: amare (beminnen), mentiri (liegen), enz.
    
\end{description}
\end{punt}

\sectie{Verbuiging der zelfstandige naamwoorden}

Er zijn vijf verbuigingen, geclassificeerd op basis van de uitgang van de genitief enkelvoud:

| Verbuiging | Genitief eindigt op |
|------------|--------------------|
| 1e         | -ae                |
| 2e         | -i                 |
| 3e         | -is                |
| 4e         | -us                |
| 5e         | -ei                |

------------------------------------------------------

I. EERSTE VERBUIGING (Genitief op -ae)

Deze omvat:
- Vrouwelijke zelfstandige naamwoorden op -a
- Enkele mannelijke op -a

Voorbeeld van verbuiging: **Rosa (vr.) - de roos**

| Naamval | Enkelvoud  | Meervoud  |
|---------|-----------|-----------|
| Nominatief  | rosa  | rosae  |
| Vocatief    | rosa  | rosae  |
| Genitief    | rosae | rosarum |
| Datief      | rosae | rosis  |
| Accusatief  | rosam | rosas  |
| Ablatief    | rosa  | rosis  |

------------------------------------------------------

II. TWEEDE VERBUIGING (Genitief op -i)

Deze omvat:
- Mannelijke zelfstandige naamwoorden op -us
- Enkele vrouwelijke op -us
- Onzijdige op -um
- Mannelijke op -er

Voorbeelden:

1. **Avus (m.) - grootvader** en **Donum (onz.) - geschenk**

| Naamval | Avus (m.)  | Meervoud | Donum (onz.) | Meervoud |
|---------|-----------|----------|--------------|----------|
| Nominatief  | avus  | avi      | donum       | dona     |
| Vocatief    | ave   | avi      | donum       | dona     |
| Genitief    | avi   | avorum   | doni        | donorum  |
| Datief      | avo   | avis     | dono        | donis    |
| Accusatief  | avum  | avos     | donum       | dona     |
| Ablatief    | avo   | avis     | dono        | donis    |

2. **Liber (m.) - boek, Puer (m.) - jongen**

| Naamval | Liber (m.)  | Meervoud | Puer (m.) | Meervoud |
|---------|------------|----------|-----------|----------|
| Nominatief  | liber   | libri    | puer      | pueri    |
| Vocatief    | liber   | libri    | puer      | pueri    |
| Genitief    | libri   | librorum | pueri     | puerorum |
| Datief      | libro   | libris   | puero     | pueris   |
| Accusatief  | librum  | libros   | puerum    | pueros   |
| Ablatief    | libro   | libris   | puero     | pueris   |

------------------------------------------------------

III. DERDE VERBUIGING (Genitief op -is)

Bevat:
1. Consonantstammen (stam eindigt op medeklinker)
2. Vocaalstammen op -i

Voorbeeld:

**Dux (m.) - leider, Corpus (onz.) - lichaam**

| Naamval | Dux (m.) | Meervoud | Corpus (onz.) | Meervoud |
|---------|---------|----------|--------------|----------|
| Nominatief  | dux  | duces    | corpus       | corpora  |
| Vocatief    | dux  | duces    | corpus       | corpora  |
| Genitief    | ducis | ducum   | corporis     | corporum |
| Datief      | duci  | ducibus | corpori      | corporibus |
| Accusatief  | ducem | duces   | corpus       | corpora  |
| Ablatief    | duce  | ducibus | corpore      | corporibus |




\end{document}
