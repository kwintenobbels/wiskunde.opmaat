\documentclass{ximera} 

\input{../preamble}
\input{../preamblekdn}

% voor figuren met PSTricks:
\usepackage{pstricks} 
\usepackage{pstricks-add}
\usepackage{pst-plot}
\usepackage{pst-node}
\usepackage{pst-coil}
\usepackage{auto-pst-pdf}


\addPrintStyle{..}

\begin{document}
	\author{Koen De Naeghel - Wiskunde Op Maat}
	\xmtitle{Oefeningen reeks 3}{}
    \xmsource

	\label{xim:veeltermen_basisbegrippen_oefeningen_reeks3}

%%\section*{Oefeningen reeks 3}



\begin{exercise}
	Gegeven zijn de veeltermen 
	\[
		A(x) = 2x^2 + 5x + 13, \quad B(x) =  x^3-3x-4  \quad \text{ en } \quad C(x) = xA(x) + cB(x)
		\]
		waarbij $c \in \R$.
		
		
		\begin{question} Bepaal \( A\bigl(B(3)\bigr) =  \answer[onlineshowanswerbutton]{475} \) \end{question}
		\begin{question} $\gr C(x) = 2$ voor parameterwaarde \(c = \answer[onlineshowanswerbutton]{-2} \) \end{question}
		
\end{exercise}
	

		
\begin{exercise}
Bepaal een kwadratische veelterm $A(x)$ waarvoor $A(0) = 2$ en waarbij $-2$ en $1$ nulwaarden zijn. \(\answer[onlineshowanswerbutton]{A(x) = -x^2-x+2}\) 
\end{exercise}

\begin{exercise}
Bepaal een kubische veelterm $A(x)$ zonder constante term waarvoor
\[
A(x) = A(x-1) + x(x-1).
\]
A(x) = \(\answer[onlineshowanswerbutton]{\frac{1}{3}\,x^3-\frac{1}{3}\,x}\)
\end{exercise}

\begin{exercise}
Als je weet dat er precies één veelterm is waarvan de derde macht gelijk is aan 
\[
P(x) = 8x^6 + 36x^5 + 66x^4 + 63x^3 + 33x^2 + 9x + 1,
\]
bepaal dan deze veelterm.
\begin{oplossing}\(2x^2+3x+1\) \end{oplossing}
\end{exercise}

\begin{exercise}
Gegeven is de veelterm
\[
A(x) = x^4 - 4x^3 + ax^2 + 2x + b
\]
waarbij $a,b \in \R$. Bepaal de waarde(n) van de parameters $a$ en $b$ waarvoor $A(x)$ het kwadraat is van een veelterm. 
\begin{oplossing} \(a = 3\) en \(b = \frac{1}{4}\) \end{oplossing}
\end{exercise}

\begin{exercise}
Beschouw de veelterm $P(x) = (3x - 1)^{12}$. Bereken algebraïsch de som van de coëfficiënten. 
\begin{oplossing} De som van de coëfficiënten is $4096$. \end{oplossing}
\end{exercise}

\begin{exercise}
Als $a,b,c,d$ reële getallen zijn waarvoor geldt dat
\[
ax^3 + bx^2 + cx + d = (x^2-6x-8)^{10}\cdot(4x+2)^5 - (x+1)^8\cdot(x^2+5x+4)^{27}
\]
bepaal dan algebraïsch de waarde van $a-b+c-d$.
\begin{oplossing} De waarde van \(a-b+c-d = 32\).\end{oplossing}
\end{exercise}

\begin{exercise}
Van een rechthoekig stuk karton met afmetingen $20\:\text{cm}$ op $10\:\text{cm}$ knippen we in elke hoek een vierkant met zijde $x$ weg. Nadien plooien we het karton langs de stippellijnen, om zo een doos zonder deksel rechts te verkrijgen. 

 
	\begin{question} Schrijf een veelterm $V(x)$ op dat het volume van de doos weergeeft.                                                                                 \( \answer[onlineshowanswerbutton]{  } \) \end{question}
	\begin{question} Geef de graad en de constante term van deze veelterm, en verklaar jouw antwoord zowel algebraïsch als met behulp van de context van deze oefening.   \( \answer[onlineshowanswerbutton]{  } \) \end{question}
	\begin{question} Geef alle nulwaarden van deze veelterm, en verklaar jouw antwoord zowel algebraïsch als met behulp van de context van deze oefening.                 \( \answer[onlineshowanswerbutton]{  } \) \end{question}


\medskip

%%%%%%%%%%%%%%%%%%%%%%%%%%%%%%%%%%%%%%%%%%%%%%%%%%%%%%%%%%%%%%%%%%%%%%%%%
% KOEN ZIJN PSTRIKS; TIKZPICTURE VOOR GEMAAKT 
% \begin{center}
% \psset{xunit=1cm,yunit=1cm}
% \begin{pspicture}(0,-1)(6,4)% co linksonder, co rechtsboven
% \psline[](0,0)(6,0)(6,3)(0,3)(0,0)

% \psline[linestyle=dashed](0,0.5)(6,0.5)
% \psline[linestyle=dashed](0,2.5)(6,2.5)
% \psline[linestyle=dashed](0.5,0)(0.5,3)
% \psline[linestyle=dashed](5.5,0)(5.5,3)

% \psline[linecolor=blue]{<->}(0,3.5)(6,3.5)
% \uput[u](3,3.5){\color{blue}\SI{20}{\cm}}

% \psline[linecolor=blue]{<->}(-0.5,0)(-0.5,3)
% \uput[l](-0.5,1.5){\color{blue}\SI{10}{\cm}}

% \psline[linecolor=red]{<->}(0,-0.5)(0.5,-0.5)
% \uput[d](0.25,-0.5){\color{red}$x$}

% \psline[linecolor=red]{<->}(5.5,-0.5)(6,-0.5)
% \uput[d](5.75,-0.5){\color{red}$x$}

% \psline[linecolor=red]{<->}(6.5,0)(6.5,0.5)
% \uput[r](6.5,0.25){\color{red}$x$}

% \psline[linecolor=red]{<->}(6.5,2.5)(6.5,3)
% \uput[r](6.5,2.75){\color{red}$x$}
% \end{pspicture}
% \end{center}

\begin{image}
\begin{tikzpicture}[x=1cm,y=1cm]

	% Rectangle
	\draw (0,0) rectangle (6,3);
	
	% Dashed lines
	\draw[dashed] (0,0.5) -- (6,0.5);
	\draw[dashed] (0,2.5) -- (6,2.5);
	\draw[dashed] (0.5,0) -- (0.5,3);
	\draw[dashed] (5.5,0) -- (5.5,3);
	
	% Blue double arrows for dimensions
	\draw[<->,blue] (0,3.5) -- (6,3.5); %node[midway, above]{\SI{20}{\cm}};
	\draw[<->,blue] (-0.5,0) -- (-0.5,3); % node[midway, left]{\SI{10}{\cm}};
	
	% Red arrows for x distances
	\draw[<->,red] (0,-0.5) -- (0.5,-0.5) node[midway, below] {$x$};
	\draw[<->,red] (5.5,-0.5) -- (6,-0.5) node[midway, below] {$x$};
	\draw[<->,red] (6.5,0) -- (6.5,0.5) node[midway, right] {$x$};
	\draw[<->,red] (6.5,2.5) -- (6.5,3) node[midway, right] {$x$};
	
\end{tikzpicture}
\end{image}

%%%%%%%%%%%%%%%%%%%%%%%%%%%%%%%%%%%%%%%%%%%%%%%%%%%%%%%%%%%%%%%%%%%%%%%%%
\end{exercise}

%%% \clearpage

\begin{Uitbreiding}
\begin{exercise}
\label{oefgraadnulveelterm}
{\bf (min oneindig en de graad van de nulveelterm)} 
Om aan de nulveelterm ook een graad te kunnen toekennen, breiden we de verzameling van de reële getallen uit met een element, voorgesteld door het symbool $- \infty$, lees als: min oneindig. Het element $-\infty$ is dus geen (reëel) getal. Men spreekt ook wel van het \textit{oneigenlijk getal} $-\infty$. Ook de orde en de optelling in $\R$ worden uitgebreid door de volgende definities (waarbij $a \in \R$):
\[
\begin{aligned}
\\[-0.5cm]
-\infty & < a \\
(-\infty) + a & = -\infty \\
a + (-\infty) & = -\infty \\
(-\infty) + (-\infty) & = (-\infty).
\end{aligned}
\]
De \underline{graad van de nulveelterm}\index{graad}\index{veelterm!graad} is nu per definitie gelijk aan het oneigenlijk getal $-\infty$. In symbolen: $\gr 0 = - \infty$.

Bewijs de volgende eigenschappen, waarbij $A(x)$ staat voor een willekeurige veelterm verschillend van de nulveelterm.

\begin{enumerate}

	\item
	$\gr\bigl( A(x) \cdot 0 \bigr) \, = \, \gr A(x) + \gr 0$
	\item
	$\gr\bigl( 0 \cdot 0 \bigr) \, = \, \gr 0 + \gr 0$
	\item
	$\D \gr\bigl( A(x) + 0 \bigr) \, \leq \, \max \bigl\{ \,\gr A(x)\, , \, \gr 0 \, \bigr\}$
	\item
	$\D \gr\bigl( A(x) + (-A(x)) \bigr) \, \leq \, \max \bigl\{ \,\gr A(x)\, , \, \gr \left(-A(x)\right) \, \bigr\}$
	
\end{enumerate}
\end{exercise}
\end{Uitbreiding}

\end{document}