\documentclass{ximera} 

\input{../preamble}
\input{../preamblekdn}
\addPrintStyle{..}

\begin{document}
	\author{Koen De Naeghel - Wiskunde Op Maat}
	\xmtitle{Oefeningen reeks 2}{}
    \xmsource
	\label{xim:veeltermen_deling_door_xa_oefeningen_reeks2}

%%\section*{Oefeningen reeks 2}

\begin{exercise}
Gegeven zijn de veeltermen
\[
A(x) = -x^4 - kx^2+\frac{3}{2} \quad \text{ en } \quad B(x) = x+2
\]
waarbij \(k \in \R\). Bepaal de waarde(n) van de parameter \(k\) zodat de rest bij deling van \(A(x)\) door \(B(x)\) gelijk is aan \(\frac{4}{5}\). 
\begin{uitkomst} \( k = - \frac{153}{40} \) \end{uitkomst} 
\end{exercise}

\begin{exercise}
{\bf (toelatingsexamen arts)}\index{toelatingsexamen arts} 
De deling van de veelterm \(P(x) = x^3 + mx^2 + mx + 4\) door \(x-2\) en \(x+2\) levert dezelfde rest op. Hoeveel is die rest?
\begin{multipleChoice}
\choice{\(-16\)}
\choice[correct]{\(-12\)}
\choice{\(-8\) }
\choice{\(-4\) }
\end{multipleChoice}
\end{exercise}

\begin{exercise}
Bepaal telkens de exacte waarde(n) van de parameters zodat \(A(x)\) deelbaar is door \(B(x)\). 
	\begin{question} \( A(x) = 5x^2+2x-7                    \quad \text{ en } \quad B(x) = x-2a       \) \begin{uitkomst} \( a \in \left\{\frac{1}{2}, -\frac{7}{10}\right\} \) \end{uitkomst} \end{question} 
	\begin{question} \( A(x) = (2-a)x^2 + 5ax + a^2         \quad \text{ en } \quad B(x) = x-3        \) \begin{uitkomst} \( \text{een enkel reëel getal \(a\) voldoet}        \) \end{uitkomst} \end{question} 
	\begin{question} \( A(x) = -2x^2 + a\sqrt{2}\,x + a-5   \quad \text{ en } \quad B(x) = x-\sqrt{2} \) \begin{uitkomst} \( a = \frac{5+2\sqrt{2}}{3}                       \) \end{uitkomst} \end{question} 
	\begin{question} \( A(x) = 2x^3 + ax^2 + (1-6a^2)x + 2a \quad \text{ en } \quad B(x) = x+2a       \) \begin{uitkomst} \( a \in \R                                        \) \end{uitkomst} \end{question} 
	\begin{question} \( A(x) = ax^3 + 19x^2 + bx + 8        \quad \text{ en } \quad B(x) = (x-2)(x+4) \) \begin{uitkomst} \( a = 10\( \text{ en } \)b = -82                             \) \end{uitkomst} \end{question} 
	\begin{question} \( A(x) = x^4 + ax^3 - 9x^2 + 18x + b  \quad \text{ en } \quad B(x) = x(x-2)     \) \begin{uitkomst} \( a = -2\( \text{ en } \)b = 0                               \) \end{uitkomst} \end{question} 
	\begin{question} \( A(x) = -2x^3 + bx - 2ax^2 + 3a      \quad \text{ en } \quad B(x) = x+a        \) \begin{uitkomst} \( a = 0\( \text{ of } \)b = 3                                \) \end{uitkomst} \end{question} 

\end{exercise}


%%% \clearpage

\begin{exercise}
{\bf (toelatingsexamen arts)}\index{toelatingsexamen arts}
De veelterm \(P(x) = 8x^3 + 8\) is deelbaar door \(x+a\), met \(a \in \R\). Hoeveel is de rest van de deling van \(P(x)\) door \(x+2a\)?
\begin{multipleChoice}
\choice{\(-60\)}  
\choice[correct]{\(-56\)}  
\choice{\(-52\)}  
\choice{\(-50\)}  
\end{multipleChoice}
\end{exercise}

\begin{exercise}
{\bf (toelatingsexamen arts)}\index{toelatingsexamen arts}
Als de veelterm \(P(x) = x^2 + ax + a\) deelbaar is door \(x+b\), met \(a\) en \(b\) reële getallen, dan geldt
\begin{multipleChoice}
\choice{\(b \neq 0\) en \(\D a = -\frac{b}{b-1}\)  }
\choice{\(b \neq 1\) en \(\D a = -\frac{b^2}{b-1}\)}
\choice[correct]{\(b \neq 1\) en \(\D a = \frac{b^2}{b-1}\) }
\choice{\(b \neq 1\) en \(\D a = \frac{b}{b-1}\)   }
\end{multipleChoice}
\end{exercise}

\begin{exercise}
Gegeven is de veelterm
\[
A(x) = x^3 - ax^2 + bx - 12
\]
waarbij \(a,b \in \R\). Bepaal de waarde(n) van de parameters \(a\) en \(b\) waarvoor \(A(x)\) deelbaar is door \((x-1)(x-3)\). 

\begin{uitkomst} \(a = 8\) \text{en} \(b = 19\) \end{uitkomst}
\end{exercise}

\begin{exercise}
{\bf (toelatingsexamen arts)}\index{toelatingsexamen arts}
We beschouwen de veelterm \(A(x) = 2x^3 + px^2 + qx + r\). Deze veelterm is deelbaar door \(x^2 - 1\) en de rest bij deling door \(x-3\) is \(8\). Geef de waarde van de uitdrukking \((p-r)\cdot q\).
\begin{multipleChoice}
\choice{\(-20\)}
\choice{\(-10\)}
\choice{\(10\) }
\choice[correct]{\(20\)}
\end{multipleChoice}
\end{exercise}

\begin{exercise}
Gegeven is de veelterm
\[
A(x) = (b-c)x^2 + b^2(c-x) + c^2(x-b)
\]
waarbij \(b,c \in \R\) met \(b \neq c\). Toon aan dat \(A(x)\) deelbaar is door \((x-b)(x-c)\) en schrijf \(A(x)\) als een product. 
\begin{uitkomst} \( A(x) = (b-c)(x-b)(x-c) \) \end{uitkomst}
\end{exercise}

\begin{exercise}
{\bf (toelatingsexamen arts)}\index{toelatingsexamen arts}
We beschouwen de veelterm \(F(x) = x^3 + px^2 - 8x + q\). Deze veelterm is deelbaar door \(x-1\) en de rest bij deling door \(x^2-9\) is \(x-9\). Geef de waarde van \(q\).
\begin{multipleChoice}
\choice{\(-9\)}
\choice{\(-2\)}
\choice[correct]{\(9\) }
\choice{\(16\)}
\end{multipleChoice}
\end{exercise}

\begin{exercise}
Bepaal telkens het quotiënt en de rest bij deling van \(A(x)\) door \(B(x)\).

	\begin{question} \(A(x) = 4x^3-6x+2\) \quad en \quad \(B(x) = 2x-6\)                                             \begin{uitkomst} Het quotiënt is \(2x^2+6x+15\)                 en rest \(92\)              \end{uitkomst} \end{question}
	\begin{question} \(A(x) = 18x^3-10\) \quad en \quad \(\D B(x) = x + \sqrt{6}\)                                   \begin{uitkomst} Het quotiënt is \(18x^2 - 18\sqrt{6}\,x+108\)  en rest \(-10-108\sqrt{6}\) \end{uitkomst} \end{question}
	\begin{question} \(A(x) = 2x^4 + 17x^3 - 68x\) \quad en \quad \(\D B(x) = x+\frac{1}{2}\)                        \begin{uitkomst} Het quotiënt is \(2x^3+16x^2-8x-64\)           en rest \(32\)              \end{uitkomst} \end{question}
	\begin{question} \(A(x) = \sqrt{2}\,x^2 + 3\sqrt{10}\,x - 20\sqrt{2}\) \quad en \quad \(\D B(x) = x - \sqrt{5}\) \begin{uitkomst} Het quotiënt is \(\sqrt{2}\,x+4\sqrt{10}\)     en rest \(0\)               \end{uitkomst} \end{question}
\end{exercise}

\begin{exercise}
Jeroen deelt de veelterm \(6x^4-7x^3+22x^2-24x-13\) door \(2x-1\) en vindt als quotiënt \(6x^3-4x^2+20x-14\) en als rest \(-20\). 
\begin{enumerate}
\item[(a)]
Hoe kan Jeroen snel inzien dat hij een fout gemaakt heeft zonder de deling opnieuw uit te voeren? 
\item[(b)]
Bepaal het juiste quotiënt en de juiste rest.
\item[(c)]
Hoe kun je zeker weten dat jouw quotiënt en rest correct is? Voer dit uit.
\end{enumerate}
\begin{uitkomst} Het quotiënt is \(3x^3-2x^2+10x-7\) en de rest \(-20\) \end{uitkomst}
\end{exercise}






\end{document}