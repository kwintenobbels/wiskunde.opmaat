\documentclass{ximera} 
\input{../preamble}
\input{../preamblekdn}

\addPrintStyle{..}

\begin{document}
	\author{Koen De Naeghel - Wiskunde Op Maat}
	\xmtitle{Oefeningen reeks 3}{}
    \xmsource
	\label{xim:veeltermen_deling_door_veelterm_oefeningen_reeks3}


%%\section*{Oefeningen reeks 3}

\begin{exercise}
Beschouw \(a,b,c,d \in \R\) willekeurig met \(a \neq 0\). Bewijs:
\[
\exists k \in \R: (x^2 + k) \mid (ax^3 + bx^2 + cx + d) \quad \Leftrightarrow \quad ad = bc
\]
\end{exercise}

% \begin{Uitbreiding}
\begin{exercise} 
Zij \(A(x)\) en \(B(x)\) twee veeltermen waarbij \(A(x) \neq 0\) en \(B(x) \neq 0\). Vul aan en bewijs de volgende eigenschap:
\[
B(x) \mid A(x) \quad \Rightarrow \quad \gr\biggl(\frac{A(x)}{B(x)}\biggr) 
\, = \, \ldots
\]
\end{exercise}

\begin{exercise}
Zij \(r \in \R_0\) en \(A(x), B(x), C(x), D(x), E(x), K(x)\) en \(L(x)\) veeltermen waarbij \(D(x) \neq 0\) en \(E(x) \neq 0\). Bewijs de volgende basiseigenschappen van deelbaarheid. 


	\begin{question} Als \(\D A(x)  \neq 0\) dan is \(A(x) \mid 0\) en \(\D \bigl(rA(x)\bigr) \mid A(x)\).                                   \end{question}
	\begin{question} Als \(D(x) \mid A(x)\) dan is \(\D \bigl(rD(x)\bigr) \mid A(x)\).                                                     \end{question}
	\begin{question} Als \(D(x) \mid E(x)\) en \(E(x) \mid A(x)\) dan is \(D(x) \mid A(x)\).                                                 \end{question}
	\begin{question} Als \(D(x) \mid A(x)\) dan is \(D(x) \mid \bigl(K(x)A(x)\bigr)\) en \(\bigl(E(x)D(x)\bigr) \mid \bigl(E(x)A(x)\bigr)\). \end{question}
	\begin{question} Als \(D(x) \mid A(x)\) en \(D(x) \mid B(x)\) dan is \(D(x) \mid \bigl(K(x)A(x)+L(x)B(x)\bigr)\).                        \end{question}
	\begin{question} Als \(D(x) \mid A(x)\) dan is \(\gr D(x) \leq \gr A(x)\) of \(A(x) = 0\).                                               \end{question}
	
\end{exercise}
% \end{Uitbreiding}

\begin{exercise}[\bf \ref{antw2.17}.]\setcounter{enumi}{17}  
Stel dat het deling van een veelterm \(A(x)\) door een veelterm \(B(x) \neq 0\) als quotiënt \(Q(x)\) en als rest \(R(x)\) heeft. Als \(\gr A(x) = 12\) en \(\gr R(x) = 2\), wat zijn dan de mogelijke waarden voor de graad van de veelterm \(Q(x)\)?
\( \answer[onlineshowanswerbutton]{ \gr Q(x) \in \left\{0,1,2,3,4,5,6,7,8,9\right\} } \)
\end{exercise}

\begin{exercise}[\bf \ref{antw2.18}.]\setcounter{enumi}{18} 
Gegeven is de veelterm
\[
A(x) = x^3+3x^2+ax+13
\]
waarbij \(a \in \R\). Bepaal de waarde(n) van de parameter \(a\) waarvoor de deling van \(A(x)\) door \(x^2+3x-2\) als als rest een getal heeft. 
\( \answer[onlineshowanswerbutton]{ a = -2 } \)

\end{exercise}

\begin{exercise}[\bf \ref{antw2.19}.]\setcounter{enumi}{19}
Bepaal de rest bij deling van \(A(x)\) door \(B(x)\) waarbij
\[
A(x) = x^{81} + x^{49} + x^{25} + x^{9} + x \quad \text{ en } \quad B(x) = x^3-x.
\]
\( \answer[onlineshowanswerbutton]{ x} \) 
\end{exercise}

\begin{exercise}[\bf \ref{antw2.20}.]\setcounter{enumi}{20}
Toon voor elk van onderstaande gevallen het bestaan van de veeltermen \(Q(x)\) en \(R(x)\) in de stelling van de euclidische deling aan. Doe dat telkens door expliciet de veeltermen \(Q(x)\) en \(R(x)\) te geven. 

	\begin{question} \(A(x) = 0\) en \(B(x) \neq 0\) een willekeurige veelterm                                                              \( = \answer[onlineshowanswerbutton] {Q(x) = 0\( en \)R(x) = 0                                      } \) \end{question} 
	\begin{question} \(A(x)\) een willekeurige veelterm en \(B(x) = 1\)                                                                     \( = \answer[onlineshowanswerbutton] {Q(x) = A(x)\( en \)R(x) = 0                                   } \) \end{question}
	\begin{question} \(A(x)\) een willekeurige veelterm en \(B(x) = -7\)                                                                    \( = \answer[onlineshowanswerbutton] {Q(x) = -\frac{1}{7}\( en \)R(x) = 0                           } \) \end{question}
	\begin{question} \(A(x)\) een willekeurig reëel getal en \(B(x) \neq 0\) een willekeurige veelterm met constante term gelijk aan nul. \( = \answer[onlineshowanswerbutton] {Q(x) = 0\( en \)R(x) = A(x)                                   } \) \end{question}
	\begin{question} \(A(x) \neq 0\) een willekeurige veelterm en \(B(x) = A(x)\)                                                           \( = \answer[onlineshowanswerbutton] {Q(x) = 1\( en \)R(x) = 0                                      } \) \end{question}
	\begin{question} \(A(x) \neq 0\) een willekeurige veelterm en \(B(x) = \sqrt{2}\cdot A(x)+\pi\)                                         \( = \answer[onlineshowanswerbutton] {Q(x) = \frac{1}{\sqrt{2}}\( en \)R(x) =-\frac{\pi}{\sqrt{2}}  } \) \end{question}

\end{exercise}


\end{document}