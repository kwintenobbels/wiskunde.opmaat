%%%%%%%%%%%%%%%%%%%%%%%%%%%%%%%%%%%%%%%%%%%%%%%%%%%%%%%%%%%%%%%%%%%%%%%%%%%%%%%%%%%%%%%%%%

% DIT MATERIAAL VERTROK VAN DE OPEN-SOURCE CURSUS VEELTERMEN VAN KOEN DE NAEGHEL         
% GEKOPIEËRD OP 24 MAART 2025                                                            
% ORIGINEEL BESCHIKBAAR VIA https://www.koendenaeghel.be/opensource.htm         

%%%%%%%%%%%%%%%%%%%%%%%%%%%%%%%%%%%%%%%%%%%%%%%%%%%%%%%%%%%%%%%%%%%%%%%%%%%%%%%%%%%%%%%%%%



\documentclass{ximera}
\input{../preamble}
\input{../preamblekdn}

\addPrintStyle{..}
\begin{document}
	\author{Koen de Naeghel - Wiskunde Op Maat}
	\xmtitle{Kenmerken van deelbaarheid}{}
    \xmsource


Om na te gaan of een geheel getal deelbaar is door $2$, $3$, $5$, $7$, $9$, $10$ of $11$ kun je een kenmerk van deelbaarheid gebruiken. Voor grotere getallen gaat dat heel wat sneller dan het uitvoeren van de staartdeling. Zo kun je meteen inzien dat $2025$ deelbaar is door $3$ en door $5$, en dus ook door $15$.  

Ook bij veeltermen kun je spreken over kenmerken van deelbaarheid. Met behulp van de reststelling worden er in dit onderdeel voor twee soorten veeltermen een kenmerk van deelbaarheid gegeven. 

De staartdeling gaat in het algemeen na of een veelterm $A(x)$ deelbaar is door een veelterm $B(x) \neq 0$, met volgend resultaat:

\[
B(x) \mid A(x) \quad \Leftrightarrow \quad \text{ de rest bij deling van $A(x)$ door $B(x)$ is gelijk aan $0$.}
\] 

In het geval dat $B(x) = x-a$ met $a \in \R$ geeft de reststelling een veel eenvoudiger criterium.

\begin{theorem}(kenmerk van deelbaarheid door \(x-a\))
Zij $A(x)$ een veelterm en $a \in \R$. Dan geldt:
\[
(x-a) \mid A(x) \quad \Leftrightarrow \quad A(a) = 0.
\]
\end{theorem} 


\begin{expandable}{proof}

Wegens de reststelling geldt:
\begin{align}
(x-a) \mid A(x) \quad 
& \Leftrightarrow \quad \underbrace{\text{ de rest bij deling van $A(x)$ door $x-a$ }}_{A(a)} \text{ is gelijk aan $0$} \nonumber \\
& \Leftrightarrow \quad A(a) = 0. % \tag*{\qedhere} bugt op html 
\end{align}

\end{expandable}

Bij veeltermen met parameters hangt de deelbaarheid af van de waarden van de parameters. Zoals volgend voorbeeld aantoond kan het bepalen van deze waarden aanleiding geven tot een vergelijking. Verwacht wordt dat je vlot eerstegraads-, tweedegraads- en bikwadratische vergelijkingen kan oplossen waarbij je jouw redenering correct kan opschrijven. 

\begin{example} 
We bepalen de waarde(n) van de reële parameter $p$ waarvoor de veelterm $A(x) = 5x^2-4x-3$ deelbaar is door $x-2p$. Het kenmerk van deelbaarheid door $x-a$ geeft:

\begin{align*}
(x-2p) \mid A(x) \quad 
& \Leftrightarrow \quad A(2p) = 0 \\
& \Leftrightarrow \quad 5 \cdot (2p)^2 - 4 \cdot 2p - 3 = 0 \\
& \Leftrightarrow \quad 20p^2 - 8p - 3 = 0 \\
% & \mph{\Leftrightarrow} \qquad 
\end{align*}

De gezochte waarden voor de parameter \(p \) is de oplossing van deze kwadratische vergelijking. \href{https://wiskunde.opmaat.org/rekenoffensief/vaardigheden/tweede_graad/kwadratische_vergelijkingen}{Zorg ervoor dat je het oplossen van tweedegraadsvergelijkingen goed beheerst}. 


Een rechtstreekse berekening levert de discriminant: \(D = b^2-4ac = (-8)^2-4 \cdot 20 \cdot (-3) = 304 > 0 \). Om de wortel van \(304\) te kunnen vereenvoudigen bepaal je de priemfactoristatie: 

\[
\begin{array}{lc|cr}
\text{schema: } & 304 & 2 \\
& 152 & 2 \\
& 76 & 2 \\
& 38 & 2 \\
& 19 & 19 \\
& 1 & & \text{ zodat } \,\, 304 = 2^4 \cdot 19 = 4^2 \cdot 19
\end{array}
\]

\begin{align*} \\
& \Leftrightarrow \quad p = \frac{-b \pm \sqrt{D}}{2a} \\ 
& \Leftrightarrow \quad p = \frac{-(-8) \pm \sqrt{304}}{2 \cdot 20} \\ 
% & \mph{\Leftrightarrow} \qquad  % Dit geeft in HTML \mathcolor bug 
& \Leftrightarrow \quad p = \frac{8 \pm 4\sqrt{19}}{40} \\
& \Leftrightarrow \quad p = \frac{2 + \sqrt{19}}{10} \,\, \text{ of } \,\, p = \frac{2 - \sqrt{19}}{10}.
\end{align*}

\end{example} 


Het kenmerk van deelbaarheid door $x-a$ kan gemakkelijk veralgemeend worden. De uitspraak die we hieronder moeten aantonen, is een equivalentie. Een bewijs van zo'n equivalentie $P \Leftrightarrow Q$ bestaat vaak uit twee deelbewijzen: een bewijs van de implicatie $P \Rightarrow Q$ en een bewijs van de implicatie $Q \Rightarrow P$. In beide gevallen zullen we de implicatie aantonen met een rechtstreeks bewijs.

\begin{theorem}(kenmerk van deelbaarheid door \((x-a)(x+a)\))

Zij $A(x)$ een veelterm en $a,b \in \R$ met $a \neq b$. Dan geldt:
\[
(x-a)(x-b) \mid A(x) \quad \Leftrightarrow \quad A(a) = 0 \,\, \text{ en } \,\, A(b) = 0.
\]

\end{theorem} 

\begin{expandable}{proof}

We bewijzen de eigenschap in twee delen.

Veronderstel eerst dat $(x-a)(x-b) \mid A(x)$. We moeten aantonen dat $A(a) = 0$ en $A(b) = 0$.

Omdat $(x-a)(x-b) \mid A(x)$ bestaat er een veelterm $Q(x)$ zodat
\[
A(x) = (x-a)(x-b)Q(x).
\]
Vervangen we $x$ door $a$ dan vinden we 
\[
A(a) = \underbrace{(a-a)}_{0}(a-b)Q(a) = 0
\]
en analoog is
\[
A(b) = (b-a)\underbrace{(b-b)}_{0}Q(a) = 0.
\]
Omgekeerd, veronderstel dat $A(a) = 0$ en $A(b) = 0$. Te bewijzen is dat $(x-a)(x-b) \mid A(x)$.

Uit het kenmerk van deelbaarheid door $x-a$ volgt dat $(x-a) \mid A(x)$. Er bestaat dus een veelterm $Q'(x)$ zodat
\[
A(x) = (x-a)Q'(x).
\]
Als we hierin $x$ vervangen door $b$, dan verkrijgen we 
\[
\underbrace{A(b)}_{0} = \underbrace{(b - a)}_{\neq 0} Q'(b)
\]
zodat $Q'(b) = 0$. Dus $(x-b) \mid Q'(x)$ zodat $Q'(x) = (x-b)Q''(x)$ voor een zekere veelterm $Q''(x)$. Samengevat is dan
\[
A(x) = (x-a)Q'(x) = (x-a)(x-b)Q''(x)
\]
waaruit we mogen besluiten dat $(x-a)(x-b) \mid A(x)$.

\end{expandable}


\begin{example} 
We gaan na of de veelterm $A(x) = x^{2025} - x^{2009}$ deelbaar is door $x^2 - 1$. Omdat $x^2-1 = (x-1)(x+1)$ kunnen we het kenmerk van deelbaarheid door $(x-a)(x-b)$ toepassen:
\begin{align*}
(x^2-1) \mid (x^{2025} - x^{2009}) \quad 
& \Leftrightarrow \quad (x-1)(x+1) \mid A(x) \\
& \Leftrightarrow \quad A(1) = 0 \,\, \text{ en } \,\, A(-1) = 0 \\
& \Leftrightarrow \quad 1^{2025} - 1^{2009} = 0 \,\, \text{ en } \,\, (-1)^{2025} - (-1)^{2009} = 0 \\
& \Leftrightarrow \quad 1 - 1 = 0 \,\, \text{ en } \,\, (-1) - (-1) = 0.
\end{align*}
Die laatste uitspraak $1 - 1 = 0$ en $(-1) - (-1) = 0$ is waar, en omdat ze gelijkwaardig is met de eerste uitspraak $(x^2-1) \mid (x^{2025} - x^{2009})$ besluiten we dat $A(x)$ deelbaar is door $x^2-1$. 
\end{example} 

	

\end{document}
