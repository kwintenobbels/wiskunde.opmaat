%%%%%%%%%%%%%%%%%%%%%%%%%%%%%%%%%%%%%%%%%%%%%%%%%%%%%%%%%%%%%%%%%%%%%%%%%%%%%%%%%%%%%%%%%%

% DIT MATERIAAL VERTROK VAN DE OPEN-SOURCE CURSUS VEELTERMEN VAN KOEN DE NAEGHEL         
% GEKOPIEËRD OP 24 MAART 2025                                                            
% ORIGINEEL BESCHIKBAAR VIA https://www.koendenaeghel.be/opensource.htm         

%%%%%%%%%%%%%%%%%%%%%%%%%%%%%%%%%%%%%%%%%%%%%%%%%%%%%%%%%%%%%%%%%%%%%%%%%%%%%%%%%%%%%%%%%%



\documentclass{ximera}
\input{../preamble}
\input{../preamblekdn}

\addPrintStyle{..}
\begin{document}
\author{Koen de Naeghel - Wiskunde Op Maat}
\xmtitle{Veeltermen}{}
    \xmsource



De som van de eentermen  $ax^n$ en $bx^m$ levert een \textit{tweeterm} $ax^n + bx^m$ . Door nogmaals te sommeren verkrijg je een \textit{drieterm}, een \textit{vierterm} enzovoort. In het algemene geval spreekt men van een \textit{veelterm}.

\begin{definition} 
Een \textbf{(reële) veelterm \(A(x)\) in (de variabele) $x$} is een eindige som van eentermen in $x$.

In symbolen: $A(x) = a_0 + a_1x + a_2x^2 + \dots + a_n x^n$ waarbij $n \in \N$ en $a_0,a_1,a_2,\ldots,a_n \in \R$.


Met de afspraak $x^0 = 1$ kan deze definitie van een veelterm herschreven worden met het sommatieteken:
\[
A(x) = a_0 + a_1x + a_2x^2 + \dots + a_n x^n = \sum_{i=0}^n a_i x^i.
\]

Hierbij zijn de getallen $a_0, a_1, a_2, \ldots, a_n$ de \textbf{coëfficiënten} van de veelterm. De term $a_0$ wordt de \textbf{constante term} van de veelterm genoemd.


Als $a_0 + a_1x + a_2x^2 + \dots + a_n x^n$ een veelterm is waarbij $a_n \neq 0$, dan is de \textbf{graad} van de veelterm gelijk aan $n$. Voor een veelterm $A(x)$ noteert men de graad \(n\) als $\gr A(x)$. In dat geval is $a_nx^n$ de \textbf{hoogstegraadsterm} en $a_n$ de \textbf{hoogstegraadscoëfficiënt} van de veelterm. De graad van de veelterm $0 + 0\cdot x + 0 \cdot x^2 + \dots + 0\cdot x^n = 0$ wordt in het secundair onderwijs niet gedefiniëerd.

Veeltermen van graad nul, één, twee of drie wordt respectievelijk \textit{constante}, \textit{lineaire}, \linebreak \textit{kwadratische} en \textit{kubische veeltermen} genoemd. Ook de nulveelterm $0 + 0\cdot x + 0 \cdot x^2 + \dots + 0\cdot x^n = 0$ wordt een constante veelterm genoemd.


Twee veeltermen in $x$ zijn \textbf{gelijk} als ze ofwel dezelfde graad én dezelfde coëfficiënten hebben, ofwel beide gelijk zijn aan nul. 


\end{definition}




\begin{notation}
    
    In het vervolg van deze cursus geldt -tenzij anders gespecifeerd- dat voor \textit{ een veelterm} $a_0 + a_1x + a_2x^2 + \dots + a_n x^n$ telkens dat $n \in \N$ en $a_0, \ldots, a_n\in \R$ en $b_0, \ldots, b_m\in \R$.
    
\end{notation}



\begin{example} 
De volgende uitdrukkingen zijn veeltermen in $x$:
\[
3x^2, 
\qquad \sqrt{5} - 8x, 
\qquad -1 + 3x -2x^2 
\qquad \text{ en } \qquad -\frac{4}{3}x^5 - 2x^3 + \pi^2\,x + 5. 
\]      
\end{example} 
    

\begin{example} 
In bovenstaande definitie de waarden $n = 4$, $a_0 = -5$, $a_1 = 7$, $a_2 = -3$, $a_3 = 0$ en $a_4 = 0$ invullen levert de veelterm: 
\begin{align*}
A(x) 
& = a_0 + a_1x + a_2x^2 + \dots + a_n x^n \\
& = a_0 + a_1x + a_2x^2 + a_3 x^3 + a_4 x^4 \\ 
& = -5 + 7x - 3x^2 + 0x^3 + 0x^4 \\
& = -5 + 7x - 3x^2.
\end{align*}
Net zo goed leveren $n = 2$, $a_0 = -5$, $a_1 = 7$ en $a_2 = -3$ een veelterm.
\end{example} 

\begin{exercise}
Schrijf zelf een veelterm op door de graad \(n\) en coëficiënten \(a_i\) te kiezen.
\end{exercise}

\begin{example} 
Beschouw de veeltermen
\[
A(x) = \sum_{i=0}^3 (2i+1) x^i \qquad \text{ en } \qquad P(t) = \sum_{i=1}^5 \frac{1}{i^2} \, t^i.
\] 
Deze veeltermen uitschrijven levert: 
\begin{align*}
A(x)  = (2\cdot0+1)x^0 + (2\cdot1+1)x^1 + (2\cdot2+1)x^2 + (2\cdot3+1)x^3 = 1 + 3x + 5x^2 + 7x^3
\end{align*}
en
\begin{align*}
P(x) 
& = \frac{1}{1^2} \, t^1 + \frac{1}{2^2} \, t^2 + \frac{1}{3^2} \, t^3 + \frac{1}{4^2} \, t^4 + \frac{1}{5^2} \, t^5 \\
& = t + \frac{1}{4}\,t^2 + \frac{1}{9}\,t^3 + \frac{1}{16}\,t^4 + \frac{1}{5}\,t^5.
\end{align*}
\end{example} 



\begin{example} 
De constante term van de veelterm $A(x) = -5 + 7x - 3x^2$ is gelijk aan $-5$.  
\end{example} 


\begin{example} 
Door deze veeltermen te vereenvoudigen kan telkens de graad, de hoogstegraadscoëfficiënt en constante term eenvoudig worden afgelezen. 
\begin{enumerate}[(a)]
\item
$A(x) = 1-4x+5x^3-7x^3 = 1 - 4x - 2x^3$ heeft graad $3$, hoogstegraadscoëfficiënt $-2$ en constante term $1$.
\item
$B(x) = 0x^2 + 3x = 3x$ heeft graad $1$, hoogstegraadscoëfficiënt $3$ en constante term $0$.
\item
$C(x) = \sqrt{7}-\sqrt{28} = \sqrt{7}-2\sqrt{7} = -\sqrt{7}$ heeft graad $0$, hoogstegraadscoëfficiënt $-\sqrt{7}$ en constante term $-\sqrt{7}$.
\end{enumerate}
\end{example} 


\begin{notation}

    De verzameling van alle veeltermen in $x$ wordt genoteerd met $\R[x]$. In symbolen:
    \[
    \R[x] = \{a_0 + a_1 x + a_2 x^2 + \dots + a_n x^n \mid \text{$n \in \N$ en $a_0,a_1,a_2,\ldots,a_n \in \R$} \}.
    \]
    Wegens een eerdere afspraak is $a = a\cdot x^0$, zodat elk reëel getal $a$ kan geschreven worden als een veelterm. De verzameling van de reële getallen is dus een deelverzameling
    van de verzameling van alle veeltermen, in symbolen: $\R \subseteq \R[x]$.
    
    
    \begin{example} 
       Zo zijn $1-4x+2x^3 \in \R[x]$, $x^5-\sqrt{2} \in \R[x]$ en $\D -\frac{13}{7} \in \R[x]$.   
    \end{example} 
    
\end{notation}



% DIT COMPILEERT NOG NIET 

\renewcommand{\TJa}{\makebox[2.5cm]{een veelterm }}
\renewcommand{\TNee}{\makebox[2.5cm]{eeen veelterm}}
  
 
\begin{exercise}
    \begin{question} $2x+3$                       is \choiceYes                               \begin{feedback} Elke eerstegraadsfunctie is een veelterm! $ax+b$ kan gezien worden als een eerstegraadsveelterm, met 'parameters' of 'constanten' $a$ en $b$ in plaats van concrete getallen.\end{feedback} \end{question}
    \begin{question} $5x^2+7x+1$                  is \choiceYes                               \begin{feedback} Elke tweedegraadsfunctie is een veelterm!                                                                                                                                    \end{feedback} \end{question}
    \begin{question} $x^{2025}-1$                 is \choiceYes   van graad $\answer{2025}$.                                                                                                                                                                                                               \end{question}
    \begin{question} $7$                          is \choiceYes   van graad \(\answer{0}\).    \begin{feedback} Elke reëel getal is een reële veelterm!                                                                                                                                      \end{feedback} \end{question}
    \begin{question} $\sqrt{\pi}x^5-e^y+\alpha^2$ is \choiceYes   in $\answer{x}$ van graad \(\answer{5}\), met parameters $y$ en $\alpha$.                                                                                                                                                                \end{question}
    \begin{question} $x^2-e^x$                    is \choiceNo                              \begin{feedback} Wegens \(x\) in de exponent is dit geen veelterm in \(x\)                                                                                                                    \end{feedback} \end{question}
    \begin{question} $t+\cos\alpha$               is \choiceYes   in $\answer{t}$                                                                                                                                                                                                                          \end{question}

                                                                                                                                                                                                                                                                                                                                                            
\end{exercise}
 
Zoals vorige oefeningen duidelijk maakt zijn alle eerstegraadsfuncties en tweedegraadsfuncties voorbeelden van veeltermen (met parameters). 
Je leerde reeds hoe je de nulpunten (bijvoorbeeld met de discriminantformule) van deze functies kan berekenen. 
Ook bij veeltermen voor willekeurige graad gaat de interesse van wiskundige uit naar de getallen die de veelterm nul maken. 


\begin{definition} De nulwaarde van een veelterm. 


    Door in een veelterm $A(x)$ de variabele $x$ te vervangen met een reëel getal $r$ bekomt men de \textbf{getalwaarde} $A(r)$ van $A(x)$ in $x = r$.  In symbolen:
    \[
    A(x) = a_0 + a_1x + a_2x^2 + \dots + a_n x^n \quad 
    \Rightarrow
    \quad A(r) = a_0 + a_1 r + a_2r^2 + \dots + a_n r^n
    \]
    met $n \in \N$ en $r, a_0, a_1, \ldots, a_n \in \R$, waarbij ook hier het geval $n = r = 0$ uitsloten wordt. % voor ons heeft $0^0$ geen betekenis.
    
    Is de getalwaarde van een veelterm in een reëel getal $r$ gelijk aan nul, dan is $r$ een \textbf{nulwaarde} van die veelterm. De uitspraak \textit{ $r$ is een nulwaarde van $A(x)$} is dus gelijkwaardig met de uitspraak \textit{ $A(r) = 0$}. Met behulp van het symbool $\Leftrightarrow$ voor equivalentie wordt dit in symbolen:
    \[
    r \text{ is een nulwaarde van } A(x) \quad \Leftrightarrow \quad A(r) = 0.
    \]
    Hierbij is $r \in \R$ en $A(x)$ een veelterm in $x$. In de literatuur wordt een nulwaarde soms ook een \textbf{nulpunt}genoemd. Onze voorkeur gaat uit naar de volgende afspraak: een nulpunt is een punt (dus een meetkundig object), een nulwaarde is een waarde (dus een getal).
    
\end{definition} 
    


\begin{example} 
    Als $A(x) = 2x^3+3x-5$ dan is de getalwaarde van $A(x)$ in $x = 10$ gelijk aan $A(10) = 2\cdot 10^3 + 3 \cdot 10 - 5 = 2025$.
\end{example} 

\begin{exercise}
    Is \(1\) een nulwaarde van de veelterm \( A(x) = x^{2025} + 7x^{1042} - 3x^{42} - 5\)? 
    \begin{oplossing}
    Om te controleren of \(1\) een nulwaarde van een veelterm is bereken je de som van de coëficiënten. In dit geval geldt is \(A(1) = 1 + 7 - 3 - 5 = 0\). Bijgevolg is \(1\) een nulwaarde van \(A(x)\).  
    \end{oplossing}
\end{exercise}



\begin{expandable}{remark}{De vruchtbare zoektocht naar nulpunten van veeltermen.}

    Op het eerste zicht lijken de nulwaarden van veeltermen geen bijzonder interessante objecten. In de geschiedenis van de wiskunde leverde deze zoektocht naar de nulpunten \textit{complexe getallen} op.  
    Evariste Galois leverde op 18jarige leeftijd een bewijs voor de de stelling dat er voor 5de graads veeltermen \textit{geen formule kan bestaan...}. 
    
\end{expandable}




\end{document}
