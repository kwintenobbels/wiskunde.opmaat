%%%%%%%%%%%%%%%%%%%%%%%%%%%%%%%%%%%%%%%%%%%%%%%%%%%%%%%%%%%%%%%%%%%%%%%%%%%%%%%%%%%%%%%%%%

% DIT MATERIAAL VERTROK VAN DE OPEN-SOURCE CURSUS VEELTERMEN VAN KOEN DE NAEGHEL         
% GEKOPIEËRD OP 24 MAART 2025                                                            
% ORIGINEEL BESCHIKBAAR VIA https://www.koendenaeghel.be/opensource.htm         

%%%%%%%%%%%%%%%%%%%%%%%%%%%%%%%%%%%%%%%%%%%%%%%%%%%%%%%%%%%%%%%%%%%%%%%%%%%%%%%%%%%%%%%%%%



\documentclass{ximera}
\input{../preamble}
\input{../preamblekdn}

\addPrintStyle{..}
\begin{document}
	\author{Koen de Naeghel - Wiskunde Op Maat}
	\xmtitle{Schema van Horner}{}
    \xmsource


% KOEN ZIJN MATHCOLORS STANDEN ZELF NOG TUSSEN {} --> DEZE KUNNEN WAARSCHIJNLIJK WEG 

\section{Schema van Horner}

Om in het algemeen het quotiënt en de rest bij deling van een veelterm $A(x)$ door een veelterm $B(x) \neq 0$ te vinden, voeren we het schema van de staartdeling uit. In het geval dat $B(x) = x-a$ voor een zekere $a \in \R$ kun je de rest veel sneller vinden door de reststelling toe te passen: de rest is gelijk aan $A(a)$. Hieronder leer je om ook het quotiënt op een meer efficiente manier te vinden. 

\begin{algorithm} 
Beschouw de veeltermen $A(x) = 3x^3+5x^2+8x+13$ en $B(x) = x-2$. Om quotiënt en rest te vinden, kunnen we de veelterm 
\textit{ de veelterm $B(x)$ forceren in de veelterm $A(x)$}, een manier die ook al in het vorige hoofdstuk aan bod kwam, zie Werkwijze \ref{werkwijze:staartdeling}:


\begin{align*}
3x^3+5x^2+8x+13
& = {\red{3}}x^2(x-2) + {\red{3}}\cdot 2x^2 + 5x^2 + 8x + 13 \\
& = {\red{3}}x^2(x-2) + {\green{11}}x^2 + 8x + 13 \\
& = {\red{3}}x^2(x-2) + {\green{11}}x(x-2) + {\green{11}}\cdot 2x + 8x + 13 \\
& = {\red{3}}x^2(x-2) + {\green{11}}x(x-2) + {\blue{30}}x + 13 \\
& = {\red{3}}x^2(x-2) + {\green{11}}x(x-2) + {\blue{30}}(x-2) + {\blue{30}} \cdot 2 + 13 \\
& = {\red{3}}x^2(x-2) + {\green{11}}x(x-2) + {\blue{30}}(x-2) + {\gray{73}} %\\
%& = (x-2) \cdot \left({\red{3}}x^2 + {\green{11}}x + {\blue{30}}\right) + {\gray{73}}.
\end{align*}
zodat $\D \underbrace{3x^3+5x^2+8x+13}_{A(x)} = \underbrace{(x-2)}_{B(x)}\cdot\underbrace{({\red{3}}x^2 + {\green{11}}x + {\blue{30}})}_{Q(x)} \,\, + \,\, \underbrace{{\gray{73}}}_{R(x)}$ waarbij $\gr R(x) < \gr B(x)$.

Dat schrijfwerk kan ingekort worden met het schema van de staartdeling uit Hoofdstuk \ref{hoofdstuk:2}. Omdat de deler $B(x)$ van de vorm $x-a$ is, kunnen we quotiënt en rest op een andere manier berekenen die nog minder schrijfwerk vergt: het zogenaamde \textit{ schema van Horner}.\footnote{De meest gangbare naam voor dit rekenschema is \textit{ synthetische deling}. Uit dat schema volgt dat de rest bij deling van een veelterm $A(x) = a_0 + a_1x + a_2x^2 + \dots + a_nx^n$ door $x-a$, dat wegens de reststelling gelijk is aan de getalwaarde van $A(x)$ in $x=a$, kan geschreven worden als $A(a) = ((a_n a +a_{n-1})a+\dots)a+a_0$. Het belang ervan is dat ze het aantal vermenigvuldigingen tot een minimum beperkt, wat leidt tot een grotere numerieke stabiliteit van de berekende waarden. Deze gemotiveerde schrijfwijze voor $A(a)$ werd voor het eerst in 1819 door William George Horner\index{Horner, William George} beschreven. In de Lage Landen wordt het ganse rekenschema van de synthetische deling ook het \textit{ schema van Horner} genoemd, maar dat is historisch gezien onjuist. Zie \cite{Cajori} en \cite{wiki:Hornerschema}.} 
\index{schema!van Horner}\index{Horner, schema van}
\renewcommand{\kolbreed}{\widthof{$2 \cdot 30$}}

\tikzit{
\(
\begin{array}{c|HHHH}
	& 3 & 5 & 8 & 13 \\[0.2cm]
2 & \downarrow  & 2 \cdot {\red{3}}  & 2 \cdot {\green{11}}  & 2 \cdot {\blue{30}}  \\[0.2cm]
\hline 
\vrule height 1.2em width 0pt 
	& {\red{3}} & {\green{11}} & {\blue{30}} & \multicolumn{1}{|c}{{\gray{73}}} 
\end{array}
\)
}
\end{algorithm} 

\begin{example} 
Beschouw de veelterm $A(x) = 2x^3 + 5x^2 - 3$. We bepalen met behulp van het schema van Horner het quotiënt en de rest bij deling van $A(x)$ door $x+3$. 
\renewcommand{\kolbreed}{\widthof{$-3 \cdot (-1)$}}
\tikzit{
\(
\begin{array}{c|HHHH}
	& 2 & 5 & 0 & -3 \\[0.2cm]
-3 & \downarrow  & -3 \cdot 2  & -3 \cdot (-1)  & -3 \cdot 3  \\[0.2cm]
\hline 
\vrule height 1.2em width 0pt 
	& 2 & -1 & 3 & \multicolumn{1}{|c}{-12} 
\end{array}
\)
}

Hieruit lezen we af dat het quotiënt $Q(x)$ gelijk is aan $2x^2-x+3$ en de rest $R(x)$ gelijk is aan $-12$. We controleren ons antwoord door de deler $x+3$ te vermenigvuldigen met $Q(x)$ en vervolgens $R(x)$ op te tellen:
\begin{align*}
(x+3)\cdot(2x^2-x+3) - 12 
& = 2x^3 - x^2 + 3x + 6x^2 - 3x + 9 - 12 \\
& = 2x^3 + 5x^2 - 3 = A(x).
\end{align*}
\end{example} 



Is een veelterm $A(x)$ deelbaar door $x-a$ en $x-b$, dan kan het schema van Horner twee keer na elkaar uitgevoerd worden om zo het quotiënt bij deling van $A(x)$ door $(x-a)(x-b)$ te bepalen. 

\begin{example} 
Gegeven is de veelterm $A(x) = x^4 - 3x^3 - 7x^2 - 9 x - 30$. Er is ook gegeven dat die veelterm deelbaar is door $(x-5)(x+2)$. Om het quotiënt van die deling te vinden, kunnen we als volgt te werk gaan.

Eerst zoeken we het quotiënt bij deling van $A(x)$ door $x-5$. Daartoe voeren we het schema van Horner uit.
\renewcommand{\kolbreed}{\widthof{$-30$}}

\tikzit{
\(
\begin{array}{c|HHHHH}
	& 1 & -3 & -7 & -9 & -30 \\[0.2cm]
5 & \downarrow  & 5  & 10  & 15 & 30  \\[0.2cm]
\hline 
\vrule height 1.2em width 0pt 
	& 1 & 2 & 3 & 6 & \multicolumn{1}{|c}{0}
\end{array}
\)
}


Hieruit lezen we af dat
\[
A(x) = (x-5)\underbrace{(x^3 + 2x^2 + 3x + 6)}_{Q(x)}.
\] 
Omdat $A(x)$ deelbaar is door $x+2$ (gegeven), is ook de veelterm $Q(x)$ deelbaar door $x+2$. Dat kunnen we inzien door in de bovenstaande gelijkheid elke $x$ te vervangen door $-2$:
\[
A(-2) = (-2-5)\cdot Q(-2)
\]
en omdat $(x+2) \mid A(x)$ is $A(-2) = 0$ (kenmerk van deelbaarheid) zodat $Q(-2) = 0$ en dus ook $(x+2) \mid Q(x)$. We bepalen nu het quotiënt bij deling van $Q(x)$ door $x+2$ door middel van een tweede schema van Horner.
\renewcommand{\kolbreed}{\widthof{$-6$}}

\tikzit{
\(
\begin{array}{c|HHHH}
	& 1 & 2 & 3 & 6 \\[0.2cm]
-2 & \downarrow  & -2  & 0  & -6  \\[0.2cm]
\hline 
\vrule height 1.2em width 0pt 
	& 1 & 0 & 3 & \multicolumn{1}{|c}{0}
\end{array}
\)
}


Op die manier vinden we dat $Q(x) = (x+2)(x^2+3)$ zodat uiteindelijk
\begin{align*}
A(x) 
& = (x-5)Q(x) \\
& = (x-5)(x+2)(x^2+3).
\end{align*}
Het quotiënt bij deling van $A(x)$ door $(x-5)(x+2)$ is dus gelijk aan $x^2+3$. 
\end{example} 


In het vervolg zullen een redenering zoals in het voorbeeld hierboven wat korter opschrijven: we voeren dan de twee schema's van Horner meteen na elkaar uit
\renewcommand{\kolbreed}{\widthof{$-30$}}

\tikzit{
\(
\begin{array}{c|HHHHH}
	& 1 & -3 & -7 & -9 & -30 \\[0.2cm]
5 & \downarrow  & 5  & 10  & 15 & 30  \\[0.2cm]
\hline 
\vrule height 1.2em width 0pt 
	& 1 & 2 & 3 & 6 & \multicolumn{1}{|c}{0} \\[0.2cm]
-2 & \downarrow  & -2  & 0  & -6 \\[0.2cm]
\cline{1-5}
\vrule height 1.2em width 0pt 
	& 1 & 0 & 3 & \multicolumn{1}{|c}{0}
\end{array}
\)
}

en lezen dan af dat  
\begin{align*}
x^4 - 3x^3 - 7x^2 - 9 x - 30 
& = (x-5)(x^3 + 2x^2 + 3x + 6) \\
& = (x-5)(x+2)(x^2+3).
\end{align*}

\begin{example} 
Gegeven is de veelterm $A(x) = 3 x^3 - x^2 - 7 x + 5$. Er is ook gegeven dat die veelterm deelbaar is door $x^2-2x+1$. Om het quotiënt van die deling te bepalen, merken we op dat de deler gelijk is aan $(x-1)^2 = (x-1)(x-1)$. We kunnen het quotiënt dus vinden door het schema van Horner twee keer na elkaar kunnen uitvoeren:
\renewcommand{\kolbreed}{\widthof{$-30$}}

\tikzit{
\(
\begin{array}{c|HHHH}
	& 3 & -1 & -7 & 5  \\[0.2cm]
1 & \downarrow  & 3  & 2  & -5   \\[0.2cm]
\hline 
\vrule height 1.2em width 0pt 
	& 3 & 2 & -5 & \multicolumn{1}{|c}{0} \\[0.2cm]
1 & \downarrow  & 3  & 5 \\[0.2cm]
\cline{1-5}
\vrule height 1.2em width 0pt 
	& 3 & 5 & \multicolumn{1}{|c}{0}
\end{array}
\)
}

waaruit we dan aflezen dat 
\begin{align*}
A(x) 
& = 3x^3 - x^2 - 7x + 5 \\
& = (x-1)(3x^2+2x-5) \\
& = (x-1)(x-1)(3x+5).
\end{align*}
\end{example} 

Meer algemeen kunnen we het schema van Horner ook gebruiken om quotiënt en rest bij deling van een veelterm door $B(x) = bx-a$ te vinden, waarbij $a,b \in \R$ met $b \neq 0$. 

\begin{example} 
Gegeven zijn de veeltermen $A(x) = 3x^3-8x^2+13x-7$ en $B(x) = 3x-2$. We beschouwen de deling van $A(x)$ door $B(x)$. Eerst herschrijven we het verband tussen deeltal, deler, quotiënt $Q(x)$ en rest $R(x)$ als volgt:
\begin{align*}
3x^3-8x^2+13x-7 
& = (3x-2)\cdot Q(x) + R(x) \\
& = \left(x-\frac{2}{3}\right) \cdot \,3Q(x) + R(x)
\end{align*}
waaruit we afleiden dat $3Q(x)$ en $R(x)$ het quotiënt en de rest zijn bij deling van $A(x)$ door $x - \frac{2}{3}$. Die vinden we door het schema van Horner op te schrijven.
\renewcommand{\kolbreed}{\widthof{$-8$}}

\tikzit{
\(
\begin{array}{c|HHHH}
	& 3 & -8 & 13 & -7 \\[0.2cm]
\D \frac{2}{3} & \downarrow  & 2  & -4  & 6  \\[0.2cm]
\hline 
\vrule height 1.2em width 0pt 
	& 3 & -6 & 9 & \multicolumn{1}{|c}{-1} 
\end{array}
\)
}

Hieruit lezen we af dat $R(x) = -1$ en $3Q(x) = 3x^2 - 6x + 9$ zodat $Q(x) = x^2-2x+3$.

We kunnen de controle uitvoeren door de deler met het quotiënt $Q(x)$ te vermenigvuldigen, en er daarna de rest $R(x)$ bij op te tellen:
\begin{align*}
B(x) \cdot Q(x) + R(x) 
& = (3x-2)(x^2-2x+3) - 1 \\
& = 3x(x^2-2x+3) - 2(x^2-2x+3) - 1 \\
& = 3x^3 - 6x^2 + 9x - 2x^2 + 4x - 6 - 1 \\
& =  3x^3-8x^2+13x-7 \\
& = A(x).
\end{align*}
\end{example} 





\end{document}