%%%%%%%%%%%%%%%%%%%%%%%%%%%%%%%%%%%%%%%%%%%%%%%%%%%%%%%%%%%%%%%%%%%%%%%%%%%%%%%%%%%%%%%%%%

% DIT MATERIAAL VERTROK VAN DE OPEN-SOURCE CURSUS VEELTERMEN VAN KOEN DE NAEGHEL         
% GEKOPIEËRD OP 24 MAART 2025                                                            
% ORIGINEEL BESCHIKBAAR VIA https://www.koendenaeghel.be/opensource.htm         

%%%%%%%%%%%%%%%%%%%%%%%%%%%%%%%%%%%%%%%%%%%%%%%%%%%%%%%%%%%%%%%%%%%%%%%%%%%%%%%%%%%%%%%%%%


\documentclass{ximera}

\input{../preamblekdn}

% voor figuren met PSTricks:
\usepackage{pstricks} 
\usepackage{pstricks-add}
\usepackage{pst-plot}
\usepackage{pst-node}
\usepackage{pst-coil}
\usepackage{auto-pst-pdf}


\addPrintStyle{..}
\begin{document}
	\author{Koen de Naeghel - Wiskunde Op Maat}
	\xmtitle{Veeltermen Algebraïsch bepalen van nulpunten}{}
    \xmsource

	\definecolor{graf}{RGB}{0,100,0} 

Om de nulwaarden van een veelterm \(A(x)\) te bepalen, moeten we een vergelijking oplossen, namelijk \(A(x) = 0\). Ontbinden in factoren is een strategie om zo'n vergelijking algebraïsch op te lossen, want zoals reeds aangetoond geldt voor twee veeltermen \(A(x)\) en \(B(x)\):

\[
A(x) \cdot B(x) = 0 \quad \Leftrightarrow \quad A(x) = 0 \,\, \text{ of } \,\, B(x)=0. 
\]  

\begin{example} 

We bepalen telkens algebraïsch de nulwaarden van de veelterm.


\begin{question} \( x^4-34x^2-2x^3-2x-35 = 0 \)

Na herschikken van de termen is dit gelijk aan \(\underbrace{x^4 - 2x^3 - 34x^2 - 2x - 35}_{A(x)} = 0\). 
\renewcommand{\kolbreed}{\widthof{\(-35\)}}


Delers van de constante term \(-35\) zijn kanshebbers gehele nulwaarden. Met ICT vinden we: \(A(7) = 0\) en \(A(-5) = 0\) dus \(A(x)\) is deelbaar door \((x-7)(x+5)\). Het schema van horner tweemaal uitvoeren levert: 

\tikzit{
\(
\begin{array}{c|HHHHH}
	& 1 & -2 & -34 & -2 & -35 \\[0.2cm]
7 & \downarrow  & 7  & 35  & 7 & 35  \\[0.2cm]
\hline 
\vrule height 1.2em width 0pt 
	& 1 & 5 & 1 & 5 & \multicolumn{1}{|c}{0} \\[0.2cm]
-5& \downarrow & -5 & 0 & -5 \\[0.2cm]
\cline{1-5}
\vrule height 1.2em width 0pt
	& 1 & 0 & 1 & \multicolumn{1}{|c}{0} 
\end{array} 
\)
}

De veelterm kan dus als volgt ontbonden worden: 
\[
\begin{array}{rl}
\D x^4 - 2x^3 - 34x^2 - 2x - 35 & \Leftrightarrow \quad (x-7)(x+5)(x^2+1) = 0 \\
\D & \Leftrightarrow \quad x - 7 = 0 \,\, \text{ of } \,\,  x + 5 = 0 \,\, \text{ of } \,\, x^2 + 1 = 0 \\
\D & \Leftrightarrow \quad x = 7 \,\, \text{ of } \,\,  x = -5 \,\, \text{ of } \,\, \underbrace{x^2}_{\geq 0} = \underbrace{-1}_{< 0} 
\end{array}
\]

De oplossingsverzameling is \( V = \{7,-5\} \).
\end{question} 

\begin{question}
\( \frac{1}{25}\,x^4-\frac{9}{25}\,x^3+\frac{14}{25}\,x^2+\frac{7}{5}\,x-1 = 0\) 

Beide leden vermenigvuldigen met \(25\) levert \(\D \underbrace{x^4 - 9x^3 + 14x^2 + 35x - 25}_{A(x)} = 0\). 
\renewcommand{\kolbreed}{\widthof{\(-25\)}}



Delers van de constante term \(-25\) zijn kanshebbers gehele nulwaarden. Met ICT vinden we: \(A(5) = 0\) dus \(A(x)\) is deelbaar door \((x-5)\). Het schema van horner tweemaal uitvoeren levert: 


\tikzit{
\(
\begin{array}{c|HHHHH}
	& 1 & -9 & 14 & 35 & -25 \\[0.2cm]
5 & \downarrow  & 5  & -20  & -30 & 25  \\[0.2cm]
\hline 
\vrule height 1.2em width 0pt 
	& 1 & -4 & -6 & 5 & \multicolumn{1}{|c}{0} \\[0.2cm]
5 & \downarrow & 5 & 5 & -5 \\[0.2cm]
\cline{1-5}
\vrule height 1.2em width 0pt
	& 1 & 1 & -1 & \multicolumn{1}{|c}{0} 
\end{array}
\)
}

De veelterm kan dus als volgt ontbonden worden: 

\(
\begin{array}{rl}
\D x^4 - 9x^3 + 14x^2 + 35x - 25 = 0 &\Leftrightarrow \quad (x-5)^2(x^2+x-1) = 0 \\
\D &\Leftrightarrow \quad x - 5 = 0 \,\, \text{ of } \,\,  x^2+x-1 = 0 \\
\D &\Leftrightarrow \quad x = 5 \,\, \text{ of } \,\,  x = \frac{-1 \pm \sqrt{5}}{2} \\
\end{array}
\)

De oplossingsverzameling is \( \D V = \left\{5,\frac{-1 + \sqrt{5}}{2},\frac{-1 - \sqrt{5}}{2} \right\} \).
\end{question}
\end{example} 


Elke veelterm \(A(x)\) kan opgevat worden als het voorschrift van een functie, waarvan we de grafiek kunnen plotten. De snijpunten van die grafiek met de \(x\)-as komen dan overeen met de oplossingen van de vergelijking \(A(x) = 0\), en dus met de nulwaarden van de veelterm. De ontbinding in factoren geeft de ligging van de grafiek ten opzichte van de \(x\)-as aan, en in welke punten de grafiek raakt aan de \(x\)-as. 




\begin{example} 
Gegeven is de functie \(f\) met als voorschrift \(f(x) = x^4 + x^3 - 13x^2-x\) waarvan de grafiek hieronder staat afgebeeld. Er worden ook drie snijpunten van de grafiek met de \(x\)-as aangeduid: de oorsprong \(O\) en de punten \(P\) en \(Q\). 

\medskip

% %%%%%%%%%%%%%%%%%%%%%%%%%%%%%%%%%%%%%%%%%%%%%%%%%%%%%%%%%%%%%%%%%%%%%%%%%
% \begin{center}
% \psset{xunit=0.525cm,yunit=0.525cm}
% \begin{pspicture}(-4.5,-4.99)(4,5.7) % co linksonder, co rechtsboven
% \psaxes[labels=none,ticks=none]{->}(0,0)(-4.5,-4.99)(4,5.7)
% \uput[l](0,5.7){\(y\)}
% \uput[d](4,0){\(x\)}
% \psplot[plotpoints=200,linewidth=0.4mm,linecolor=graf]{-4.5}{3.85}{x 4 add x 1 add mul x 1 neg add mul x 3 neg add mul 14 div 0.5 add 19 14 div neg add}
% \uput[r](-4.5,5){\color{graf} \(f\)}
% \uput[ur](0,0){\(O\)}
% \psline[linewidth=0.4mm,linecolor=graf]{*-*}(0,0)(0,0)
% \uput[ur](-4.1064,0){\(P\)}
% \psline[linewidth=0.4mm,linecolor=graf]{*-*}(-4.1064,0)(-4.1064,0)
% \uput[ul](3.1829,0){\(Q\)}
% \psline[linewidth=0.4mm,linecolor=graf]{*-*}(3.1829,0)(3.1829,0)
% \end{pspicture}
% \end{center}
% %%%%%%%%%%%%%%%%%%%%%%%%%%%%%%%%%%%%%%%%%%%%%%%%%%%%%%%%%%%%%%%%%%%%%%%%%

\begin{image}[0.3\textwidth]
	
	\begin{tikzpicture}[x=0.525cm, y=0.525cm]
		% Axes
		\draw[->] (-4.5,0) -- (4,0) node[below right] {\(x\)};
		\draw[->] (0,-4.99) -- (0,5.7) node[left] {\(y\)};
		\node[anchor=north east] at (0,0) {\(O\)};
		
		% Function plot
		\draw[green, thick, domain=-4.5:3.85, samples=200]
		plot(\x, {((\x+4)*(\x+1)*(\x-1)*(\x-3)/14 + 0.5 - 19/14)});
		
		% Function label
		\node[right, green] at (-4.5,5) {\(f\)};
		
		% Points P and Q on the x-axis
		\fill[green] (-4.1064,0) circle (1pt);
		\node[above left, green] at (-4.1064,0) {\(P\)};
		
		\fill[green] (3.1829,0) circle (1pt);
		\node[above left, green] at (3.1829,0) {\(Q\)};
		
		% Optional: origin dot (can be made invisible)
		\fill[green] (0,0) circle (1pt);
		
	\end{tikzpicture}
	
\end{image}


Het lijkt het erop dat de punten \(O\), \(P\) en \(Q\) de enige snijpunten van de grafiek met de \(x\)-as zijn. Meer bepaald, we hebben de indruk dat de grafiek van \(f\) de \(x\)-as raakt in de oorsprong. Schrijven we \(\co(P) = (a,0)\) en \(\co(Q) = (b,0)\) met \(a,b \in \R\) dan vermoeden we dat de tekentabel van \(f\) gegeven wordt door
\renewcommand{\kolbreed}{\widthof{\(a\)}}

\tikzit{
\(
\begin{array}{c|HHHHHHH}
x  & & a &  & 0 & & b & \\
\hline 
\vrule height 1.2em width 0pt 
f(x) & + & 0 & - & 0 & - & 0 & +
\end{array} 
\)
}

Om dat met zekerheid te weten, redeneren we als volgt. Alvast is 
\[
f(x) = x(x^3 + x^2 - 13x - 1)
\]
zodat \(f(0) = 0\), en de grafiek van \(f\) de \(x\)-as snijdt in de oorsprong.  Omdat \(f(a) = 0\) weten we dat \(f(x)\) deelbaar is door \(x-a\). Analoog is \(f(x)\) ook deelbaar door \(x-b\). Hieruit volgt dat 
\[
f(x) = x(x-a)(x-b)(x-c)
\]
voor een zeker reëel getal \(c\), en omdat \(abc = 1\) en \(a < 0\) en \(b > 0\) weten we dat \(c < 0\). Naast de drie snijpunten \(O(0,0)\), \(P(a,0)\) en \(Q(b,0)\) is er dus nog een vierde snijpunt \(R(c,0)\) van de grafiek van \(f\) met de \(x\)-as, die zich net links van de \(y\)-as situeert. De correcte tekentabel van de functie \(f\) wordt dus gegeven door
\renewcommand{\kolbreed}{\widthof{\(a\)}}

\tikzit{
\(
\begin{array}{c|HHHHHHHHH}
x  & & a & & c & & 0 & & b & \\
\hline 
\vrule height 1.2em width 0pt 
f(x) & + & 0 & - & 0 & + & 0 & - & 0 & +
\end{array} 
\)
}

Ter controle gebruiken we ICT om de veelterm \(x^4 + x^3 - 13x^2-x\) te ontbinden in factoren:
\[
f(x) = x(x+4,1064\ldots)(x-3,1829\ldots)(x+0,0765\ldots).
\]
De reden waarom de grafiek van \(f\) de \(x\)-as snijdt in de oosprong maar niet raakt in de oorsprong, is omdat \(f(x)\) deelbaar is door \(x-0\), maar niet door \((x-0)^2\). 
\end{example} 

	



\end{document}
