%%%%%%%%%%%%%%%%%%%%%%%%%%%%%%%%%%%%%%%%%%%%%%%%%%%%%%%%%%%%%%%%%%%%%%%%%%%%%%%%%%%%%%%%%%

% DIT MATERIAAL VERTROK VAN DE OPEN-SOURCE CURSUS VEELTERMEN VAN KOEN DE NAEGHEL         
% GEKOPIEËRD OP 24 MAART 2025                                                            
% ORIGINEEL BESCHIKBAAR VIA https://www.koendenaeghel.be/opensource.htm         

%%%%%%%%%%%%%%%%%%%%%%%%%%%%%%%%%%%%%%%%%%%%%%%%%%%%%%%%%%%%%%%%%%%%%%%%%%%%%%%%%%%%%%%%%%


\documentclass{ximera}
\input{../preamble}
\input{../preamblekdn}

\addPrintStyle{..}
\begin{document}
	\author{Koen de Naeghel - Wiskunde Op Maat}
	\xmtitle{Reststelling}{}
    \xmsource

In het vorige hoofdstuk kwam de deling van een veelterm $A(x)$ door een veelterm $B(x) \neq 0$ aan bod. In dit hoofdstuk bestuderen we het specifiek geval dat $B(x) = x-a$ met $a \in \R$. De delers $B(x) = x-a$ zullen veel eenvoudiger blijken, maar toch nog steeds veel 



De staartdeling geeft in het algemeen de rest bij deling van een veelterm $A(x)$ door een veelterm $B(x) \neq 0$ te vinden. In het geval dat $B(x) = x-a$ voor een zekere $a \in \R$ kun je die rest op een meer efficiënte manier vinden.

Beschouw bijvoorbeeld de deling van de veelterm $A(x) = x^3-2x^2+3x-1$ door $B(x) = x-2$. Het quotiënt en de rest bij deling van $A(x)$ door $B(x)$ bereken je met behulp van de staartdeling alsvolgt:


\tikzit{
\(
\begin{array}{l|l}
& x-2 \\
\cline{2-2}
\vrule height 1.2em width 0pt
& x^2+3 \\[-0.96cm]
\stackunder[0.05cm]{%
	\stackon[0pt]{x^3-2x^2+3x-1}{}%
}{%
	\Shortstack[r]{
	{x^3-2x^2\mph{+3x-1}}
	{\staartmin \staartstreep{x^3-2x^2+3x-1}}
	{3x-1}
	{3x-6} 
	{\staartmin \staartstreep{3x-6}}
	{5}
}
}  
\end{array}
\)
}


Uit dit schema volgt het onderstaand verband tussen deeltal, deler, quotiënt en rest.

\[
\underbrace{x^3-2x^2+3x-1}_{A(x)} = \underbrace{(x-2)}_{B(x)}\cdot\underbrace{(x^2+3)}_{Q(x)} \,\, + \,\, \underbrace{5}_{R(x)} 
\]

Deze rest $R(x)$ kan je ook vinden zonder het schema van de staartdeling. Merk eerst op dat die rest een getal moet zijn, uit de stelling van de euclidische deling volgt immers: 

\[
\gr R(x) < \underbrace{\gr B(x)}_{1} \quad \text{ of } \quad R(x) = 0
\]

zodat $\gr R(x) = 0$ of $R(x) = 0$, dus $R(x) = r$ voor een zekere $r \in \R$. Invullen levert:

\begin{align*}
A(x) & = (x-2) \cdot Q(x) + \underbrace{R(x)}_{r} \\
\Rightarrow \quad A(2) & = \underbrace{(2-2)}_{0} \cdot \, Q(2) + r = r. 
\end{align*}


Op deze manier kan de rest eenvoudig berekent worden: $R(x) = r = A(2) = 2^3 - 2 \cdot 2^2 + 3 \cdot 2 - 1 = 5$.



Deze vaststekkubg geeft een erg eenvoudige manier om meteen de rest bij deling van een veelterm door $x-a$ te vinden, zonder het schema van de staartdeling te hoeven uitvoeren. Dat resultaat wordt de \textit{ reststelling} genoemd.

\begin{theorem} (Reststelling)
Zij $A(x)$ een veelterm en $a \in \R$. Dan is de rest bij deling van $A(x)$ door $x-a$ gelijk aan $A(a)$.
\end{theorem} 


%\begin{expandable}{proof}{}

% Wegens de stelling van de euclidische deling bestaat er precies  één veelterm $Q(x)$ en één veelterm $R(x)$ zodat
% \[
% A(x) = (x-a)\cdot Q(x) + R(x) \quad \text{ waarbij } \quad \gr R(x) < \underbrace{\gr (x-a)}_{1} \,\, \text{ of } \,\, R(x) = 0.
% \]
% Hieruit volgt dat $\gr R(x) = 0$ of $R(x) = 0$. Dus is $R(x) = r$ voor een zekere $r \in \R$. \\
% Zo vinden we:
% \begin{align*}
% A(x) & = (x-a) \cdot Q(x) + \underbrace{R(x)}_{r} \\
% \Rightarrow \quad A(a) & = \underbrace{(a-a)}_{0} \cdot \, Q(a) + r = r. 
% \end{align*}
% We besluiten dat de rest bij deling van $A(x)$ door $x-a$ gelijk is aan $R(x) = r = A(a)$. 

% \end{expandable}


De reststelling bepaal je de rest bij deling van een veelterm door $x-a$ te bepalen. Volgend voorbeeld toont aan dat ook voor veeltermen van de vorm $bx-a$ de reststelling kan toegepast worden. 

\begin{example} 
Beschouw de veelterm $A(x) = 5x^7 - 3x^2 + 2$. Dan is, wegens de reststelling, de rest bij deling van $A(x)$ door $x+1$ gelijk aan

\[
A(-1) = 5 \cdot (-1)^7 - 3 \cdot (-1)^2 + 2 = -5-3+2 = -6.
\]

Pas de stelling van de euclidische deling toe om de rest bij deling van $A(x)$ door $2x-1$ te vinden: er bestaan (unieke) veeltermen $Q(x)$ en $R(x)$ waarvoor

\[
A(x) = (2x-1)\cdot Q(x) + R(x) \quad \text{ waarbij } \quad \gr R(x) < \underbrace{\gr (2x-1)}_{1} \,\, \text{ of } \,\, R(x) = 0
\]

zodat $R(x) = r \in \R$. Hieruit volgt:

\begin{align*}
A(x) & = (2x-1) \cdot Q(x) + \underbrace{R(x)}_{r} \\
\Rightarrow \quad A\left(\frac{1}{2}\right) & = \underbrace{\left(2 \cdot \frac{1}{2}-1\right)}_{0} \cdot \,Q\left(\frac{1}{2}\right) + r = r. 
\end{align*} 
Bijgevolg is de rest bij deling van $A(x)$ door $2x-1$ gelijk aan
\begin{align*}
r = A\left(\frac{1}{2}\right) & = 5 \cdot \left(\frac{1}{2}\right)^7 - 3 \cdot \left(\frac{1}{2}\right)^2 + 2 \\
& = \frac{5}{128} - \frac{3}{4} + 2 \\
& = \frac{5}{128} - \frac{96}{128} + \frac{256}{128} \\
& = \frac{165}{128}.
\end{align*}
\end{example} 



\end{document}
