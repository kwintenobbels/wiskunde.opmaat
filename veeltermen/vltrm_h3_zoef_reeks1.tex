\documentclass{ximera} 

\input{../preamble}
\input{../preamblekdn}

\addPrintStyle{..}

\begin{document}
	\author{Koen De Naeghel - Wiskunde Op Maat}
	\xmtitle{Oefeningen reeks 1}{}
    \xmsource

	\label{xim:veeltermen_deling_door_veelterm_oefeningen_reeks1}

%%\section*{Oefeningen reeks 1}


\begin{exercise}[\bf \ref{antw2.1}.]\setcounter{enumi}{1} 
	Ga telkens na of de eenterm deelbaar is door \(6x^6\) en geef in dat geval het quotiënt van de deling.
	\begin{xmmulticols}{2}
	
		\begin{question} \( 36x^8                    = \answer[onlineshowanswerbutton]{  } \) \end{question}
		\begin{question} \( -12x^6                   = \answer[onlineshowanswerbutton]{  } \) \end{question}
		\begin{question} \( 9x^{10}                  = \answer[onlineshowanswerbutton]{  } \) \end{question}
		\begin{question} \( 0                        = \answer[onlineshowanswerbutton]{  } \) \end{question}
		\begin{question} \( -24x^4                   = \answer[onlineshowanswerbutton]{  } \) \end{question}
		\begin{question} \( \D -\frac{3}{5}\,x^7     = \answer[onlineshowanswerbutton]{  } \) \end{question}
		\begin{question} \( \sqrt{2}\,x^6            = \answer[onlineshowanswerbutton]{  } \) \end{question}
		\begin{question} \( 42                       = \answer[onlineshowanswerbutton]{  } \) \end{question}
	\end{xmmulticols}
	\end{exercise}
	
\begin{exercise}[\bf \ref{antw2.2}.]\setcounter{enumi}{2} 
Ga telkens na of de veelterm deelbaar is door \(6\), door \(3x\) en door \(3x^2\); en geef in die gevallen telkens het quotiënt van de deling.
\begin{xmmulticols}{2}

	\begin{question} \( 12x^3-18x^2+6x           = \answer[onlineshowanswerbutton]{  } \) \end{question}
	\begin{question} \( 6x^2-3x                  = \answer[onlineshowanswerbutton]{  } \) \end{question}
	\begin{question} \( 0                        = \answer[onlineshowanswerbutton]{  } \) \end{question}
	\begin{question} \( \D 2x^2+\frac{3}{5}\,x^2 = \answer[onlineshowanswerbutton]{  } \) \end{question}
	\begin{question} \( \sqrt{3}\,x-8            = \answer[onlineshowanswerbutton]{  } \) \end{question}
	\begin{question} \( x^8-\pi\,x^5             = \answer[onlineshowanswerbutton]{  } \) \end{question}
\end{xmmulticols}
\end{exercise}
	
\begin{exercise}[\bf \ref{antw2.3}.]\setcounter{enumi}{3} 
Bepaal de veelterm waarbij de deling door \(-3x^2+2x-1\) als quotiënt \(x-3\) en als rest \(2x-1\) heeft. Werk de veelterm uit en vereenvoudig zoveel mogelijk.

\( \answer[onlineshowanswerbutton]{   } \) 
\end{exercise}

\begin{exercise}[\bf \ref{antw2.4}.]\setcounter{enumi}{4}  
	Gegeven is de veelterm \(P(x) = 4x^3 - px^2 + qx - 4\) waarbij \(p,q \in \R\). Bepaal de waarde(n) van de parameters \(p\) en \(q\) waarvoor de deling van \(P(x)\) door \(2x^2+3\) als quotiënt \(2x-1\) en als rest \(2x-1\) heeft. 
	\( \answer[onlineshowanswerbutton]{   } \) 
\end{exercise}

\begin{exercise}[\bf \ref{antw2.5}.]\setcounter{enumi}{5} 
Bepaal telkens het quotiënt en de rest bij deling van \(A(x)\) door \(B(x)\). Schrijf ook telkens het verband op tussen deeltal, deler, quotiënt en rest. Geef ook aan of \(A(x)\) deelbaar is door \(B(x)\).
	\begin{question} \(A(x) = 5x^2 + 3x\)     \quad \text{ en } \quad \(B(x) = x^2            =\answer[onlineshowanswerbutton]{  } \) \end{question}
	\begin{question} \(A(x) = x^2\)           \quad \text{ en } \quad \(B(x) = x              =\answer[onlineshowanswerbutton]{  } \) \end{question}
	\begin{question} \(A(x) = 5x^3+2\)        \quad \text{ en } \quad \(B(x) = 2x^2           =\answer[onlineshowanswerbutton]{  } \) \end{question}
	\begin{question} \(A(x) = -3\)            \quad \text{ en } \quad \(B(x) = 5              =\answer[onlineshowanswerbutton]{  } \) \end{question}
	\begin{question} \(A(x) = 7x^2-8x+5\)     \quad \text{ en } \quad \(B(x) = 3x^4+2x^2      =\answer[onlineshowanswerbutton]{  } \) \end{question}
	\begin{question} \(A(x) = 3x^2 - 5x + 3\) \quad \text{ en } \quad \(B(x) = x^2 - 5x + 25  =\answer[onlineshowanswerbutton]{  } \) \end{question}
	\begin{question} \(A(x) = 8-8x^2-8x^3\)   \quad \text{ en } \quad \(B(x) = 3x^2 - 3x + 3  =\answer[onlineshowanswerbutton]{  } \) \end{question}
\end{exercise}


\end{document}