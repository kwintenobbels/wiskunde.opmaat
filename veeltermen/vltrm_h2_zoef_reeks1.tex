\documentclass{ximera} 


%\input{../preamblekdn}

%\addPrintStyle{..}

\begin{document}
	\author{Koen De Naeghel - Wiskunde Op Maat}
	\xmtitle{Oefeningen reeks 1}{}
    \xmsource
	\label{xim:veeltermen_deling_door_xa_oefeningen_reeks1}

	%%\section*{Oefeningen reeks 1}


\begin{exercise}
Bepaal telkens de rest bij deling van \(A(x) = x^3-x+6\) door de gegeven veelterm.
\begin{xmmulticols}{2}

                                    
	\begin{question} \(x-3\)           \( \answer[onlineshowanswerbutton]{ 30            } \) \end{question}
	\begin{question} \(x+2\)           \( \answer[onlineshowanswerbutton]{ 0             } \) \end{question}
	\begin{question} \(x + 2\sqrt{2}\) \( \answer[onlineshowanswerbutton]{ 6-14\sqrt{2}  } \) \end{question}
	\begin{question} \(x\)             \( \answer[onlineshowanswerbutton]{6              } \) \end{question}
	\begin{question} \(3x-5\)          \( \answer[onlineshowanswerbutton]{\frac{242}{27} } \) \end{question}
	\begin{question} \(-1\)            \( \answer[onlineshowanswerbutton]{ 0             } \) \end{question}

\end{xmmulticols}
\end{exercise}



\begin{exercise}
{\bf (toelatingsexamen arts)}\index{toelatingsexamen arts} 
Gegeven zijn de reële getallen \(a\) en \(b\). Bij deling van de veelterm \(P(x) = x^2 + bx + ab\) door \(x+a\) is de rest gelijk aan
\begin{xmmulticols}{4}
	\begin{enumerate}
		\item \(a^2 \)
		\item \(b-a \)
		\item \(a-b \)
		\item \(b   \)
	\end{enumerate}
	\( \answer[onlineshowanswerbutton]{\text{A}} \)                                                            

\end{xmmulticols}
\end{exercise}

\begin{exercise}
Ga telkens na of de gegeven veelterm een deler is van \(A(x) = x^4 - 10x^2 + 9\). 
\begin{xmmulticols}{2}


	\begin{question} \( x-2  =\answer[onlineshowanswerbutton]{ \text{geen deler}} \) \end{question}                                                              
	\begin{question} \( x-3  =\answer[onlineshowanswerbutton]{ \text{deler     }} \) \end{question}                                                              
	\begin{question} \( x+1  =\answer[onlineshowanswerbutton]{ \text{deler     }} \) \end{question}                                                              
	\begin{question} \( x    =\answer[onlineshowanswerbutton]{ \text{geen deler}} \) \end{question}                                                                

\end{xmmulticols}
\end{exercise}

\begin{exercise}
Bepaal telkens het quotiënt bij deling van \(A(x)\) door \(B(x)\).% met behulp van het schema van Horner.


	\begin{question} \( A(x) = 2x^3+5x^2-3\) \quad en \quad \(B(x) = x-3                   =\answer[onlineshowanswerbutton]{\text{quotiënt \(2x^2+11x+33\) en rest \(96\)       }} \) \end{question}.                  
	\begin{question} \( A(x) = x^5 - x^4 + x^3 - x^2 + x - 1\) \quad en \quad \(B(x) = x+1 =\answer[onlineshowanswerbutton]{\text{quotiënt \(x^4-2x^3+3x^2-4x+5\) en rest \(-6\)}} \) \end{question} 

\end{exercise}



\end{document}