%%%%%%%%%%%%%%%%%%%%%%%%%%%%%%%%%%%%%%%%%%%%%%%%%%%%%%%%%%%%%%%%%%%%%%%%%%%%%%%%%%%%%%%%%%

% DIT MATERIAAL VERTROK VAN DE OPEN-SOURCE CURSUS VEELTERMEN VAN KOEN DE NAEGHEL         
% GEKOPIEËRD OP 24 MAART 2025                                                            
% ORIGINEEL BESCHIKBAAR VIA https://www.koendenaeghel.be/opensource.htm         

%%%%%%%%%%%%%%%%%%%%%%%%%%%%%%%%%%%%%%%%%%%%%%%%%%%%%%%%%%%%%%%%%%%%%%%%%%%%%%%%%%%%%%%%%%


\documentclass{ximera}
\input{../preamble}
\input{../preamblekdn}

\addPrintStyle{..}
\begin{document}
	\author{Koen de Naeghel - Wiskunde Op Maat}
	\xmtitle{Parameters}{}
    \xmsource



In wiskundige uitdrukkingen kunnen symbolen voorkomen zoals \(a\), \(b\), \(p\) en \(q\) die getallen voorstellen. Pas als die symbolen een waarde toegekend krijgen, wordt de uitdrukking volledig vastgelegd. Zo'n symbool noemt men een \textit{ parameter}.\index{parameter} 

Een triviaal voorbeeld is de eerstegraadsfunctie \(y = x + b\) met parameter \(b\). Dit is de identieke functie \(y = x\) waarbij de parameter \(b\) een verticale verschijving weergeeft. Als je \(b = 0\) invult, bekom je de identieke functie \(y = x\). Voor de waarde \(b =3\) bekom je de functie \(y = x+3\). Dit is de identieke functie drie eenheden vertikaal verschoven. Een eenvoudige observatie levert dat \(y(0) = 0 + b = b\). Grafisch komt in dit voorbeeld de parameter \(b\) dus overeen met het snijpunt van de rechte met de \(y\)-as. 

Niks houdt ons tegen om in de algemene definitie van een veelterm \(A(x) = \sum_{i=0}^n a_i x^i = a_0 + a_1x + a_2x^2 + \dots + a_n x^n\) de coëficiënten \(a_0, a_1, a_2, \ldots, a_n\)  als parameter te beschouwen.  

\begin{example} 
Beschouw \(A(x) = ax^5 - 4x^2 + bx + 2\) en \(B(x) = 7x^m + 2ax - c^2x^2 - d\) waarbij \(a,b,c,d \in \R\) en \(m \in \N\). In dit voorbeeld worden de waarde(n) van de parameters \(a,b,c,d\) en \(m\) bepaald waarvoor de veelterm \(A(x)\) gelijk is aan de veelterm \(B(x)\): 
\begin{align*}
A(x) = B(x) \quad 
& \Leftrightarrow \quad ax^5 - 4x^2 + bx + 2 = 7x^m + 2ax - c^2x^2 - d \\
& \Leftrightarrow \quad 5 = m \,\,\text{ en }\,\, a = 7 \,\,\text{ en }\,\, -4 = -c^2 \,\,\text{ en }\,\, b = 2a \,\,\text{ en }\,\, 2 = -d \\
& \Leftrightarrow \quad m=5 \,\,\text{ en }\,\, a = 7 \,\,\text{ en }\,\, c^2 = 4 \,\,\text{ en }\,\, b = 14 \text{ en } d = -2 \\
& \Leftrightarrow \quad 
\left\{
\begin{aligned}
& m=5 \,\,\text{ en }\,\, a = 7 \,\,\text{ en }\,\, c = 2 \,\,\text{ en }\,\, b = 14 \text{ en } d = -2 \\
& \text{of} \\
& m=5 \,\,\text{ en }\,\, a = 7 \,\,\text{ en }\,\, c = -2 \,\,\text{ en }\,\, b = 14 \text{ en } d = -2.
\end{aligned}
\right.
\end{align*}
\end{example} 

% HIER MOET EEN LINK KOMEN NAAR HET OPLOSSEN VAN STELSELS ZODRA DAT IS GEMAAKT BIJ REKENVAARDIGHEDEN 
Bepalen van de waarde(n) van parameters kan leiden tot een stelsel. 
In het derde jaar heb je geleerd hoe je een stelsel van twee vergelijkingen in twee onbekenden kan oplossen. Die technieken moet je nog steeds kunnen toepassen. 



\begin{example} 
Bepaal de waarde(n) van de parameters \(a\) en \(b\) waarvoor
\[
a(x-3) - b(x+1) = 1-3x.
\]
Met deze voorwaarden kan je een stelsel opstellen:
\begin{align*}
a(x-3) - b(x+1) = 1-3x \quad 
& \Leftrightarrow \quad ax-3a-bx-b=1-3x \\
& \Leftrightarrow \quad (a-b)x +(-3a-b) = 1-3x \\
& \Leftrightarrow \quad
\left\{
\begin{aligned}
a - b & = -3 && (1)\\
-3a - b & = 1. && (2)
\end{aligned}
\right.
\end{align*}
Uit \((1)-(2)\) volgt \(4a = -4\) zodat \(a = -1\). Invullen in \((2)\) geeft \(-1-b =  -3\) waaruit \(b = 2\). 
\end{example} 



% OOK HIER MOET EEN OEFENING NAAR REKENVAARDIGHEDEN 
Bij een rekenoefening kan gevraagd worden om de berekening \textit{ algebraïsch} uit te voeren: met de hand, waarbij je jouw tussenstappen met berekeningen opschrijft. Bij het algebraïsch rekenwerk is het ook de bedoeling om het resultaat te vereenvoudigen. Je moet dus vlot kunnen rekenen met vierkantswortels en derdemachtswortels. Indien gevraagd maak je ook de noemers (vierkants)wortelvrij. 


Probeer eerst op een kladblad volgend voorbeeld zelf uit te werken alvorens heel de oplossing te lezen. 

\begin{example} 
Gegeven zijn de veeltermen
\[
A(x) = \frac{1}{5}\,x^2-3x+\frac{2}{3}, \quad B(x) = 3(bx)^2-6x+1 \quad \text{ en } \quad C(x) = x^3 + \sqrt[3]{2} x^2 + \sqrt[3]{4}x+5
\]
waarbij \(b \in \R\). 


\begin{question}

De getalwaarde van de veelterm \(A(x)\) in \(x = -2\) is gelijk aan
\[
A(-2) = \frac{1}{5}\,\cdot (-2)^2-3\cdot(-2)+\frac{2}{3} = \frac{4}{5} +6+ \frac{2}{3} = \frac{12 + 90 + 10}{15} = \frac{112}{15}.
\]
\end{question}

\begin{question}
We bepalen de waarde (n) van de parameter \(b\) waarvoor \(B(4) = 6\). We hebben:
\begin{align*}
B(4) = 6 \quad 
& \Leftrightarrow \quad 3(b\cdot 4)^2-6\cdot 4+1 = 6 \\ 
& \Leftrightarrow \quad 48b^2 = 29 \\
& \Leftrightarrow \quad b = \sqrt{\frac{29}{48}} \,\,\text{ of }\,\, b = - \sqrt{\frac{29}{48}} \\
& \Leftrightarrow \quad b = \frac{\sqrt{29}}{4\sqrt{3}} \,\,\text{ of }\,\, b = - \frac{\sqrt{29}}{4\sqrt{3}}
\end{align*}
\end{question}

\begin{question}
De getalwaarde van de veelterm \(C(x)\) in \(x = \sqrt[3]{2}\) is gelijk aan
\begin{align*}
C(\sqrt[3]{2}) & = (\sqrt[3]{2})^3 + \sqrt[3]{2}\cdot(\sqrt[3]{2})^2 + \sqrt[3]{4}\cdot \sqrt[3]{2}+5 \\
& = 2 + (\sqrt[3]{2})^3 + \sqrt[3]{2^2}\cdot \sqrt[3]{2} + 5 \\
& = 2 + 2 + (\sqrt[3]{2})^2 \cdot \sqrt[3]{2} + 5 \\
& = 2 + 2 + 2 + 5 \\
& = 11.
\end{align*}
\end{question}
\end{example} 





\end{document}