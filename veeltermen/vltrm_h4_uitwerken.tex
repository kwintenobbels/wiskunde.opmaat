%%%%%%%%%%%%%%%%%%%%%%%%%%%%%%%%%%%%%%%%%%%%%%%%%%%%%%%%%%%%%%%%%%%%%%%%%%%%%%%%%%%%%%%%%%

% DIT MATERIAAL VERTROK VAN DE OPEN-SOURCE CURSUS VEELTERMEN VAN KOEN DE NAEGHEL         
% GEKOPIEËRD OP 24 MAART 2025                                                            
% ORIGINEEL BESCHIKBAAR VIA https://www.koendenaeghel.be/opensource.htm         

%%%%%%%%%%%%%%%%%%%%%%%%%%%%%%%%%%%%%%%%%%%%%%%%%%%%%%%%%%%%%%%%%%%%%%%%%%%%%%%%%%%%%%%%%%



\documentclass{ximera}
\input{../preamble}
\input{../preamblekdn}

\addPrintStyle{..}
\begin{document}
	\author{Koen de Naeghel - Wiskunde Op Maat}
	\xmtitle{Veeltermen uitwerken}{}
    \xmsource





In dit laatste hoofdstuk worden drie vragen bekijken die je telkens over een veelterm \(A(x)\) kan stellen. 
Om deze vragen te beantwoorden wordt telkens leerstof uit vorige hoofdstukken gebruikt. 

\begin{enumerate}[(1)]
\item
Een veelterm uitwerken: schrijven als een som van eentermen. 
\item
Een veelterm ontbinden in factoren: schrijven als een product van veeltermen.
\item
Nulwaarden bepalen van een veelterm \(A(x)\): alle \(x \in \R\) bepalen waarvoor \(A(x) = 0\). 
\end{enumerate}




Het uitwerken van een veelterm is eenvoudig: je maakt simpelweg gebruik van de rekenregels en tekenregels voor som, verschil en product. Het is de gewoonte om daarna de termen te herschikken naar toenemende of afnemende graad. Zo is bijvoorbeeld

\begin{align*}
7x^3-2(5x^3-5)-(x^3-8)x^2 & = 7x^3 - 10x^3 + 10 - x^5 + 8x^2 \\
& = -x^5 - 3x^3 + 8x^2 + 10.
\end{align*}

Soms kun je het rekenwerk verkorten door middel van merkwaardige producten
waarvan de belangrijkste hieronder staan. Je kan elke gelijkheid aantonen door het linkerlid uit te werken en daarna te vereenvoudigen. We bewijzen één gelijkheid. De andere bewijzen laten we over als oefening voor de lezer. 

\begin{proposition} 
Zij \(a,b,c \in \R\). Dan geldt:
\begin{align*}
& (a+b)^2 = a^2+2ab+b^2 && \text{kwadraat van som van twee termen} \\
& (a-b)^2 = a^2-2ab+b^2 && \text{kwadraat van verschil van twee termen} \\
& (a-b)(a+b) = a^2-b^2 && \text{product van tweeterm met zijn toegevoegde} \\
& (a+b+c)^2 = a^2 + b^2 + c^2 + 2ab + 2bc + 2ca && \text{kwadraat van som van drie termen} \\
& (a+b)^3 = a^3+3a^2b+3ab^2+b^3 && \text{derde macht van som van twee termen} \\
& (a-b)^3 = a^3-3a^2b+3ab^2-b^3 && \text{derde macht van verschil van twee termen.}
\end{align*}
\end{proposition} 


\begin{expandable}{proof}{Bewijs van merkwaardig poduct voor kwadraat van som van drie termen}

\begin{align}
(a+b+c)^2 
& = (a+b+c)(a+b+c) \nonumber \\
& = a^2 + ab + ac + ba + b^2 + bc + ca + cb + c^2 \nonumber \\
& = a^2 + b^2 + c^2 + 2ab + 2bc + 2ca 
\end{align}

\end{expandable}


\begin{example} 
We werken telkens de veelterm uit door gebruik te maken van merkwaardige producten, en vereenvoudigen nadien zoveel als mogelijk.

\begin{question} \(\D (-3x+5)^2 = (5-3x)^2 = 5^2 - 2 \cdot 5 \cdot 3x + (3x)^2 = 9x^2 - 30x + 25\) \end{question} 
\begin{question} \(\D \left(\frac{1}{2}\,x-3\right)\left(3+\frac{1}{2}\,x\right) = \left(\frac{1}{2}\,x-3\right)\left(\frac{1}{2}\,x+3\right) = \left(\frac{1}{2}\,x\right)^2 - 3^2 = \frac{1}{4}\,x^2 - 9\) \end{question}
\begin{question} \(\D (4x^2 - 5x + 3)^2 = (4x^2)^2 + (-5x)^2 + 3^2 + 2 \cdot 4x^2 \cdot (-5x) + 2\cdot (-5x) \cdot 3 + 2 \cdot 4x^2 \cdot 3 = 16x^4 - 40x^3 + 49x^2 - 30x + 9\) \end{question}

% \(\D \mph{(4x^2 - 5x + 3)^2} = 16x^4 + 25x^2 + 9 -40x^3 - 30x + 24x^2\) TUSSENSTAP WEGGELATEN 
% $\D \mph{(4x^2 - 5x + 3)^2} = 16x^4 - 40x^3 + 49x^2 - 30x + 
\end{example} 





	
	
	
	
	
	
\end{document}