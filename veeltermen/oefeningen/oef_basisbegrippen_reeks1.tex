\documentclass{ximera} 

\input{../../preamble.tex}
\input{../../preamblekdn.tex}

\addPrintStyle{..}

\begin{document}
	\author{Koen De Naeghel - Wiskunde Op Maat}
	\xmtitle{Oefeningen reeks}{}
	\xmsource
	\label{xim:veeltermen_basisbegrippen_oefeningen_reeks1}


\begin{exercise}
\hypertarget{oef1.1}{Welke} uitdrukkingen stellen na vereenvoudiging een eenterm in $x$ voor? 
\renewcommand{\TJa }{\makebox[2.5cm]{Een eenterm }}
\renewcommand{\TNee}{\makebox[2.5cm]{Geen eenterm}}

\begin{multicols}{2}
	\begin{question} \choiceYes  \( x^{17}                         \)   \end{question}
	\begin{question} \choiceYes  \( \D -\frac{17}{3}\,x^2          \)   \end{question}
	\begin{question} \choiceNo   \( x^{-1}                         \)   \end{question}
	\begin{question} \choiceYes  \( \D x + 5\cdot \frac{x}{3}      \)   \end{question}
	\begin{question} \choiceNo   \( x + x^2                        \)   \end{question}
	\begin{question} \choiceYes  \( \D \frac{6}{5}\cdot\frac{3}{8} \)   \end{question}
	\begin{question} \choiceYes  \( (2x-3)(2x+3)+9                 \)   \end{question}
	\begin{question} \choiceYes  \( \sqrt{2}\,x^4 + \sqrt{3}\,x^4  \)   \end{question}
	\begin{question} \choiceNo   \( \sqrt{x}                       \)   \end{question}
	\begin{question} \choiceNo   \( \sqrt{x^2}                     \)   \end{question}
\end{multicols}
\end{exercise}

\begin{exercise}
\hypertarget{oef1.2}{Schrijf} de volgende veeltermen uit.

% \, is een 'kleine spatie'
	\begin{question} \( \D A(x) = \sum_{i=0}^3 ix^i                  =\answer[onlineshowanswerbutton]{x + 2x^2 + 3x^3  } \) \end{question}
	\begin{question} \( \D B(x) = \sum_{n=0}^3 \frac{x^{2n+1}}{n+1}  =\answer[onlineshowanswerbutton]{x + \frac{1}{2}\,x^3 + \frac{1}{3}\,x^5 + \frac{1}{4}\,x^7  } \) \end{question}
	\begin{question} \( \D C(x) = \sum_{i=2}^4 (-x)^i                =\answer[onlineshowanswerbutton]{x^2 - x^3 + x^4  } \) \end{question}
	% \begin{question} \( \D A(x) = \sum_{i=0}^3 ix^i                  =\answer[onlineshowanswerbutton]{x + 2x^2 + 3x^3  } \) \end{question}
	% \begin{question} \( \D B(x) = \sum_{n=0}^3 \frac{x^{2n+1}}{n+1}  =\answer[onlineshowanswerbutton]{x + \frac{1}{2}\,x^3 + \frac{1}{3}\,x^5 + \frac{1}{4}\,x^7  } \) \end{question}
	% \begin{question} \( \D C(x) = \sum_{i=2}^4 (-x)^i                =\answer[onlineshowanswerbutton]{x^2 - x^3 + x^4  } \) \end{question}

\end{exercise}


\begin{exercise}
\hypertarget{oef1.3}{Zijn} volgende uitdrukkingen na vereenvoudiging een veelterm in $x$? 
% \hypertarget{oef1.3}{Welke} uitdrukkingen stellen na vereenvoudiging een veelterm in $x$ voor? 
\renewcommand{\TJa }{\makebox[2.5cm]{Veelterm}}
\renewcommand{\TNee}{\makebox[2.5cm]{Geen veelterm}}
\begin{multicols}{2}

	\begin{question} \choiceYes \( 5x^4-3x^3+2x^2+x-1                    \)   \end{question}
	\begin{question} \choiceYes \( \D \frac{1}{2}\,x^3 + 3x^2-\pi\,x+2   \)   \end{question}
	\begin{question} \choiceYes \( \D \frac{1}{2}+\frac{1}{3}            \)   \end{question}
	\begin{question} \choiceYes \( x                                     \)   \end{question}
	\begin{question} \choiceYes \( \sqrt{2}\,x + \sqrt{3}                \)   \end{question}
	\begin{question} \choiceYes \( \D \frac{1-x^2}{2}                    \)   \end{question}
	\begin{question} \choiceNo  \( \D 1-\frac{2}{x^2}                    \)   \end{question}
	\begin{question} \choiceYes \( (x-4)(x^2+2x+3)                       \)   \end{question}
	\begin{question} \choiceYes \( (-2x^2+5x-1)^{9}                      \)   \end{question}
	\begin{question} \choiceNo  \( \sqrt{9x^4+25}                        \)   \end{question}
	\begin{question} \choiceYes \( \sqrt{9+25}                           \)   \end{question}
	\begin{question} \choiceYes \( \sqrt{x^4}                            \)   \end{question}

\end{multicols}
\end{exercise}


% van deze oefening heeft koen geen eindoplossingen!!! 

\begin{exercise} % oef 4
Bepaal telkens de graad, de hoogstegraadscoëfficiënt en de constante term.

\def\myopts[#1]{%
\def\isO{}%
\def\isI{}%
\def\isII{}%
\def\isIII{}%
\def\isIV{}%
\def\isV{}%
\def\isVI{}%
%
\expandafter\def\csname is#1\endcsname{correct}%
\quad%
\wordChoice{%
	\choice[\isO]  {$0$}%
	\choice[\isI]  {$1$}%
	\choice[\isII] {$2$}%
	\choice[\isIII]{$3$}%
	\choice[\isIV] {$4$}%
	\choice[\isV]  {$5$}%
	\choice[\isVI] {$6$}%
}%
}

\newlength{\xmQuestionSepWidth}
\setlength{\xmQuestionSepWidth}{4.2cm}

\providecommand{\A}{}
\renewcommand{\A}{\tabto{\xmQuestionSepWidth}}
\renewcommand{\choiceminimumverticalsize}{\vphantom{$2$}} 

	\begin{question} \( x^4-3x^2+2x-1             \)  \A heeft graad \myopts[IV],  hoogstegraadscoëfficient $\answer{ 1 }$ en constante term $\answer{1}$. \end{question}
	\begin{question} \( \sqrt{5}\,x^3-2x          \)  \A heeft graad \myopts[III], hoogstegraadscoëfficient $\answer{\sqrt{5}   }$ en constante term $\answer{0}$. \end{question}
	\begin{question} \( \D x+\frac{1}{3}\,x^3+x^2 \)  \A heeft graad \myopts[III], hoogstegraadscoëfficient $\answer{\frac{1}{3}}$ en constante term $\answer{0}$. \end{question}
	\begin{question} \( 5x+2                      \)  \A heeft graad \myopts[I],   hoogstegraadscoëfficient $\answer{ 5 }$ en constante term $\answer{ 2}$. \end{question}
	\begin{question} \( 2                         \)  \A heeft graad \myopts[O],   hoogstegraadscoëfficient $\answer{ 2 }$ en constante term $\answer{ 2}$. \end{question}
	\begin{question} \( (2x^3-7)^2                \)  \A heeft graad \myopts[VI],  hoogstegraadscoëfficient $\answer{ 4 }$ en constante term $\answer{49}$. \end{question}
	\begin{question} \( (-5x+7)(3x^2-5x+8)        \)  \A heeft graad \myopts[III], hoogstegraadscoëfficient $\answer{-15}$ en constante term $\answer{56}$. \end{question}
	\begin{question} \( (x-2)^5                   \)  \A heeft graad \myopts[V],   hoogstegraadscoëfficient $\answer{ 1 }$ en constante term $\answer{32}$. \end{question}
\end{exercise}

\end{document}