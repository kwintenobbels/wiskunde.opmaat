\documentclass{ximera} 

\input{../../preamble.tex}
\input{../../preamblekdn.tex}

\addPrintStyle{..}

\begin{document}
	\author{Koen De Naeghel}
	\xmtitle{Oefeningen reeks 2}{}
	\label{xim:veeltermen_toepassingen_oefeningen_reeks2}
%%\section*{Oefeningen reeks 2}

\begin{exercise}[\bf \ref{antw4.5}.]\setcounter{enumi}{5} 
\hypertarget{oef4.5}{Ontbind} telkens de veelterm zo ver mogelijk in factoren. Ga algebra\"isch te werk en schrijf jouw redenering uit.  
\begin{xmmulticols}{2}

	\begin{question} $6x^2+x-15$                        \( \answer[onlineshowanswerbutton]{(6x+40)(x-3)                                       } \) \end{question}
	\begin{question} $\D \frac{1}{27} - \frac{x^3}{8}$  \( \answer[onlineshowanswerbutton]{ \frac{1}{216}\,(2-3x)(4+6x+9x^2)                  } \) \end{question}
	\begin{question} $2x^5-7x^4-38x^3$                  \( \answer[onlineshowanswerbutton]{\frac{1}{8}\,x^3(4x-7-\sqrt{353})(4x-7+\sqrt{353}) } \) \end{question}
	\begin{question} $2x^2 - 6073x + 4\,098\,600$       \( \answer[onlineshowanswerbutton]{(x-2024)(2x-2025)                                  } \) \end{question}
	\begin{question} $\D 1-x^2+\frac{1}{4}\,x^4$        \( \answer[onlineshowanswerbutton]{\frac{1}{4}\,(x-\sqrt{2})^2(x+\sqrt{2})^2          } \) \end{question}
	\begin{question} $2x^3 - 15x^2 + 19x - 6$           \( \answer[onlineshowanswerbutton]{(x-1)(x-6)(2x-1)                                   } \) \end{question}
\end{xmmulticols}
\end{exercise}


\begin{exercise}[\bf \ref{antw4.6}.]\setcounter{enumi}{6}  
\hypertarget{oef4.6}{Bepaal} telkens algebra\"isch alle nulwaarden van de veelterm. Schrijf jouw redenering uit.
\begin{xmmulticols}{2}
	\begin{question} $3x^2-11\sqrt{3}\,x+30$                   \( \answer[onlineshowanswerbutton]{V = \left\{ 2\sqrt{3},\frac{5\sqrt{3}}{3} \right\}                    } \) \end{question}
	\begin{question} $12x^4-5x^3-28x^2$                        \( \answer[onlineshowanswerbutton]{V = \left\{0,\frac{7}{4},-\frac{4}{3}\right\}                         } \) \end{question}
	\begin{question} $x^3-5x+4$                                \( \answer[onlineshowanswerbutton]{V = \left\{ 1, \frac{\sqrt{17}-1}{2}, -\frac{\sqrt{17}+1}{2} \right\} } \) \end{question}
	\begin{question} $2x^3 - 5x^2 - 39x - 18$                  \( \answer[onlineshowanswerbutton]{V = \left\{-3,6,-\frac{1}{2}\right\}                                  } \) \end{question}
	\begin{question} $x^4+8x^3+17x^2-6x-36$                    \( \answer[onlineshowanswerbutton]{V = \left\{-3,-1+\sqrt{5}, -1-\sqrt{5}\right\}                        } \) \end{question}
	\begin{question} $2x^3+x^2-17x-12$                         \( \answer[onlineshowanswerbutton]{V = \left\{ 3, \frac{\sqrt{17}-7}{4}, -\frac{\sqrt{17}+7}{4} \right\} } \) \end{question}
	\begin{question} $3x^5-30x^4+120x^3-240x^2+240x-96$        \( \answer[onlineshowanswerbutton]{V = \{ 2 \}                                                           } \) \end{question}
	\begin{question} $x^3 - 3\sqrt{2}\,x^2 + 6x - 2\sqrt{2}\,$ \( \answer[onlineshowanswerbutton]{V = \{\sqrt{2}\}                                                      } \) \end{question}

\end{xmmulticols}
\end{exercise}


\begin{exercise}[\bf \ref{antw4.7}.]\setcounter{enumi}{7}   
\hypertarget{oef4.7}{Gegeven} is de veelterm
\[
A(x) = 3x^4-21x^2-7x^3+35x+30.
\]
Er is ook gegeven dat de getalwaarde van $A(x)$ in $\D x = -\frac{2}{3}$ gelijk is aan nul. Bepaal algebra\"isch alle nulwaarden van de veelterm $A(x)$.
\(\answer[onlineshowanswerbutton]{V = \left\{ 3, -\frac{2}{3}, \sqrt{5}, -\sqrt{5} \right\}}\)
\end{exercise}

\begin{exercise}[\bf \ref{antw4.8}.]\setcounter{enumi}{8}
	\hypertarget{oef4.8}{Beschouw} de veelterm
    \[
	P(x) = 3x^3-20x^2+kx+12
	\]
	waarbij $k \in \R$. Er gegeven dat $P(x)$ deelbaar is door $x-3$. Bepaal algebra\"isch alle nulwaarden van de veelterm $P(x)$.
		\(\answer[onlineshowanswerbutton]{V = \left\{ 3, 4, -\frac{1}{3}\right\}}\)
\end{exercise}
	


	
\end{document}