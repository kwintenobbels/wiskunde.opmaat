\documentclass{ximera} 

\input{../../preamble.tex}
\input{../../preamblekdn.tex}

\addPrintStyle{..}

\begin{document}
	\author{Koen De Naeghel}
	\xmtitle{Oefeningen reeks 2}{}
	\label{xim:veeltermen_deling_door_xa_oefeningen_reeks2}

%%\section*{Oefeningen reeks 2}

\begin{exercise}[\bf \ref{antw3.5}.]\setcounter{enumi}{5}
\hypertarget{oef3.5}{Gegeven} zijn de veeltermen
\[
A(x) = -x^4 - kx^2+\frac{3}{2} \quad \text{ en } \quad B(x) = x+2
\]
waarbij $k \in \R$. Bepaal de waarde(n) van de parameter $k$ zodat de rest bij deling van $A(x)$ door $B(x)$ gelijk is aan $\frac{4}{5}$. 
\( \answer[onlineshowanswerbutton]{k = - \frac{153}{40} } \) 
\end{exercise}

\begin{exercise}[\bf \ref{antw3.6}.]\setcounter{enumi}{6} 
\hypertarget{oef3.6}{{\bf (toelatingsexamen arts)}}\index{toelatingsexamen arts} 
De deling van de veelterm $P(x) = x^3 + mx^2 + mx + 4$ door $x-2$ en $x+2$ levert dezelfde rest op. Hoeveel is die rest?
\begin{xmmulticols}{4} 
\begin{enumerate}

\item 
$-16$
\item 
$-12$ 
\item 
$-8$ 
\item
$-4$
\end{enumerate}
\( \answer[onlineshowanswerbutton]{ \text{B}} \) 
\end{xmmulticols}
\end{exercise}

\begin{exercise}[\bf \ref{antw3.7}.]\setcounter{enumi}{7}  
\hypertarget{oef3.7}{Bepaal} telkens de exacte waarde(n) van de parameters zodat $A(x)$ deelbaar is door $B(x)$. 
	\begin{question} \( A(x) = 5x^2+2x-7                    \quad \text{ en } \quad B(x) = x-2a        \answer[onlineshowanswerbutton]{a \in \left\{\frac{1}{2}, -\frac{7}{10}\right\} } \) \end{question} 
	\begin{question} \( A(x) = (2-a)x^2 + 5ax + a^2         \quad \text{ en } \quad B(x) = x-3         \answer[onlineshowanswerbutton]{\text{een enkel re\"eel getal $a$ voldoet}      } \) \end{question} 
	\begin{question} \( A(x) = -2x^2 + a\sqrt{2}\,x + a-5   \quad \text{ en } \quad B(x) = x-\sqrt{2}  \answer[onlineshowanswerbutton]{a = \frac{5+2\sqrt{2}}{3}                       } \) \end{question} 
	\begin{question} \( A(x) = 2x^3 + ax^2 + (1-6a^2)x + 2a \quad \text{ en } \quad B(x) = x+2a        \answer[onlineshowanswerbutton]{a \in \R                                       } \) \end{question} 
	\begin{question} \( A(x) = ax^3 + 19x^2 + bx + 8        \quad \text{ en } \quad B(x) = (x-2)(x+4)  \answer[onlineshowanswerbutton]{a = 10$ en $b = -82                             } \) \end{question} 
	\begin{question} \( A(x) = x^4 + ax^3 - 9x^2 + 18x + b  \quad \text{ en } \quad B(x) = x(x-2)      \answer[onlineshowanswerbutton]{a = -2$ en $b = 0                               } \) \end{question} 
	\begin{question} \( A(x) = -2x^3 + bx - 2ax^2 + 3a      \quad \text{ en } \quad B(x) = x+a         \answer[onlineshowanswerbutton]{a = 0$ of $b = 3                                } \) \end{question} 

\end{exercise}


%%% \clearpage

\begin{exercise}[\bf \ref{antw3.8}.]\setcounter{enumi}{8} 
\hypertarget{oef3.8}{{\bf (toelatingsexamen arts)}}\index{toelatingsexamen arts}
De veelterm $P(x) = 8x^3 + 8$ is deelbaar door $x+a$, met $a \in \R$. Hoeveel is de rest van de deling van $P(x)$ door $x+2a$?
\begin{xmmulticols}{4} 
\begin{enumerate}

\item 
$-60$ 
\item
$-56$ 
\item
$-52$ 
\item 
$-50$ 
\end{enumerate}
\end{xmmulticols}
\( \answer[onlineshowanswerbutton]{\text{B}} \) 
\end{exercise}

\begin{exercise}[\bf \ref{antw3.9}.]\setcounter{enumi}{9} 
\hypertarget{oef3.9}{{\bf (toelatingsexamen arts)}}\index{toelatingsexamen arts}
Als de veelterm $P(x) = x^2 + ax + a$ deelbaar is door $x+b$, met $a$ en $b$ re\"ele getallen, dan geldt
\begin{xmmulticols}{2} 
\begin{enumerate}

\item 
$b \neq 0$ en $\D a = -\frac{b}{b-1}$
\item
$b \neq 1$ en $\D a = -\frac{b^2}{b-1}$
\item
$b \neq 1$ en $\D a = \frac{b^2}{b-1}$
\item
$b \neq 1$ en $\D a = \frac{b}{b-1}$
\end{enumerate}
\end{xmmulticols}
\( \answer[onlineshowanswerbutton]{\text{C}} \) 
\end{exercise}

\begin{exercise}[\bf \ref{antw3.10}.]\setcounter{enumi}{10}
\hypertarget{oef3.10}{Gegeven} is de veelterm
\[
A(x) = x^3 - ax^2 + bx - 12
\]
waarbij $a,b \in \R$. Bepaal de waarde(n) van de parameters $a$ en $b$ waarvoor $A(x)$ deelbaar is door $(x-1)(x-3)$. 

\(\answer[onlineshowanswerbutton]{$a = 8$ \text{en} $b = 19$} \)
\end{exercise}

\begin{exercise}[\bf \ref{antw3.11}.]\setcounter{enumi}{11} 
\hypertarget{oef3.11}{{\bf (toelatingsexamen arts)}}\index{toelatingsexamen arts}
We beschouwen de veelterm $A(x) = 2x^3 + px^2 + qx + r$. Deze veelterm is deelbaar door $x^2 - 1$ en de rest bij deling door $x-3$ is $8$. Geef de waarde van de uitdrukking $(p-r)\cdot q$.
\begin{xmmulticols}{4} 
\begin{enumerate}

\item 
$-20$
\item 
$-10$
\item 
$10$
\item 
$20$
\end{enumerate}
\end{xmmulticols}
\( \answer[onlineshowanswerbutton]{\text{D}} \) 

\end{exercise}

\begin{exercise}[\bf \ref{antw3.12}.]\setcounter{enumi}{12} 
\hypertarget{oef3.12}{Gegeven} is de veelterm
\[
A(x) = (b-c)x^2 + b^2(c-x) + c^2(x-b)
\]
waarbij $b,c \in \R$ met $b \neq c$. Toon aan dat $A(x)$ deelbaar is door $(x-b)(x-c)$ en schrijf $A(x)$ als een product. 
\(\answer[onlineshowanswerbutton]{A(x) = (b-c)(x-b)(x-c)}\)
\end{exercise}

\begin{exercise}[\bf \ref{antw3.13}.]\setcounter{enumi}{13}  
\hypertarget{oef3.13}{{\bf (toelatingsexamen arts)}}\index{toelatingsexamen arts}
We beschouwen de veelterm $F(x) = x^3 + px^2 - 8x + q$. Deze veelterm is deelbaar door $x-1$ en de rest bij deling door $x^2-9$ is $x-9$. Geef de waarde van $q$.
\begin{xmmulticols}{4} 
\begin{enumerate}

\item 
$-9$
\item 
$-2$
\item 
$9$
\item 
$16$
\end{enumerate}
\end{xmmulticols}
\( \answer[onlineshowanswerbutton]{\text{C}} \) 

\end{exercise}

\begin{exercise}[\bf \ref{antw3.14}.]\setcounter{enumi}{14}  
\hypertarget{oef3.14}{Bepaal} telkens het quoti\"ent en de rest bij deling van $A(x)$ door $B(x)$.

	\begin{question} $A(x) = 4x^3-6x+2$ \quad en \quad $B(x) = 2x-6$                                             \( \answer[onlineshowanswerbutton]{\text{quoti\"ent $2x^2+6x+15$ en rest $92$}                             } \) \end{question}
	\begin{question} $A(x) = 18x^3-10$ \quad en \quad $\D B(x) = x + \sqrt{6}$                                   \( \answer[onlineshowanswerbutton]{\text{quoti\"ent $18x^2 - 18\sqrt{6}\,x+108$ en rest $-10-108\sqrt{6}$} } \) \end{question}
	\begin{question} $A(x) = 2x^4 + 17x^3 - 68x$ \quad en \quad $\D B(x) = x+\frac{1}{2}$                        \( \answer[onlineshowanswerbutton]{\text{quoti\"ent $2x^3+16x^2-8x-64$ en rest $32$}                       } \) \end{question}
	\begin{question} $A(x) = \sqrt{2}\,x^2 + 3\sqrt{10}\,x - 20\sqrt{2}$ \quad en \quad $\D B(x) = x - \sqrt{5}$ \( \answer[onlineshowanswerbutton]{\text{quoti\"ent $\sqrt{2}\,x+4\sqrt{10}$ en rest $0$}                  } \) \end{question}
\end{exercise}

\begin{exercise}[\bf \ref{antw3.15}.]\setcounter{enumi}{15}  
\hypertarget{oef3.15}{Jeroen} deelt de veelterm $6x^4-7x^3+22x^2-24x-13$ door $2x-1$ en vindt als quoti\"ent $6x^3-4x^2+20x-14$ en als rest $-20$. 
\begin{enumerate}
\item[(a)]
Hoe kan Jeroen snel inzien dat hij een fout gemaakt heeft zonder de deling opnieuw uit te voeren? 
\item[(b)]
Bepaal het juiste quoti\"ent en de juiste rest.
\item[(c)]
Hoe kun je zeker weten dat jouw quoti\"ent en rest correct is? Voer dit uit.
\end{enumerate}
\( \answer[onlineshowanswerbutton]{\text{b: quoti\"ent $3x^3-2x^2+10x-7$ en rest $-20$ }}\)
\end{exercise}






\end{document}