%%%%%%%%%%%%%%%%%%%%%%%%%%%%%%%%%%%%%%%%%%%%%%%%%%%%%%%%%%%%%%%%%%%%%%%%%%%%%%%%%%%%%%%%%%

% DIT MATERIAAL VERTROK VAN DE OPEN-SOURCE CURSUS VEELTERMEN VAN KOEN DE NAEGHEL         
% GEKOPIEËRD OP 24 MAART 2025                                                            
% ORIGINEEL BESCHIKBAAR VIA https://www.koendenaeghel.be/opensource.htm         

%%%%%%%%%%%%%%%%%%%%%%%%%%%%%%%%%%%%%%%%%%%%%%%%%%%%%%%%%%%%%%%%%%%%%%%%%%%%%%%%%%%%%%%%%%

% TODO HIER KUNNEN NOG KLEINE OEFENINGEN INGEVOEGD WORDEN. 




\documentclass{ximera}
\input{../preamble}
\input{../preamblekdn}

\addPrintStyle{..}
\begin{document}
	\author{Koen de Naeghel - Wiskunde Op Maat}
	\xmtitle{Eentermen}{}
    \xmsource

	
% VOOR DEZE INLEIDINGEN HEBBEN WE NOG EEN OPLOSSING IN XIMERA NODIG; EIGEN ACTIVITY IS OVERKILL (ENVIRIONMENT?)
% In de voorbije jaren heb je leren rekenen met lettervormen, waaronder het optellen, ver\-menigvuldigen en vereenvoudigen van eentermen, tweetermen en drietermen. Dat zullen we in dit hoofdstuk uitbreiden tot viertermen, vijftermen enzovoort. Algemeen spreken we dan van \textit{ veeltermen}.


De bewerkingen

\[
2\cdot 3^5
\qquad 2\cdot 8^5
\qquad 2\cdot (-1)^5
\qquad 2\cdot\left(\frac{19}{3}\right)^5
\qquad 2\cdot\left(\sqrt{3}\right)^5 
\qquad \text{ en } \qquad 2 \cdot \pi^5
\]


hebben iets gemeen: ze zijn allemaal van de vorm $2\cdot \square^5$ waarbij $\square$ een reëel getal voorstelt. Het is gebruikelijk om in plaats van het symbool $\square$ een Latijnse letter $x$ als \textit{ variabele} te gebruiken. Uitdrukkingen van deze vorm noemt men \textit{ eentermen in $x$}. Het symbool $\cdot$ voor de vermenigvuldiging wordt meestal weggelaten. 

Het is gebruikelijk om een éénterm te noteren met een Latijnse hoofdletter $A$ of $B$. Om te beklemtonen dat de variabele van de eenterm gelijk aan $x$  is wordt de schrijfwijze $A(x)$ of $B(x)$ veel gebruikt. De eenterm hierboven kan dus geschreven worden als $A(x) = 2x^5$ en dit veralgemenen leidt tot volgende definitie:

% IN DE REST VAN DE CURSUS WORDT DE TERM ARGUMENT NIET GEBRUIKT --> WE KIEZEN OM HET WEG TE LATEN. 


\begin{definition} 
Een \textbf{(reële) eenterm in (de variabele) $x$}\index{eenterm} is een uitdrukking $A(x) = ax^n$ waarbij $a \in \R$ en $n \in \N$. 

De uitdrukking $ax^n$ is de \textit{ vermenigvuldiging} van een reëel getal $a$ met $x^n$. Om die vermenigvuldiging te benadrukken, noteert men $ax^n$  ook als $a \cdot x^n$. De uitdrukking $x^n$ is een macht met als grondtal de variabele $x$ en als exponent $n$. Volgende schrijfwijzen gelden wegens afsraak:
\[
1\cdot x^n = x^n, \qquad
0\cdot x^n = 0, \qquad  
x^0 = 1, \qquad 
a\cdot x^0 = a \qquad \text{ en } \qquad 
a\cdot x^1 = a \cdot x = ax.
\]

Het getal $a$ is de \textbf{coëfficiënt} van de eenterm $ax^n$ .

Als $ax^n$ een eenterm is waarbij $a \neq 0$, dan is $n$ de \textbf{graad} van de eenterm. De graad van een éénterm \(A(x)\) wordt genoteeerd als $\gr A(x)$. De graad van de eenterm $0\cdot x^n = 0$ wordt in het secundair onderwijs niet gedefiniëerd.


Twee eentermen in $x$ zijn \textbf{gelijksoortig} als ze ofwel dezelfde graad hebben, ofwel beide gelijk zijn aan nul.

Twee eentermen in $x$ zijn \textbf{gelijk} als ze ofwel dezelfde graad en dezelfde coëfficiënt hebben, ofwel beide gelijk zijn aan nul.


De \textbf{getalwaarde} van een eenterm $A(x)$ wordt bekomen door de variabele $x$ te vervangen met een reëel getal $r$. Die getalwaarde wordt genoteerd met $A(r)$. Als $A(x) = ax^n$ dan is $A(r) = ar^n$. In symbolen:
\[
A(x) = ax^n \quad \Rightarrow \quad A(r) = a r^n.
\]
Hierbij is $a,r \in \R$ en $n \in \N$, waarbij het geval $r = n = 0$ uitgesloten wordt. De uitdrukking $0^0$ heeft geen betekenis.


\end{definition} 

% %%%%%%%%%%%%%%%%%%%%%%%%%%%%%%%%%%%%%%%%%%%%%%%%%%%%%%%%%%%%%%%%%%%%%%%%%%%%%%%%%%%%%%%%%%%%%%%%
% % HIER ALLES DAT WE METEEN BIJVOEGEN IN DE DEFINITIE; DAARNA IN COMMENTAAR 

% De uitdrukking $ax^n$ van een eenterm wordt opgevat als de \textit{ vermenigvuldiging} van een reëel getal $a$ met $x^n$. Om die vermenigvuldiging te benadrukken, noteren we $ax^n$ soms ook als $a \cdot x^n$. Verder vatten we de uitdrukking $x^n$ op als macht met als grondtal de variabele $x$ en als exponent $n$. Bij afspraak maken we dan ook gebruik van de volgende schrijfwijzen:
% \[
% 1\cdot x^n = x^n, \qquad
% 0\cdot x^n = 0, \qquad  
% x^0 = 1, \qquad 
% a\cdot x^0 = a \qquad \text{ en } \qquad 
% a\cdot x^1 = a \cdot x = ax.
% \]

% Als $ax^n$ een eenterm is, dan noemen we het getal $a$ de \underline{coëfficiënt}\index{coëfficiënt}\index{eenterm!coëfficiënt} van de eenterm.


% Als $ax^n$ een eenterm is waarbij $a \neq 0$, dan noemen we $n$ de \underline{graad}\index{graad}\index{eenterm!graad} van de eenterm. Schrijven we de eenterm als $A(x)$, dan noteren we de graad als $\gr A(x)$. De graad van de eenterm $0\cdot x^n = 0$ wordt in het secundair onderwijs niet gedefiniëerd.


% Twee eentermen in $x$ zijn \underline{gelijksoortig}\index{gelijksoortig} als ze ofwel dezelfde graad hebben, ofwel beide gelijk zijn aan nul.




% Twee eentermen in $x$ zijn \underline{gelijk}\index{eenterm!gelijk} als ze ofwel dezelfde graad en dezelfde coëfficiënt hebben, ofwel beide gelijk zijn aan nul.


% Vervangen we in een eenterm $A(x)$ de variabele $x$ door een reëel getal $r$, dan verkrijgen we de \underline{getalwaarde}\index{eenterm!getalwaarde} van $A(x)$ in $x = r$. Die getalwaarde noteren we met $A(r)$. Samengevat: als $A(x) = ax^n$ dan is $A(r) = ar^n$. Met behulp van het symbool $\Rightarrow$ voor implicatie wordt dit in symbolen:
% \[
% A(x) = ax^n \quad \Rightarrow \quad A(r) = a r^n.
% \]
% Hierbij is $a,r \in \R$ en $n \in \N$, waarbij we het geval $r = n = 0$ uitsluiten: aan de uitdrukking $0^0$ geven we geen betekenis.

%%%%%%%%%%%%%%%%%%%%%%%%%%%%%%%%%%%%%%%%%%%%%%%%%%%%%%%%%%%%%%%%%%%%%%%%%%%%%%%%%%%%%%%%%%%%%%%%



\begin{expandable}{remark}
In hogere wiskunde stelt men $\gr 0 = - \infty$ omdat op deze manier rekenregels in verband met de graad van een eenterm geldig blijven, bijvoorbeeld 
\textit{ de graad van het product van twee reële eentermen is gelijk aan de som van de graden van die eentermen}. Daarvoor verwijzen we naar Oefening \ref{oefgraadnulveelterm} op het einde van dit hoofdstuk.
\end{expandable}


% dit moet afspraak of conventie worden 
\begin{notation}

	In het vervolg van deze cursus geldt -tenzij anders gespecifeerd- dat voor \textit{ een eenterm} $A(x) = ax^n$ of $B(x) = bx^m$ telkens $n,m\in \N$ en $a,b\in \R$.
\end{notation}
	
\begin{example} 
De volgende uitdrukkingen zijn eentermen in $x$:
	
\[
A(x) = 2x^5, 
\quad B(x) = -\frac{3}{7}\,x^1, 
\quad C(x) = 541x^0 \quad \text{ en } 
\quad D(x) = 0x^{2024}.
\]
Ook $A(t) = 2t^5$ is een eenterm, maar dan in de variabele $t$. 
\end{example}

\begin{example}
De volgende uitdrukkingen zijn \textit{geen} eentermen in $x$:
\[
2x^5+3x, 
\quad \frac{1}{x}, 
\quad 3x^{-5}, 
\quad \sqrt{x} 
\quad \text{en} \quad \left|x\right|.
\] 
\end{example} 



Je kan telkens eenvoudig de coëeficiënt van een éénterm bepalen. Een éénterm die te schrijven is zonder variabele behoort tot de verzameling van reële getallen \(\R\). 

% \begin{enumerate}[(a)]  dit heeft weinig meerwaarde --> ITEMIZE VAN MAKEN 


\begin{example} 
% We vereenvoudigen de onderstaande eentermen, geven telkens de coëeficiënt en vermelden ook of de eenterm al dan niet behoort tot de verzameling van de reële getallen.
\begin{itemize}
\item  $A(x) = \sqrt{2}\cdot x \cdot x \cdot x = \sqrt{2}\cdot x^3$  met coëfficiënt $\sqrt{2}$, en $A(x) \not\in \R$
\item  $B(x) = 541x^0 = 541$                                         met coëfficiënt $541$, en $B(x) \in \R$ 
\item  $\D C(x) = -\frac{4}{7}\,x^1 = -\frac{4}{7}\,x$               met coëfficiënt $\D -\frac{4}{7}$, en $C(x) \not\in \R$
\item  $D(x) = 0 x^{2024} = 0$                                       met coëfficiënt $0$, en $D(x) \in \R$ 
\end{itemize}
\end{example} 


De graad van een éénterm kan -eventueel na een korte berekening- rechtstreeks afgelezen worden. 

\begin{example} 
Beschouw de volgende eentermen.
\[
A(x) = -7x^3, \,\,\, B(x) = -\frac{17}{89}\,x^5\cdot x^3, \,\,\, C(x) = 1, \,\,\, D(x) = \sqrt{5}\,x, \,\,\, P(x) = 0 x^4, \,\,\,  Q(x) = 0 x^7.
\]
De graad van de eenterm $A(x)$ is gelijk aan $3$, in symbolen: $\gr A(x) = 3$. 
Voor de andere eentermen geldt:  
\[
B(x) = -\frac{17}{89}\,x^8, \qquad C(x) = 1\cdot x^0, \qquad D(x) = \sqrt{5}\,x^1, \qquad P(x) = 0 = Q(x)
\]
hieruit volgt dat
\[
\gr B(x) = 8, \qquad \gr C(x) = 0, \qquad \gr D(x) = 1. 
\]
Omdat $P(x) = 0$ bestaat de graad van de eenterm $P(x)$ niet en bijgevolg is $\gr P(x) = /$. Analoog is ook $\gr Q(x) = /$.
\end{example} 


Met de graad van een eenterm bepaal kan je controleren welke eentermen gelijksoortig zijn: 
\begin{example} 
Gegeven zijn de eentermen
\[
A(x) = -\frac{1}{3}\,x^9, \quad B(x) = 5\,x^9, \quad C(x) = 0x^9, \quad D(x) = 2^9, \quad P(x) = \pi \quad \text{ en } \quad Q(x) = 0 x^7.
\]
Dan zijn $A(x)$ en $B(x)$ gelijksoortig. Omdat $C(x) = 0 = Q(x)$ zijn ook $C(x)$ en $Q(x)$ gelijksoortig. 
Ook $D(x)$ en $P(x)$ zijn gelijksoortig want $D(x) = 2^9 \cdot x^0$ en $P(x) = \pi \cdot x^0$.  
\end{example} 

De gelijkheid van eentermen wordt bepaald door de coëeficiënten én de graden: 
\begin{example} 
Wegens de coëeficiënt gelijk aan nul geldt de gelijkheid $0x^{2024} = 0x^{2025}$. Stel $\sqrt{2}\,x^n = bx^{35}$ dan is $n = 35$ en $b = \sqrt{2}$.
\end{example} 


Een getal \(r\) invullen voor de variable \(x\) levert na een rechtstreekse berekening de getalwaarde $A(r)$. 
\begin{example} 
Als $A(x) = 5x^3$ dan is de getalwaarde van $A(x)$ in $x = -2$ gelijk aan 
\[
A(-2) = 5 \cdot (-2)^3 = 5 \cdot (-8) = - 5 \cdot 8 = -40.
\]
\end{example} 

% In deze url staat nog een keer teveel rekenvaardigheden eigenlijk 


De basisbewerken met eentermen volgens rechtstreeks uit rekenregels voor machten en de distributiviteitseigenschap. Het is belangrijk deze \href{https://wiskunde.opmaat.org/wiskundeplan/rekenvaardigheden/rekenvaardigheden/rekenvaardigheden_inleiding}{rekenvaardigheden} goed onder de knie te hebben!  


\begin{proposition}
	
De som of het verschil van twee gelijksoortige eentermen in $x$ is opnieuw een eenterm in $x$, met als rekenregels:
\[
ax^n + bx^n = (a+b)x^n \quad \text{ en } \quad ax^n - bx^n = (a-b)x^n
\]
of kortweg: $ax^n \pm bx^n = (a\pm b)x^n$. 

Ook het product van twee eentermen in $x$ is opnieuw een eenterm in $x$, waarbij:
\[
ax^n \cdot bx^m = abx^{n+m}.
\]
Na het toepassen van deze rekenregels kan een som $a+b$, een verschil $a-b$ of een product $ab$ soms verder uitgewerkt en nadien vereenvoudigd worden.

\end{proposition}


\begin{example} 
Als $\D A(x) = \frac{3}{4}\,x^5$ en $\D B(x) = -\frac{7}{6}\,x^5$ dan is 
\begin{align*}
A(x) + B(x) & = \frac{3}{4}\,x^5 + \left(-\frac{7}{6}\,x^5\right) 
= \frac{3}{4}\,x^5 - \frac{7}{6}\,x^5
= \left(\frac{3}{4} - \frac{7}{6}\right)x^5 
= \frac{3 \cdot 3 - 7 \cdot 2}{12}\,x^5 
= -\frac{5}{12}\,x^5, \\
A(x) \cdot B(x) & = \frac{3}{4}\,x^5 \cdot \left(-\frac{7}{6}\,x^5\right) = - \frac{3 \cdot 7}{4 \cdot 6}\,x^{5+5} = -\frac{1\cdot 7}{4 \cdot 2}\, x^{10} = -\frac{7}{8}\,x^{10}. 
\end{align*}
\end{example}




\end{document}
