\documentclass{ximera} 
\input{../preamble}
\input{../preamblekdn}
\addPrintStyle{..}

\begin{document}
	\author{Koen De Naeghel - Wiskunde Op Maat}
	\xmtitle{Oefeningen reeks 2}{}
\xmsource
	\label{xim:veeltermen_basisbegrippen_oefeningen_reeks2}

%%\section*{Oefeningen reeks 2}

\begin{exercise}
Bepaal telkens de graad, de hoogstegraadscoëfficiënt en de constante term van de veelterm zonder deze veelterm uit te werken of te vereenvoudigen. 
	
	
		\begin{question} \( (x^2-3)(x^3+5)                                                              \) heeft graad \( \answer{5}\), hoogstegraadscoëfficiënt \( \answer{1           } \), constante term \( \answer{-15          } \)  \end{question}
		\begin{question} \( (2x+3)(x^5+x^2+1)                                                           \) heeft graad \( \answer{6}\), hoogstegraadscoëfficiënt \( \answer{2           } \), constante term \( \answer{3            } \)  \end{question}
		\begin{question} \( \D \left(\frac{2}{9}\,x^3+x^2-1\right)\left(3x^4+3x+\frac{3}{7}\right)      \) heeft graad \( \answer{7}\), hoogstegraadscoëfficiënt \( \answer{\frac{2}{3} } \), constante term \( \answer{-\frac{3}{7} } \)  \end{question}
		\begin{question} \( (-x^3+5)(3x^2-7)(5x+2)                                                      \) heeft graad \( \answer{6}\), hoogstegraadscoëfficiënt \( \answer{-15         } \), constante term \( \answer{-70          } \)  \end{question}
		\begin{question} \( (5x-3)^3+27                                                                 \) heeft graad \( \answer{3}\), hoogstegraadscoëfficiënt \( \answer{125         } \), constante term \( \answer{0            } \)  \end{question}
		\begin{question} \( (4x^2+5)^3-x^5-x^4-x^3-x^2-x                                                \) heeft graad \( \answer{6}\), hoogstegraadscoëfficiënt \( \answer{64          } \), constante term \( \answer{125          } \)  \end{question}
	
	\end{exercise}

		

	\pagebreak

	%  van deze oefeningn zijn nog geen eindoplossingen 
	
	\begin{exercise} % oef 6
	Werk de volgende veeltermen uit, en vereenvoudig telkens zoveel mogelijk. Geef daarna telkens de graad, de constante term en de hoogstegraadscoëfficiënt.
	\begin{xmmulticols}{2}
	
	
		\begin{question} \( x(x^3-3x+\sqrt{2})                                                          =\answer[onlineshowanswerbutton]{  } \) met graad \( \answer{} \), hoogstegraadscoëfficiënt \( \answer{} \), constante term \(\answer{} \) \end{question}
		\begin{question} \( x+2x^3-1-2(x^7+8x^8+x^3)                                                    =\answer[onlineshowanswerbutton]{  } \) met graad \( \answer{} \), hoogstegraadscoëfficiënt \( \answer{} \), constante term \(\answer{} \) \end{question}
		\begin{question} \( (x+1)(x-3)                                                                  =\answer[onlineshowanswerbutton]{  } \) met graad \( \answer{} \), hoogstegraadscoëfficiënt \( \answer{} \), constante term \(\answer{} \) \end{question}
		\begin{question} \( (2x-3)(-6x-1)                                                               =\answer[onlineshowanswerbutton]{  } \) met graad \( \answer{} \), hoogstegraadscoëfficiënt \( \answer{} \), constante term \(\answer{} \) \end{question}
		\begin{question} \( (x^2+1)(x^3-x+3)                                                            =\answer[onlineshowanswerbutton]{  } \) met graad \( \answer{} \), hoogstegraadscoëfficiënt \( \answer{} \), constante term \(\answer{} \) \end{question}
		\begin{question} \( (3x^2-6+2x)(x+2)^2                                                          =\answer[onlineshowanswerbutton]{  } \) met graad \( \answer{} \), hoogstegraadscoëfficiënt \( \answer{} \), constante term \(\answer{} \) \end{question}
		\begin{question} \( -(\sqrt{3}+4x)(-\sqrt{3}+4x)+3x^2-1                                         =\answer[onlineshowanswerbutton]{  } \) met graad \( \answer{} \), hoogstegraadscoëfficiënt \( \answer{} \), constante term \(\answer{} \) \end{question}
		\begin{question} \( \D \Bigl(-x^6 + \frac{1}{3}x\Bigr)(2x-1) - 3x^2\Bigl(-x^5-\frac{1}{9}\Bigr) =\answer[onlineshowanswerbutton]{  } \) met graad \( \answer{} \), hoogstegraadscoëfficiënt \( \answer{} \), constante term \(\answer{} \) \end{question}
	
	\end{xmmulticols} 
	\end{exercise}


	\begin{exercise}
		
			\def\myopts[#1]{%
		\def\isO{}%
		\def\isI{}%
		\def\isII{}%
		\def\isIII{}%
		\def\isIV{}%
		\def\isV{}%
		\def\isVI{}%
		%
		\expandafter\def\csname is#1\endcsname{correct}%
		\quad%
		\wordChoice{%
			\choice[\isO]  {\(0\)}%
			\choice[\isI]  {\(1\)}%
			\choice[\isII] {\(2\)}%
			\choice[\isIII]{\(3\)}%
			\choice[\isIV] {\(4\)}%
			\choice[\isV]  {\(5\)}%
			\choice[\isVI] {\(6\)}%
		}%
		}
			
     Werk telkens uit, vereenvoudig en bepaal de graad. 
	\begin{xmmulticols}{2}
	
	
		\begin{question} \( 2x-(x^3-x^2-x-1)                                         =\answer[onlineshowanswerbutton]{-x^3+x^2+3x+1                                         } \) met graad \myopts[III] \end{question}
		\begin{question} \( x(x^2-3x+2)-2x(x-1)                                      =\answer[onlineshowanswerbutton]{x^3-5x^2+4x                                           } \) met graad \myopts[III] \end{question}
		\begin{question} \( \left(\sqrt{2}\,x+\sqrt{3}\right)\sqrt{5}                =\answer[onlineshowanswerbutton]{\sqrt{10}\,x+\sqrt{15}                                } \) met graad \myopts[I]   \end{question}
		\begin{question} \( -3(3x^3-x+2)(x-5)                                        =\answer[onlineshowanswerbutton]{-9x^4+45x^3+3x^2-21x+30                               } \) met graad \myopts[IV]  \end{question}
		\begin{question} \( \D \frac{x^3-(x^2+1)(x+4)}{5}                            =\answer[onlineshowanswerbutton]{-\frac{4}{5}\,x^2-\frac{1}{5}\,x-\frac{4}{5}          } \) met graad \myopts[II]  \end{question}
		\begin{question} \( (x+2)(x-2)-(x-2)^2                                       =\answer[onlineshowanswerbutton]{4x-8                                                  } \) met graad \myopts[I]   \end{question}
		\begin{question} \( 17-4x^2-(-2x+3)(-2x-3)-8                                 =\answer[onlineshowanswerbutton]{-8x^2+18                                              } \) met graad \myopts[II]  \end{question}
		\begin{question} \( \D \left(\frac{5}{6}\,x+\frac{1}{3}\right)(3x-2)(3x+2)   =\answer[onlineshowanswerbutton]{\frac{15}{2}\,x^3 + 3x^2 - \frac{10}{3}\,x-\frac{4}{3}} \) met graad \myopts[III] \end{question}
	
	\end{xmmulticols}
	\end{exercise}


\begin{exercise}
	\renewcommand{\TJa }{\makebox[2.5cm]{nulwaarde}}
	\renewcommand{\TNee}{\makebox[2.5cm]{geen nulwaarde}}
		Bepaal telkens de getalwaarde van de veelterm in de gegeven \(x\)-waarde. Gebruik de correcte notatie. Geef aan of die \(x\)-waarde ook een nulwaarde is. 
		
		\begin{question} \( A(x) = x^3-3x^2-10x+24       \quad \text{ in }  x = 2               \)  \begin{uitkomst} \( A(2)=0                       \) \end{uitkomst} \end{question}
		\begin{question} \( B(x) = x^4-7x^2-6x+1         \quad \text{ in }  x = -1              \)  \begin{uitkomst} \( B(-1)=1                      \) \end{uitkomst} \end{question}
		\begin{question} \( P(x) = x^4-4x^3-4x^2-4x-5    \quad \text{ in }  x = 5               \)  \begin{uitkomst} \( P(5)=0                       \) \end{uitkomst} \end{question}
		\begin{question} \( S(x) = x^3-x^2+x-1           \quad \text{ in }  x = 0               \)  \begin{uitkomst} \( S(0)=-1                      \) \end{uitkomst} \end{question}
		\begin{question} \( C(x) = x^4-4                 \quad \text{ in }  x = \sqrt{2}        \)  \begin{uitkomst} \( C(\sqrt{2})=0                \) \end{uitkomst} \end{question}
		\begin{question} \( D(x) = 2x^3 + 3x^2 - 11x - 6 \quad \text{ in }  \D x = -\frac{1}{2} \)  \begin{uitkomst} \( D\left(-\frac{1}{2}\right)=0 \) \end{uitkomst} \end{question}
\end{exercise}




\begin{exercise}
Bepaal telkens de waarde(n) van de reële parameters waarvoor de veeltermen \(A(x)\) en \(B(x)\) gelijk zijn.


	\begin{question} \( A(x) = (ax+7)x     \) en \(B(x) = 7x-3x^2    \) zijn gelijk voor \( a = \answer[onlineshowanswerbutton]{-3} \)                                                 \end{question} 
	\begin{question} \( A(x) = (ax+b)(3x-2)\) en \(B(x) = 3x^2+6-11x \) zijn gelijk voor \( a = \answer[onlineshowanswerbutton]{1} \) en \( b = \answer[onlineshowanswerbutton]{-3} \) \end{question} 

\end{exercise}

\begin{exercise}
Bepaal telkens de exacte waarde(n) van de reële parameters waarvoor de veelterm voldoet aan de voorwaarde. 


	\begin{question} \( A(x) = -x^3-8ax-3               \text{ met           }  A(1) = 8 \) is geldig voor \( \answer[onlineshowanswerbutton]{a = \frac{-3}{2}} \)                                   \end{question}
	\begin{question} \( G(x) = (c+1)x^2+3x-\sqrt{2}\,c  \text{ met nulwaarde } -1        \) is geldig voor \( \answer[onlineshowanswerbutton]{a = 1} \) en \(\answer[onlineshowanswerbutton]{b=-3}\) \end{question}

\end{exercise}

\begin{exercise}
Gegeven zijn de veeltermen 
\[
A(x) = -3x^4+2x^6-5x^6+6x^4+21x^4-5+12x^2+5 \quad \text{ en } \quad B(x) = -5x^2+2x-6-2x+6.
\]

	\begin{question} Vereenvoudig de veeltermen  \(A(x)\) en \(B(x)\). \( A(x) = \answer[onlineshowanswerbutton]{ -3x^6+24x^4+12x^2} \) en \( B(x) = \answer[onlineshowanswerbutton]{-5x^2} \) \end{question}
	\begin{question} Geef de graad van de veeltermen  \(A(x)\) en \(B(x)\). Hanteer de correcte notatie. \( \gr A(x) = \answer{6} \) en \( \gr B(x) = \answer{2} \) \end{question}
	\begin{question} Bepaal \(A(-2)\) en \(B(\sqrt{3})\).     \(A(-2) = \answer[onlineshowanswerbutton]{240} \) en  \(B(\sqrt{3}) = \answer[onlineshowanswerbutton]{-15}\) \end{question}
	\begin{question} Werk uit, vereenvoudig en bepaal de graad van de som en het product van de veeltermen \(A(x)\) en \(B(x)\).   \( = \answer[onlineshowanswerbutton]{  } \) \end{question}

% van d is er nog geen antwoord 

\end{exercise}

\end{document}