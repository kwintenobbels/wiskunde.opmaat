\documentclass{ximera} 

\input{../preamble}
\input{../preamblekdn}
\addPrintStyle{..}

\begin{document}
	\author{Koen De Naeghel - Wiskunde Op Maat}
	\xmtitle{Oefeningen reeks 3}{}
    \xmsource
	\label{xim:veeltermen_deling_door_xa_oefeningen_reeks3}
%%\section*{Oefeningen reeks 3}

\begin{exercise}
De rest bij deling van een veelterm door \(x-1\) is \(6\). Delen we de veelterm door \(x-2\) dan verkrijgen we als rest \(18\). Bepaal de rest bij deling van de veelterm door \(x^2-3x+2\). %\((x-1)(x-2)\). 
\( \answer[onlineshowanswerbutton]{12x-6}\)
\end{exercise}


\begin{Uitbreiding}
\begin{exercise}
{\bf (veralgemeende reststelling)}\index{reststelling!veralgemeende}\index{veralgemeende reststelling}
Zij \(A(x)\) een veelterm en \(a,b \in \R\) met \(b \neq 0\). Toon aan dat de rest bij deling van \(A(x)\) door \(bx-a\) gelijk is aan \(A\left(\frac{a}{b}\right)\).
\end{exercise}

\begin{exercise}
{\bf (kenmerk van deelbaarheid door \(bx-a\))}\index{kenmerk van deelbaarheid!door \(bx-a\)}
Zij \(A(x)\) een veelterm en \(a,b \in \R\) met \(b \neq 0\). Bewijs:
\[
(bx-a) \mid A(x) \quad \Leftrightarrow \quad A\left(\frac{a}{b}\right) = 0
\]
\end{exercise}

\begin{exercise}
{\bf (kenmerken van deelbaarheid door \(x-1\) en \(x+1\))}\index{kenmerk van deelbaarheid!door \(x-1\)}\index{kenmerk van deelbaarheid!door \(x+1\)}
Bewijs de volgende uitspraken.
\begin{enumerate}

\item
Een veelterm is deelbaar door \(x-1\) als en slechts als de som van de coëfficiënten gelijk is aan nul.
\item
Een veelterm is deelbaar door \(x+1\) als en slechts als de som van de coëfficiënten van de termen van even graad gelijk is aan de som van de coëfficiënten van de termen van oneven graad.
\end{enumerate}
\end{exercise}
\end{Uitbreiding}

\begin{exercise}
Gegeven is de veelterm
\[
A(x) = 48x^6 + 491 x^5 - 364x^4 + 456x^3 - 164x^2 + 272x + 309.
\]
Bepaal algebraïsch en zonder gebruik van te maken van ICT of rekenmachine de getalwaarde van \(A(x)\) in \(x = -11\). 
\(\answer[onlineshowanswerbutton]{100}\)
\end{exercise}

%%% \clearpage


\end{document}