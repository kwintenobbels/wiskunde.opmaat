\documentclass{ximera}

%\addPrintStyle{..}

\begin{document}
	\author{Wiskunde Op Maat}
	\xmtitle{oefening}{}

\renewcommand{\TJa }{\makebox[2.5cm]{positief }}
\renewcommand{\TNee}{\makebox[2.5cm]{negatief}}

% \choiceYes \choiceNO 

\begin{exercise}
    Zoek een gehele oplossing voor volgende vergelijking \textbf{zonder} de haakjes uit te werken...  

    \[(x+3)(x+5)(x+7)(x+9) = 9\]

    \begin{hint}
        Het is nooit een slecht idee om eens \(x = 0\) in te vullen \ldots 
    \end{hint}
    
    Ik zoek een \choiceNo getal. 
    
    \begin{feedback}
    Aangezien \( 3 \cdot 5 \cdot 7 \cdot 9 > 9 \) en het een product van positieve factoren is voor \(x > 0\) kan er geen positieve oplossing bestaan! 
    \end{feedback}

    \begin{hint}
    Het is een gelijkheid met links een product en rechts een getal \ldots Kan je daar een gelijkheid van producten van maken? 
    \end{hint}
    
    \begin{hint}
    Elk natuurlijk getal (en dus ook \(9\) \ldots) kan geschreven worden als \textbf{product van priemfactoren}! 
    \end{hint}

    \begin{oplossing}
    De delers van \(9\) zijn \(\pm 1, \pm 3 \text{ en } \pm 9\). 
    Omdat we een gehele oplossing zoeken is elke factor aan de linkerkant gelijk aan een deler. 
    Er werd reeds opgemerkt dat de oplossing negatief moet zijn. 
    Voor de eerste factor \(x+3\) geeft dit als mogelijke waarden voor \(x\): 
    \begin{itemize}
        \item \(x = -2\)
        \item \(x = -6\)
        \item \(x = -12\)
    \end{itemize}
        
    Ook de factor \(x-7\) moet gelijk zijn aan een deler, dit is enkel het geval voor \(x=-6\). Invullen levert  \((-3)\cdot(-1)\cdot(1)\cdot(3) = 9. \)

    \end{oplossing}
\end{exercise}



\end{document}
