\documentclass{ximera}


\begin{document}
	\author{Wiskunde Op Maat}
	\xmtitle{Kwadratische vergelijkingen oplossen}{}

Met deze pagina kan je twee basismethodes om deelbaarheid van veeltermen te analyseren herhalen. 
De staartdeling voor veeltermen geeft een basismethode die altijd toepasbaar is. (stelling van de Euclidische deling)
Om te delen door veeltermen van de vorm \(x - a\) wordt de reststelling en Horner gebruikt. 
Onder 'download' kan je de pdf met of zonder antwoorden downloaden.

\newcommand{\choicetwee}{{\wordChoice{\choice[correct]{twee oplossingen}\choice{één oplossing}\choice{geen oplossingen}}}}
\newcommand{\choiceeen}{{\wordChoice{\choice{twee oplossingen}\choice[correct]{één oplossing}\choice{geen oplossingen}}}}
\newcommand{\choicegeen}{{\wordChoice{\choice{twee oplossingen}\choice{één oplossing}\choice[correct]{geen oplossingen}}}}

\newcommand{\choicepositief}{{\wordChoice{\choice[correct]{positief}\choice{nul}\choice{negatief}}}}
\newcommand{\choicenul}{{\wordChoice{\choice{positief}\choice[correct]{nul}\choice{negatief}}}}
\newcommand{\choicenegatief}{{\wordChoice{\choice{positief}\choice{nul}\choice[correct]{negatief}}}}
 

% Voorbeeld uit schema van de staartdeling 
\begin{exercise}

Beschouw de veeltermen \(A(x) = 6x^3 + 19x^2 + 8x - 5\) en \(B(x) = 2x+5\). Ga na of \(A(x)\) deelbaar is door \(B(x)\). 


\begin{question}
Geef eerst een voorbeeld van de Euclidische deling voor natuurlijke getallen. 
    \begin{hint}
        Hoe kan je \(10 \) delen door \(3\)? 
    \end{hint}

    \begin{oplossing}

    Zeggen dat de deling van \(13\) door \(4\) gelijk is aan \(3\) met rest \(1\), betekent dat
    \[
    13 = 3 \cdot 4  + 1
    \]
    omdat \(4\) hoogstens \(3\) keer in \(13\) kan, en er blijft dan nog \(1\) over.
     
    Op dezelfde manier betekent \(20\) delen door \(3\) dat
    \[
    20 = 6\cdot 3 + 2
    \]
    of met symbolen: als we \textit{deeltal} \(D\) delen door \textit{deler} \(d\), krijgen we \textit{quotiënt} \(q\) en \textit{rest} \(r\) als volgende formule geldt:
    \[
    D = q\cdot d + r   \text{ met } r<d.
    \]
     
     
    Daarbij is het erg belangrijk dat \(r<d\): als dat niet het geval is, heb je nog niet ver genoeg gedeeld, want je kan dan de rest \(r\) nog verder delen door \(d\).  

    Indien de rest gelijk is aan nul spreken we van \textit{deelbaarheid}. Zo is \(6\) deelbaar door \(3\) en 15 deelbaar door \(5\). 
    
    \end{oplossing}

\end{question}


\begin{question}
    Leg in je eigen woorden de Euclidische deling uit voor een veelterm \(A(x)\) door een veelterm \(B(x)\). Wanneer zeggen we dat de veelterm \(A(x)\) deelbaar is door \(B(x)\)? 

    \begin{oplossing}
        
    % \begin{theorem}(deling met rest, euclidische deling)
    % \begin{theorem}
    % Zij \(A(x)\) en \(B(x)\) twee veeltermen met \(B(x) \neq 0\). Dan bestaat er precies één veelterm \(Q(x)\) en precies één veelterm \(R(x)\) zodat
    
    % % \begin{equation}
    % \[
    % A(x) = B(x)\cdot Q(x) + R(x) \quad \text{ waarbij } \quad \gr R(x) < \gr B(x) \,\, \text{ of } \,\, R(x) = 0.
    % \]
    % % \end{equation}

    % met \(A(x)\) het \textbf{deeltal}, \(B(x)\) de \textbf{deler}, \(Q(x)\) het \textbf{quotiënt}en \(R(x)\) de \textbf{rest} bij de deling van \(A(x)\) door \(B(x)\). 
    % \end{theorem} 
    
    Indien \( R(x) = 0 \) zeggen de we dat de veelterm \(A(x)\) \textbf{deelbaar} is door de veelterm \(B(x)\). 
  
    \end{oplossing}
\end{question}

\begin{question}
    Bepaal of de veeltermen \(A(x) = 6x^3 + 19x^2 + 8x - 5\) deelbaar is door \(B(x) = 2x+5\) door de staartdeling uit te voeren. 

    \begin{oplossing}
        
    \tikzit{
        \(
        \begin{array}{l|l}
        & 2x+5 \\
        \cline{2-2}
        \vrule height 1.2em width 0pt
        & 3x^2+2x-1 \\[-0.96cm]
        \stackunder[0.05cm]{%
          \stackon[0pt]{6x^3+19x^2+\mph{0}8x-5}{}%
        }{%
          \Shortstack[r]{
            {6x^3+15x^2\mph{+08x-5}} 
            {\staartmin \staartstreep{6x^3+15x^2} \staartphstreep{+10x-50}}
            {4x^2+\mph{0}8x-5} 
            {4x^2+10x\mph{-5}} 
            {\staartmin \staartstreep{4x^2+10x-5}}
            {-2x-5}
            {-2x-5}
            {\staartmin \staartstreep{-2x-5}}
            {0}
        }
        }  
        \end{array}
        \)
        }
        
        Uit dit schema volgt het verband tussen deeltal, deler, quotiënt en rest:
        \[
        \underbrace{6x^3 + 19x^2 + 8x - 5}_{A(x)} = \underbrace{(2x+5)}_{B(x)}\cdot\underbrace{(3x^2+2x-1)}_{Q(x)} \,\, + \,\, \underbrace{0}_{R(x)} 
        \]
        waaruit we besluiten dat \(A(x)\) deelbaar is door \(B(x)\).
    \end{oplossing}

\end{question}

\end{exercise}
\newcommand{\choicemakkelijk}{{\wordChoice{\choice[correct]{makkelijker}\choice{één oplossing}}}}

% Restelling 
\begin{exercise}

    Bepaal de oplossingen van de vierkantsvergelijking \( x^2 - 4x + 5 = 0 \). 
    
    
    \begin{question}
    Het is \choicemakkelijk om de deling uit te voeren van een veelterm door \(x-a\). 
    \begin{feedback}
    

    \end{feedback}
    \end{question}
    
    \begin{question}
    Deze vergelijking heeft \choicegeen  in de reële getallen.
    \begin{feedback}
        Wanneer de discriminant negatief is, zijn er oplossingen in de reële getallen. 
        In de algemene formule staat de discriminant onder een wortel. 
        Er is geen reëel getal waarvan het kwadraat negatief is en je kan dus geen oplossingen bepalen: 
        \[
        x_{1} = \frac{-b + \sqrt{\Delta}}{2a}  \text{ en }  x_{2} = \frac{-b - \sqrt{\Delta}}{2a}
        \]

    \end{feedback}
    \end{question}
    
    \begin{oplossing}
    Voor de vergelijking \( x^2 - 4x + 5 = 0 \)
    
    hebben we:
    \[
    a = 1,\quad b = -4,\quad c = 5.
    \]
    
    De discriminant wordt berekend als:
    \[
    \Delta = b^2 - 4ac = (-4)^2 - 4\cdot1\cdot5 = 16 - 20 = -4.
    \]
    
    Aangezien \(\Delta < 0\) zijn er geen reële oplossingen. 

    De grafiek van de parabool ligt volledig boven de \(x\)-as. Er zijn geen snijpunten. 
    \begin{image}
        \begin{tikzpicture}[scale=1]
            % Teken de assen
            \draw[->] (-1,0) -- (5,0) node[right] {\(x\)};
            \draw[->] (0,-1) -- (0,6) node[above] {\(y\)};
            
            % Teken de parabool f(x) = x^2 - 4x + 5
            \draw[domain=-0.2:4.2, samples=100, smooth, very thick, blue] 
                plot (\x, {(\x)^2 - 4*(\x) + 5});
                
            \draw (2, 5) node[blue] {\( f(x) = x^2 - 4x + 5 \)};
            
        \end{tikzpicture}
    \end{image}

    
    \end{oplossing}
    
\end{exercise}
    

% \begin{expandable}{remark}{De constante factor \(c\)}
% \begin{remark}
    
    
%     In het algemeen voorschrift \( ax^2+bx+c=0 \) geeft de constante term \(c\) een verticale verschuiving weer. 
%     De waarde van \(c\) heeft dus invloed op het aantal snijpunten met de \(x\)-as en hiermee kan je inzien waar die \(c\) ook invloed heeft op het tegen van de discriminant \(\Delta = b^2 - 4ac\). 

% \usepackage{tikz}

%     \begin{image}

%         \begin{scope}[xshift=0cm]
%         \begin{tikzpicture}[scale=1]

%             % Draw axes
%             \draw[->] (-1,0) -- (5,0) node[right] {\(x\)};
%             \draw[->] (0,-1) -- (0,4) node[above] {\(y\)};
            
%             % Draw the parabola y = x^2 - 4x + 3
%             \draw[domain=-0.2:4.2, samples=100, smooth, very thick, blue] 
%                 plot (\x, {(\x)^2 - 4*(\x) + 3}); 
                
%             \draw (2, 3) node[blue] {\( f(x) = x^2 - 4x + 3 \)};
        
%             % Roots (zeros) of the parabola
%             \filldraw[red] (1,0) circle (2pt) node[below] {\(x=1\)};
%             \filldraw[red] (3,0) circle (2pt) node[below] {\(x=3\)};
        
%         \end{tikzpicture}
%         \end{scope}

%         \begin{scope}[xshift=3cm]
%         \begin{tikzpicture}[scale=1]
%             % Teken de assen
%             \draw[->] (-1,0) -- (5,0) node[right] {\(x\)};
%             \draw[->] (0,-1) -- (0,4) node[above] {\(y\)};
            
%             % Teken de parabool f(x) = x^2 - 4x + 4
%             \draw[domain=-0.2:4.2, samples=100, smooth, very thick, blue] 
%                 plot (\x, {(\x)^2 - 4*(\x) + 4});
                
%             \draw (2, 3) node[blue] {\( f(x) = x^2 - 4x + 4 \)};
            
%             % Markeer de dubbele wortel
%             \filldraw[red] (2,0) circle (2pt) node[below] {\(x=2\)};
            
%         \end{tikzpicture}
%         \end{scope}


%         \begin{scope}[xshift=6cm]    
%         \begin{tikzpicture}[scale=1]
%             % Teken de assen
%             \draw[->] (-1,0) -- (5,0) node[right] {\(x\)};
%             \draw[->] (0,-1) -- (0,6) node[above] {\(y\)};
            
%             % Teken de parabool f(x) = x^2 - 4x + 5
%             \draw[domain=-0.2:4.2, samples=100, smooth, very thick, blue] 
%                 plot (\x, {(\x)^2 - 4*(\x) + 5});
                
%             \draw (2, 5) node[blue] {\( f(x) = x^2 - 4x + 5 \)};
            
%         \end{tikzpicture}
%         \end{scope}
        
%     \end{image}
    
    
%     Voor de geïntresseerd leerlingen: de grafiek boven de \(x\)-as met een negatieve discrimant in de vierkantswortel heeft geen oplossingen in de reële getallen. 
%     In de complexe getallen hebben wiskundigen echter wel oplossingen gevonden...

%     \begin{expandable}{youtube}{A very gentle introduction to complex numbers.}
%         \youtube{https://youtu.be/f079K1f2WQk}
%     \end{expandable}   

    
% \end{remark}
% \end{expandable}


\begin{exercise} Bepaal de oplossingen van volgende tweedegraadsvergelijkingen. 
  
    \begin{question} \( x^2 - 4x + 3    = 0 \) \begin{uitkomst} Er zijn twee reële oplossingen \( 1 \) en  \( 3 \)          \end{uitkomst}\end{question}
    \begin{question} \( -2x^2 + 4x + 8  = 0 \) \begin{uitkomst} Er zijn twee reële oplossingen \( -2 \) en  \(4 \)          \end{uitkomst}\end{question}
    \begin{question} \( 2x^2 - 8x + 16  = 0 \) \begin{uitkomst} De dubbele wortel is gelijk aan \( 2 \)                     \end{uitkomst}\end{question}
    \begin{question} \( 2x^2 + 5x + 2   = 0 \) \begin{uitkomst} Er zijn twee reële oplossingen \( -\frac{1}{2}\) en \(-2 \) \end{uitkomst}\end{question}
    \begin{question} \( 3x^2 - 6x + 3   = 0 \) \begin{uitkomst} De dubbele wortel is gelijk aan \( 1   \)                   \end{uitkomst}\end{question}

\end{exercise}


\begin{exercise} Bepaal de oplossingen van volgende tweedegraadsvergelijkingen. 
    
    \begin{question} \( x^2 + 4x + 5    = 0 \) \begin{uitkomst} Er zijn geen oplossingen in de reële getallen.                \end{uitkomst} \end{question}
    \begin{question} \( x^2 + 2x - 8    = 0 \) \begin{uitkomst} Er zijn twee reële oplossingen  \( -4 \) en \( 2 \)           \end{uitkomst} \end{question}
    \begin{question} \( 4x^2 + 12x + 9  = 0 \) \begin{uitkomst} De dubbele wortel is gelijk aan \( -\frac{3}{2} \)            \end{uitkomst} \end{question}
    \begin{question} \( 3x^2 + 6x + 3   = 0 \) \begin{uitkomst} De dubbele wortel is gelijk aan \(  -1 \)                     \end{uitkomst} \end{question}
    \begin{question} \( -x^2 + 4x + 1   = 0 \) \begin{uitkomst} Er zijn twee reële oplossingen  \( -\frac{1}{2} \) en \( 5 \) \end{uitkomst} \end{question}
    
\end{exercise}



\end{document}
