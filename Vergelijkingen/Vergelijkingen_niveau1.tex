\documentclass{ximera}
\input{../preamble}
\addPrintStyle{..}
\begin{document}
	\author{Wiskunde Op Maat}
	\xmtitle{Lineaire vergelijkingen oplossen}{}

   
\begin{exercise} Los volgende lineaire vergelijking op naar de onbekende \(x\).
  \begin{xmmulticols}

    \begin{question} \( 5x - 3  = 2x + 6   \) \begin{oplossing} \( x = 3   \) \end{oplossing} \end{question}
    \begin{question} \( x + 3   = 7        \) \begin{oplossing} \( x = 4   \) \end{oplossing} \end{question}
    \begin{question} \( 4x + 7  = 3x - 2   \) \begin{oplossing} \( x = -9  \) \end{oplossing} \end{question}
    \begin{question} \( x - 5   = 2        \) \begin{oplossing} \( x = 7   \) \end{oplossing} \end{question}
    \begin{question} \( 6x - 4  = 2x + 8   \) \begin{oplossing} \( x = 3   \) \end{oplossing} \end{question}
    \begin{question} \( 2x + 4  = 10       \) \begin{oplossing} \( x = 3   \) \end{oplossing} \end{question}
    \begin{question} \( 7x + 5  = 3x + 17  \) \begin{oplossing} \( x = 3   \) \end{oplossing} \end{question}
    \begin{question} \( 3x - 6  = 9        \) \begin{oplossing} \( x = 5   \) \end{oplossing} \end{question}
    \begin{question} \( 8x - 6  = 2x + 12  \) \begin{oplossing} \( x = 3   \) \end{oplossing} \end{question}
    \begin{question} \( 5x + 2  = 17       \) \begin{oplossing} \( x = 3   \) \end{oplossing} \end{question}
    \begin{question} \( 10x + 4 = 6x + 24  \) \begin{oplossing} \( x = 5   \) \end{oplossing} \end{question}
    \begin{question} \( 4x - 8  = 12       \) \begin{oplossing} \( x = 5   \) \end{oplossing} \end{question}
    \begin{question} \( 9x - 3  = 2x + 11  \) \begin{oplossing} \( x = 2   \) \end{oplossing} \end{question}
    \begin{question} \( 6x + 3  = 21       \) \begin{oplossing} \( x = 3   \) \end{oplossing} \end{question}
    \begin{question} \( 12x + 8 = 4x + 24  \) \begin{oplossing} \( x = 2   \) \end{oplossing} \end{question}
    \begin{question} \( 7x - 2  = 19       \) \begin{oplossing} \( x = 3   \) \end{oplossing} \end{question}
    \begin{question} \( 5x + 3  = 2x + 12  \) \begin{oplossing} \( x = 3   \) \end{oplossing} \end{question}
    \begin{question} \( 9x + 5  = 50       \) \begin{oplossing} \( x = 5   \) \end{oplossing} \end{question}
    \begin{question} \( 4x - 7  = 3x + 1   \) \begin{oplossing} \( x = 8   \) \end{oplossing} \end{question}
    \begin{question} \( 8x - 4  = 28       \) \begin{oplossing} \( x = 4   \) \end{oplossing} \end{question}
    \begin{question} \( 11x + 9 = 5x + 21  \) \begin{oplossing} \( x = 2   \) \end{oplossing} \end{question}
    \begin{question} \( x + 7   = 12       \) \begin{oplossing} \( x = 5   \) \end{oplossing} \end{question}
    \begin{question} \( 3x - 5  = x + 7    \) \begin{oplossing} \( x = 6   \) \end{oplossing} \end{question}
    \begin{question} \( x - 9   = -2       \) \begin{oplossing} \( x = 7   \) \end{oplossing} \end{question}
    \begin{question} \( 2x + 5  = 11       \) \begin{oplossing} \( x = 3   \) \end{oplossing} \end{question}
    \begin{question} \( 7x + 2  = 3x + 10  \) \begin{oplossing} \( x  = 2  \) \end{oplossing} \end{question} 
    \begin{question} \( 7x + 4  = 25       \) \begin{oplossing} \( x  = 3  \) \end{oplossing} \end{question}
    \begin{question} \( 9x - 8  = 4x + 7   \) \begin{oplossing} \( x  = 3  \) \end{oplossing} \end{question} 
    \begin{question} \( x + 4   = 9        \) \begin{oplossing} \( x  = 5  \) \end{oplossing} \end{question}
    \begin{question} \( 2x + 6  = 5x - 9   \) \begin{oplossing} \( x  = 5  \) \end{oplossing} \end{question} 
    
  \end{xmmulticols}
\end{exercise}  
    
\begin{exercise} Los volgende lineaire vergelijking op naar de onbekende \(x\).
  \begin{xmmulticols}   
    
    \begin{question} \( 4x - 5  = 11       \) \begin{oplossing} \( x  = 4  \) \end{oplossing} \end{question}
    \begin{question} \( 6x - 2  = 3x + 10  \) \begin{oplossing} \( x  = 4  \) \end{oplossing} \end{question} 
    \begin{question} \( 6x - 7  = 11       \) \begin{oplossing} \( x  = 3  \) \end{oplossing} \end{question}
    \begin{question} \( 8x + 5  = 2x + 23  \) \begin{oplossing} \( x  = 3  \) \end{oplossing} \end{question} 
    \begin{question} \( 3x - 5  = 10       \) \begin{oplossing} \( x  = 5  \) \end{oplossing} \end{question}
    \begin{question} \( 10x - 3 = 4x + 15  \) \begin{oplossing} \( x  = 3  \) \end{oplossing} \end{question} 
    \begin{question} \( 8x - 3  = 29       \) \begin{oplossing} \( x  = 4  \) \end{oplossing} \end{question}
    \begin{question} \( 7x + 4  = 2x + 19  \) \begin{oplossing} \( x  = 3  \) \end{oplossing} \end{question} 
    \begin{question} \( 2x + 7  = 13       \) \begin{oplossing} \( x  = 3  \) \end{oplossing} \end{question}
    \begin{question} \( 5x - 6  = 2x + 9   \) \begin{oplossing} \( x  = 5  \) \end{oplossing} \end{question} 
    \begin{question} \( 3x + 7  = x + 15   \) \begin{oplossing} \( x  = 4  \) \end{oplossing} \end{question} 
    \begin{question} \( 10x + 2 = 32       \) \begin{oplossing} \( x  = 3  \) \end{oplossing} \end{question}
    \begin{question} \( 12x - 5 = 7x + 10  \) \begin{oplossing} \( x  = 3  \) \end{oplossing} \end{question} 
    \begin{question} \( x - 6   = -1       \) \begin{oplossing} \( x  = 5  \) \end{oplossing} \end{question}
    \begin{question} \( 4x + 9  = 2x + 19  \) \begin{oplossing} \( x  = 5  \) \end{oplossing} \end{question} 
    \begin{question} \( 4x + 8  = 24       \) \begin{oplossing} \( x  = 4  \) \end{oplossing} \end{question}
    \begin{question} \( 11x + 8 = 5x + 26  \) \begin{oplossing} \( x  = 3  \) \end{oplossing} \end{question} 
    \begin{question} \( 3x - 4  = 8        \) \begin{oplossing} \( x  = 4  \) \end{oplossing} \end{question}
    \begin{question} \( 6x - 4  = 2x + 12  \) \begin{oplossing} \( x  = 4  \) \end{oplossing} \end{question} 
    \begin{question} \( 5x + 6  = 26       \) \begin{oplossing} \( x  = 4  \) \end{oplossing} \end{question}
    
  \end{xmmulticols}
\end{exercise}

\end{document}